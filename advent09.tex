% advent09.tex
% Hero day: December 9, 2025 – 137-day (fine-structure constant)

\documentclass[a4paper,10pt]{article}

\usepackage[utf8]{inputenc}
\usepackage[T1]{fontenc}
\usepackage[english]{babel}
\usepackage{amsmath,amssymb,amsfonts}
\usepackage{xcolor}
\usepackage{tikz}
\usetikzlibrary{calc}
\usepackage[framemethod=TikZ]{mdframed}
\usepackage{titlesec}
\usepackage{lettrine}
\usepackage{eso-pic}
\usepackage{geometry}
\usepackage{multicol}
\usepackage{lmodern}

\geometry{margin=2.0cm}

% advent-layout.tex (corrected for \input usage)

\definecolor{adventred}{HTML}{B3001B}
\definecolor{adventblue}{HTML}{003366}
\definecolor{adventgreen}{HTML}{006633}
\definecolor{adventgold}{HTML}{B59410}

\pagestyle{empty}

\titleformat{\section}
  {\normalfont\large\bfseries\color{adventblue}}{\thesection}{0em}{}
  
\titleformat{\subsection}
  {\normalfont\normalsize\bfseries\color{adventblue}}{\thesubsection}{0em}{}

%----
% AdventFrameTop
%----

\newenvironment{AdventFrameTop}
{%
  \begin{mdframed}[
    linecolor=adventgreen!0,
    linewidth=0pt,
    roundcorner=0pt,
    innertopmargin=10pt,
    innerbottommargin=10pt,
    innerleftmargin=10pt,
    innerrightmargin=10pt,
    backgroundcolor=adventgreen!2
  ]%
}
{%
  \end{mdframed}
}

\newcommand{\BeginAdventPage}{}
\newcommand{\EndAdventPage}{}

%----
% AdventTitleBlock
%----

\newcommand{\AdventTitleBlock}[4]{%
  \begin{center}
    {\Large\textcolor{adventred}{\textbf{#1}}}\par\vspace{4pt}%
    \ifx&#2&\else
      {\large\textbf{#2}}\par\vspace{2pt}%
    \fi
    {\Large\textcolor{adventblue}{\textbf{#3}}}\par
    \ifx&#4&\else
      \vspace{2pt}%
      {\normalsize\textbf{#4}}\par%
    \fi
  \end{center}%
}

%----
% AdventKeyInsight
%----

\newcommand{\AdventKeyInsight}[1]{%
  \vspace{0.5em}%
  \noindent\colorbox{adventred!8}{%
    \parbox{\dimexpr\linewidth-2\fboxsep}{%
    \textbf{\textcolor{adventred}{Key Insight.}}~#1%
    }%
  }%
  \vspace{0.5em}%
}

%----
% AdventStarRule
%----

\newcommand{\AdventStarRule}{%
  \vspace{0.3em}%
  \begin{center}
    {\color{adventgold}%
    \rule[0.5ex]{0.25\linewidth}{0.4pt}\;
    $\ast\;\ast\;\ast$\;
    \rule[0.5ex]{0.25\linewidth}{0.4pt}%
    }%
  \end{center}
  \vspace{0.3em}%
}

%----
% AdventClosing
%----

\newcommand{\AdventClosing}[1]{%
  \vspace{0.4em}%
  \begin{center}
    \textcolor{adventgreen}{\emph{#1}}%
  \end{center}
}

%----
% AdventAuthor
%----

\newcommand{\AdventAuthor}{%
  \AddToShipoutPictureFG{%
    \begin{tikzpicture}[remember picture,overlay]
    \node[anchor=south, yshift=2mm] at (current page.south) {%
    \footnotesize Andreas Müller, Kempten University of Applied Sciences, %
    \texttt{andreas.mueller@hs-kempten.de}%
    };
    \end{tikzpicture}%
  }%
}

%----
% AdventInitial
%----

\newcommand{\AdventInitial}[2]{%
  \lettrine[lines=2,lhang=0.1,loversize=0.15]%
    {\textcolor{adventred}{#1}}%
    {#2}%
}

%----
% AdventPageBackground
%----

\newcommand{\AdventPageBackground}{%
  \AddToShipoutPictureBG{%
    \begin{tikzpicture}[remember picture,overlay]
    \draw[adventgreen!80!black, line width=3pt, rounded corners=12pt]
    ($(current page.north west)+(0.8cm,-0.8cm)$)
    rectangle
    ($(current page.south east)+(-0.8cm,0.8cm)$);
    \fill[adventgold]
    ($(current page.north west)+(1.0cm,-1.0cm)$) circle (1.2pt)
    ($(current page.north east)+(-1.0cm,-1.0cm)$) circle (1.2pt)
    ($(current page.south west)+(1.0cm,1.0cm)$) circle (1.2pt)
    ($(current page.south east)+(-1.0cm,1.0cm)$) circle (1.2pt);
    \end{tikzpicture}
  }%
}

%---- AdventSheet macro (1-page, single column) ----
% #1: Date + occasion
% #2: unused
% #3: Main title
% #4: Subtitle
% #5: Key Insight
% #6: Main body content
% #7: Closing statement
\newcommand{\AdventSheet}[7]{%
  \BeginAdventPage
  \vspace*{1cm}
  \begin{AdventFrameTop}
    \AdventTitleBlock{#1}{#2}{#3}{#4}
    \AdventKeyInsight{#5}
  \end{AdventFrameTop}
  \AdventStarRule
  #6%
  \EndAdventPage
  \AdventClosing{#7}%
}

%---- AdventSheetTwoCol macro (two-column hero sheet) ----
% #1: Date + occasion
% #2: unused
% #3: Main title
% #4: Subtitle
% #5: Key Insight
% #6: Main body content (wrapped in multicols{2})
% #7: Closing statement
\newcommand{\AdventSheetTwoCol}[7]{%
  \BeginAdventPage
  \begin{AdventFrameTop}
    \AdventTitleBlock{#1}{#2}{#3}{#4}
    \AdventKeyInsight{#5}
  \end{AdventFrameTop}
  \AdventStarRule
  \begin{multicols}{2}
    #6%
  \end{multicols}
  \EndAdventPage
  \AdventClosing{#7}%
}

\begin{document}

\AdventPageBackground
\AdventAuthor

\AdventSheetTwoCol
  {December 9, 2025} % #1 Date + occasion
  {}                                   % #2 unused
  {Why $\alpha \approx 1/137$?}        % #3 Main title
  {Fine-structure from rotor norms}    % #4 Subtitle
  {The fine-structure constant $\alpha$ does not enter as a free
   input parameter. In the octonionic model, $\alpha$ is understood
   as the squared norm of a specific rotor commutator in the internal
   geometry. The same mechanism also fixes the strong coupling and the
   weak mixing angle. The famous number $1/137$ becomes a geometric
   shadow of how internal directions are arranged in the exceptional
   algebra.}                             % #5 Key Insight
  { % #6 Main body (two columns)

\AdventInitial{F}{ew} numbers in physics have captured the imagination quite like the fine-structure constant, usually denoted $\alpha$. Its value is approximately $1/137.036$, and it controls the strength of electromagnetic interactions. It determines how tightly electrons are bound to nuclei, how bright stars shine, and how fast atoms emit light. It's one of the most precisely measured numbers in all of science.

But where does it come from? In conventional quantum field theory, $\alpha$ is simply a parameter—an input to the theory that we measure experimentally but don't derive from anything deeper. Richard Feynman famously called it "one of the greatest damn mysteries of physics," and for decades, physicists and even numerologists have tried to find a closed-form expression for $1/137$ in terms of fundamental mathematical constants like $\pi$ or $e$. None of these attempts have succeeded in a convincing way.

What if we're asking the wrong question? What if $\alpha$ is not a magic number waiting to be decoded, but a geometric quantity—a measure of how certain internal directions in an exceptional algebra are oriented relative to each other?

This is the claim of today's sheet. In the octonionic model, $\alpha$ is not a free parameter. It emerges as the squared norm of a commutator built from internal \textbf{rotors}—the antisymmetric operators we met two days ago that generate rotations in the internal space. The value of $\alpha$ is tied to the geometry: specifically, to how the internal directions associated with electromagnetism are arranged within the larger exceptional structure.

Here's how it works. The internal space of the model sits inside the exceptional Lie algebra associated with the Albert algebra $H_3(\mathbb{O})$ and its symmetry group $F_4$. Within this algebra, we can identify special operators $R_a$ and $R_b$ that act as rotors on selected internal subspaces—think of them as generators of rotations in the ``electromagnetic plane'' and a ``reference plane.''

The key object is the commutator $[R_a, R_b]$. If the two rotors commute—if rotating in one direction and then the other gives the same result as doing it in the opposite order—then the commutator is zero, and there's no interaction. But if they don't commute, the commutator is nonzero, and its norm measures how strongly the two directions ``clash.'' This norm becomes the coupling constant:

$\alpha \propto \|[R_a, R_b]\|^2.$

Now, you might object: ``Couldn't you tune this norm to be anything you want?'' The answer is no—not in the octonionic setting. The rotors are highly constrained by the rigid structure of the algebra:
\begin{itemize}
\item Triality and $G_2$-compatibility fix how vector and spinor directions in $\mathbb{R}^8$ are related.
\item The Jordan structure of $H_3(\mathbb{O})$ restricts which combinations of internal directions can appear as ``legal'' rotors.
\item The vacuum configuration $\langle H \rangle$ selects preferred eigen-directions and thereby preferred planes in which the rotors act.
\end{itemize}

Under these constraints, the norm $\|[R_a, R_b]\|^2$ is not a continuously tunable parameter. It falls into discrete bands determined by the eigenvalues of certain internal operators. One of these bands lands numerically in the vicinity of $1/137$, and this is identified with the observed fine-structure constant.

The same construction works for other gauge couplings. Rotors associated with the $SU(3)$ color directions lead to the strong coupling $\alpha_s$. Rotors associated with weak isospin and hypercharge lead to the weak mixing angle $\sin^2\theta_W$. All three couplings—electromagnetic, strong, weak—arise as norms of commutators of rotors in the same internal algebra. The observed pattern of couplings is a fingerprint of how the vacuum sits inside the exceptional geometry.

Of course, $\alpha$ is not a rigid number—it runs with energy scale due to quantum corrections. The octonionic model doesn't deny this. Instead, it separates two roles: the {\em bare geometric value}, determined by the internal rotor norm at a natural reference scale, and the {\em renormalized value} seen in experiments at a given energy, obtained by standard running from that reference scale.

If this picture is correct, then the traditional question ``Why $1/137$?'' shifts its focus. Instead of asking for a closed-form expression in terms of $\pi$ and $e$, we ask: ``Why does the vacuum select exactly this arrangement of internal directions in the exceptional algebra?'' The mystery moves from ``a magic number'' to ``a specific geometric configuration.'' This may or may not be more satisfying philosophically, but it's technically more tractable: you can compute norms of commutators, study their spectra, and connect them to the attractor dynamics we'll explore later in the calendar.

Think of today's sheet as a shift in perspective: the famous $1/137$ is not an inexplicable constant—it's a geometric shadow of how internal rotations are arranged in an exceptional space.

Let us see how rotor norms turn into coupling constants.

\section*{From mysterious constant to geometric norm}

Not many numbers in physics have attracted as much fascination
as the fine-structure constant,
\[
  \alpha \;=\; \frac{e^2}{4\pi\varepsilon_0\hbar c}
  \;\approx\; \frac{1}{137.035999\ldots}.
\]
Historically, $\alpha$ enters quantum electrodynamics as a dimensionless
coupling: it controls the strength of electromagnetic interactions and
the convergence of perturbation theory. For decades, it has been treated
as a parameter to be measured, not derived.

In the octonionic model, $\alpha$ is no longer a free parameter. It
appears as the squared norm of a commutator built from internal
\emph{rotors} in the exceptional geometry. The value of $\alpha$ is
tied to how certain planes inside the octonionic/Albert algebra are
oriented with respect to each other.

\section*{Internal rotors and commutators}

The internal space of the model is not a simple Lie algebra like
$\mathfrak{su}(2)$ or $\mathfrak{su}(3)$ in isolation. Instead, it sits
inside the exceptional Lie algebra associated with the Albert algebra
$H_3(\mathbb{O})$ and its symmetry group $F_4$.

Within this setting, one considers special operators $R_a$, $R_b$, \dots,
which act as \emph{rotors} on selected internal subspaces. Schematically,
one writes
\[
  R_a \;\sim\; \exp(\theta_a X_a),
  \qquad
  R_b \;\sim\; \exp(\theta_b X_b),
\]
where $X_a$, $X_b$ are generators associated with particular internal
directions (for instance, those tied to weak isospin and hypercharge).

The key object is the commutator
\[
  [R_a, R_b]
  \;\approx\;
  \theta_a\theta_b [X_a,X_b]
\]
in an appropriate small-angle limit. The \emph{norm} of this commutator
defines an effective coupling:
\[
  \alpha \;\propto\; \bigl\|[R_a,R_b]\bigr\|^2.
\]
Choosing $R_a$ and $R_b$ according to the embedding of $U(1)_{\text{em}}$
inside the internal algebra singles out the electromagnetic coupling.

\section*{From abstract norm to a concrete number}

The statement that $\alpha$ is given by a squared norm would be empty if
the norm could be tuned at will. The nontrivial part is that in the
octonionic/Albert setting the relevant rotors are highly constrained:

\begin{itemize}
  \item Triality and $G_2$-compatibility fix how vector and spinor
        directions in $\mathbb{R}^8$ are related.
  \item The Jordan structure of $H_3(\mathbb{O})$ restricts which
        combinations of internal directions can appear as ``legal''
        rotors.
  \item The vacuum configuration $\langle H\rangle$ selects preferred
        eigen-directions and thereby preferred planes in which the
        rotors act.
\end{itemize}

Under these constraints, the norm $\bigl\|[R_a,R_b]\bigr\|^2$ is not a
continuously tunable parameter. It falls into discrete bands determined
by the eigenvalues of certain internal projectors and compressor
operators. One of these bands lands numerically in the vicinity of
$1/137$, and this is identified with the observed fine-structure
constant at a particular reference scale.

\section*{Relations to strong and weak couplings}

The same construction can be repeated for other choices of rotors:

\begin{itemize}
  \item Rotors associated with the $SU(3)$ color directions lead to an
        effective strong coupling $\alpha_s$.
  \item Rotors associated with the weak isospin and hypercharge mixture
        lead to $\sin^2\theta_W$, the weak mixing angle.
\end{itemize}

The guiding principle is that \emph{all} gauge couplings arise as norms
of commutators of rotors in the same internal algebra. This leads to
relations of the symbolic form
\[
  \alpha \;\sim\; \|[R_{\text{em}},R_{\text{ref}}]\|^2,
  \qquad
  \alpha_s \;\sim\; \|[R_{\text{color}},R_{\text{ref}}]\|^2,
  \]\[
  \sin^2\theta_W \;\sim\; \|[R_{\text{weak}},R_{\text{ref}}]\|^2,
\]
where $R_{\text{ref}}$ is a common reference rotor set by the vacuum
configuration in $H_3(\mathbb{O})$.

In this picture, the observed pattern of couplings is a fingerprint of
how the vacuum sits inside the exceptional algebra, not a list of
independent constants.

\section*{Scaling and running}

Of course, $\alpha$ is not a rigid number: in quantum field theory it
runs with energy scale. The octonionic model does not deny this; instead
it separates two roles:

\begin{enumerate}
  \item The \emph{bare geometric value}, determined by the internal
        rotor norm at a natural reference scale.
  \item The \emph{renormalized value} seen in experiments at a given
        energy, obtained by standard running from that reference scale.
\end{enumerate}

The contribution of the geometry is to fix the starting point and the
relative pattern of couplings. The quantum field theoretic machinery of
running and thresholds then dresses these values to the ones we measure.

\section*{A different attitude towards constants}

If $\alpha$ comes from an internal rotor norm, then the traditional
question ``Why $1/137$\,?'' shifts its focus. Instead of asking for a
closed-form expression in terms of $\pi$ and $e$ alone, we ask:

\begin{quote}
  Why does the vacuum select exactly this arrangement of internal
  directions in the exceptional algebra?
\end{quote}

In other words, the mystery moves from ``a magic number'' to
``a specific geometric configuration''. This may or may not be more
satisfying philosophically, but it is technically more tractable: one
can compute norms of commutators, study their spectra, and connect them
to the attractor dynamics discussed later in the Advent series.

\small
\begin{thebibliography}{9}

\bibitem{Dyson1968}
F.~Dyson,
\newblock ``The role of the fine-structure constant in physics,''
\newblock {\em American Journal of Physics} \textbf{58}, 209--211 (1968).

\bibitem{Furey2018}
C.~Furey,
\newblock ``$SU(3)_C \times SU(2)_L \times U(1)_Y$ from division algebras,''
\newblock {\em Phys.\ Lett.\ B} \textbf{785}, 84--89 (2018).

\bibitem{GurseyTze1996}
F.~Gürsey and H.~C.~Tze,
\newblock {\em On the Role of Division, Jordan and Related Algebras in Particle
Physics},
\newblock World Scientific, 1996.

\end{thebibliography}
\normalsize

  } % end #6 body
  {The famous $1/137$ is read as a norm in an exceptional internal geometry, not as an inexplicable magic number.} % #7 Closing

\end{document}