% advent29.tex
% December 20, 2025 – Dark matter as a shadow sector of compressors

\documentclass[a4paper,10pt]{article}

\usepackage[utf8]{inputenc}
\usepackage[T1]{fontenc}
\usepackage[english]{babel}
\usepackage{amsmath,amssymb,amsfonts}
\usepackage{xcolor}
\usepackage{tikz}
\usetikzlibrary{calc}
\usepackage[framemethod=TikZ]{mdframed}
\usepackage{titlesec}
\usepackage{lettrine}
\usepackage{eso-pic}
\usepackage{geometry}
\usepackage{multicol}
\usepackage{lmodern}

\geometry{margin=2.0cm}

% advent-layout.tex (corrected for \input usage)

\definecolor{adventred}{HTML}{B3001B}
\definecolor{adventblue}{HTML}{003366}
\definecolor{adventgreen}{HTML}{006633}
\definecolor{adventgold}{HTML}{B59410}

\pagestyle{empty}

\titleformat{\section}
  {\normalfont\large\bfseries\color{adventblue}}{\thesection}{0em}{}
  
\titleformat{\subsection}
  {\normalfont\normalsize\bfseries\color{adventblue}}{\thesubsection}{0em}{}

%----
% AdventFrameTop
%----

\newenvironment{AdventFrameTop}
{%
  \begin{mdframed}[
    linecolor=adventgreen!0,
    linewidth=0pt,
    roundcorner=0pt,
    innertopmargin=10pt,
    innerbottommargin=10pt,
    innerleftmargin=10pt,
    innerrightmargin=10pt,
    backgroundcolor=adventgreen!2
  ]%
}
{%
  \end{mdframed}
}

\newcommand{\BeginAdventPage}{}
\newcommand{\EndAdventPage}{}

%----
% AdventTitleBlock
%----

\newcommand{\AdventTitleBlock}[4]{%
  \begin{center}
    {\Large\textcolor{adventred}{\textbf{#1}}}\par\vspace{4pt}%
    \ifx&#2&\else
      {\large\textbf{#2}}\par\vspace{2pt}%
    \fi
    {\Large\textcolor{adventblue}{\textbf{#3}}}\par
    \ifx&#4&\else
      \vspace{2pt}%
      {\normalsize\textbf{#4}}\par%
    \fi
  \end{center}%
}

%----
% AdventKeyInsight
%----

\newcommand{\AdventKeyInsight}[1]{%
  \vspace{0.5em}%
  \noindent\colorbox{adventred!8}{%
    \parbox{\dimexpr\linewidth-2\fboxsep}{%
    \textbf{\textcolor{adventred}{Key Insight.}}~#1%
    }%
  }%
  \vspace{0.5em}%
}

%----
% AdventStarRule
%----

\newcommand{\AdventStarRule}{%
  \vspace{0.3em}%
  \begin{center}
    {\color{adventgold}%
    \rule[0.5ex]{0.25\linewidth}{0.4pt}\;
    $\ast\;\ast\;\ast$\;
    \rule[0.5ex]{0.25\linewidth}{0.4pt}%
    }%
  \end{center}
  \vspace{0.3em}%
}

%----
% AdventClosing
%----

\newcommand{\AdventClosing}[1]{%
  \vspace{0.4em}%
  \begin{center}
    \textcolor{adventgreen}{\emph{#1}}%
  \end{center}
}

%----
% AdventAuthor
%----

\newcommand{\AdventAuthor}{%
  \AddToShipoutPictureFG{%
    \begin{tikzpicture}[remember picture,overlay]
    \node[anchor=south, yshift=2mm] at (current page.south) {%
    \footnotesize Andreas Müller, Kempten University of Applied Sciences, %
    \texttt{andreas.mueller@hs-kempten.de}%
    };
    \end{tikzpicture}%
  }%
}

%----
% AdventInitial
%----

\newcommand{\AdventInitial}[2]{%
  \lettrine[lines=2,lhang=0.1,loversize=0.15]%
    {\textcolor{adventred}{#1}}%
    {#2}%
}

%----
% AdventPageBackground
%----

\newcommand{\AdventPageBackground}{%
  \AddToShipoutPictureBG{%
    \begin{tikzpicture}[remember picture,overlay]
    \draw[adventgreen!80!black, line width=3pt, rounded corners=12pt]
    ($(current page.north west)+(0.8cm,-0.8cm)$)
    rectangle
    ($(current page.south east)+(-0.8cm,0.8cm)$);
    \fill[adventgold]
    ($(current page.north west)+(1.0cm,-1.0cm)$) circle (1.2pt)
    ($(current page.north east)+(-1.0cm,-1.0cm)$) circle (1.2pt)
    ($(current page.south west)+(1.0cm,1.0cm)$) circle (1.2pt)
    ($(current page.south east)+(-1.0cm,1.0cm)$) circle (1.2pt);
    \end{tikzpicture}
  }%
}

%---- AdventSheet macro (1-page, single column) ----
% #1: Date + occasion
% #2: unused
% #3: Main title
% #4: Subtitle
% #5: Key Insight
% #6: Main body content
% #7: Closing statement
\newcommand{\AdventSheet}[7]{%
  \BeginAdventPage
  \vspace*{1cm}
  \begin{AdventFrameTop}
    \AdventTitleBlock{#1}{#2}{#3}{#4}
    \AdventKeyInsight{#5}
  \end{AdventFrameTop}
  \AdventStarRule
  #6%
  \EndAdventPage
  \AdventClosing{#7}%
}

%---- AdventSheetTwoCol macro (two-column hero sheet) ----
% #1: Date + occasion
% #2: unused
% #3: Main title
% #4: Subtitle
% #5: Key Insight
% #6: Main body content (wrapped in multicols{2})
% #7: Closing statement
\newcommand{\AdventSheetTwoCol}[7]{%
  \BeginAdventPage
  \begin{AdventFrameTop}
    \AdventTitleBlock{#1}{#2}{#3}{#4}
    \AdventKeyInsight{#5}
  \end{AdventFrameTop}
  \AdventStarRule
  \begin{multicols}{2}
    #6%
  \end{multicols}
  \EndAdventPage
  \AdventClosing{#7}%
}

\begin{document}

\AdventPageBackground
\AdventAuthor

\AdventSheetTwoCol
  {December 19, 2025} % #1 Date
  {}                 % #2 unused
  {Scales as attractor fixed points} % #3 Main title
  {Why nature ``chooses'' exactly three stable scales} % #4 Subtitle
  {The three fundamental scales (Planck, electroweak, QCD) are not arbitrary
   inputs but fixed points of an attractor mechanism linked to the radius
   spectrum $(a_0,b_0,c_0)$. Renormalisation-group (RG) flow drives couplings
   and masses towards these scales, which are stable under perturbations.
   Nature ``chooses'' them because they are dynamically preferred, not because
   they were inserted by hand.} % #5 Key Insight
  { % #6 body

\section*{From radius spectrum to preferred scales}

\AdventInitial{E}{arlier} in the calendar, the radius operator $R$ was
introduced with spectrum

\[
  \mathrm{spec}(R) = (a_0,b_0,c_0).
\]

Exponentials of these radii define three candidate scales,

\[
  \Lambda_i \sim e^{a_0},\quad
  \Lambda_j \sim e^{b_0},\quad
  \Lambda_k \sim e^{c_0},
\]

which were associated with the Planck, electroweak and QCD scales. So far
this was a geometric correspondence. Today we add dynamics: these scales
are not only present in the geometry, they are \emph{dynamically
selected} by RG flow.

\section*{RG flow and attractors}

In Wilsonian renormalisation, a theory is described by a point in the
space of couplings. Changing the energy scale $\mu$ corresponds to moving
along a trajectory governed by $\beta$-functions,

\[
  \frac{d g_i}{d\ln\mu} = \beta_i(\{g\}).
\]

Fixed points are defined by $\beta_i(\{g^\ast\})=0$ and play a central role:

\begin{itemize}
  \item UV fixed points control short-distance behaviour.
  \item IR fixed points control long-distance behaviour.
  \item Some fixed points are \emph{attractors}: generic RG trajectories
        are drawn towards them.
\end{itemize}

In the octonionic model, the radius spectrum $(a_0,b_0,c_0)$ enters the
$\beta$-functions in a structured way, so that three distinguished
energies become attractor scales.

\section*{Qualitative picture of the attractor mechanism}

Heuristically, one finds:

\begin{itemize}
  \item When $\mu$ is near $e^{a_0}$, RG flow of certain couplings slows
        down and stabilises; this defines a Planck-like scale.
  \item Near $e^{b_0}$, electroweak couplings and masses experience a
        similar slowdown and stabilisation.
  \item Near $e^{c_0}$, the strong coupling crosses order one and the
        theory flows towards a confined regime.
\end{itemize}

The detailed form of $\beta$-functions is model-dependent, but the
location and number of attractor scales are dictated by $(a_0,b_0,c_0)$.
Arbitrary additional scales cannot be introduced without spoiling this
structure.

\section*{Stability instead of fine-tuning}

Traditional fine-tuning problems (``Why is the electroweak scale so small
compared to the Planck scale?'') are rephrased:

\begin{quote}
  Why are there stable RG attractors at such widely separated scales?
\end{quote}

The model answers:

\begin{itemize}
  \item Because $(a_0,b_0,c_0)$ are widely separated eigenvalues of $R$.
  \item Because RG flow is structured in such a way that these eigenvalues
        generate stable fixed points.
\end{itemize}

The hierarchy is no longer a coincidence between unrelated parameters but
a built-in feature of the internal operator spectrum.

\section*{Conceptual and phenomenological implications}

Conceptually:

\begin{enumerate}
  \item \textbf{Fewer inputs:}
    instead of three independent scale parameters, one internal operator
    with three eigenvalues suffices.
  \item \textbf{Dynamical selection:}
    scales are not only present in the geometry but are dynamically
    attractive under RG flow.
  \item \textbf{Predictive structure:}
    any attempt to introduce new fundamental scales must explain how they
    fit into, or modify, the attractor pattern.
\end{enumerate}

Phenomenologically:

\begin{itemize}
  \item Small variations in UV initial conditions do not destroy the
        large-scale structure: the flow is drawn back to the attractors.
  \item This robustness is precisely what one expects from a realistic
        description of the universe: observed scales should not be
        hypersensitive to microscopic details.
\end{itemize}

\small
\begin{thebibliography}{9}

\bibitem{Wilson1971}
K.~G.~Wilson,
\newblock ``Renormalization group and critical phenomena,''
\newblock {\em Phys.\ Rev.\ B} \textbf{4}, 3174--3183 (1971).

\bibitem{Wetterich1988}
C.~Wetterich,
\newblock ``Cosmology and the fate of dilatation symmetry,''
\newblock {\em Nucl.\ Phys.\ B} \textbf{302}, 668--696 (1988).

\bibitem{Internal}
[Internal notes on scale hierarchies and attractor mechanisms:
{\tt chap16A\_neu.tex; appK\_neu.tex; konstanten-hierarchie.tex}.]

\end{thebibliography}
\normalsize

  } % end #6 body
  {The triple $(a_0,b_0,c_0)$ of radius eigenvalues does not only suggest
   three important scales; it also defines RG attractors. Planck,
   electroweak and QCD scales are dynamically preferred fixed points,
   not arbitrary external inputs.} % #7 Closing


\end{document}