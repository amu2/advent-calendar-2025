% advent05.tex
% December 5, 2025 – Radius operator: scales as geometric invariants

\documentclass[a4paper,10pt]{article}

\usepackage[utf8]{inputenc}
\usepackage[T1]{fontenc}
\usepackage[english]{babel}
\usepackage{amsmath,amssymb,amsfonts}
\usepackage{xcolor}
\usepackage{tikz}
\usetikzlibrary{calc}
\usepackage[framemethod=TikZ]{mdframed}
\usepackage{titlesec}
\usepackage{lettrine}
\usepackage{eso-pic}
\usepackage{geometry}
\usepackage{multicol}
\usepackage{lmodern}

\geometry{margin=2.0cm}

% advent-layout.tex (corrected for \input usage)

\definecolor{adventred}{HTML}{B3001B}
\definecolor{adventblue}{HTML}{003366}
\definecolor{adventgreen}{HTML}{006633}
\definecolor{adventgold}{HTML}{B59410}

\pagestyle{empty}

\titleformat{\section}
  {\normalfont\large\bfseries\color{adventblue}}{\thesection}{0em}{}
  
\titleformat{\subsection}
  {\normalfont\normalsize\bfseries\color{adventblue}}{\thesubsection}{0em}{}

%----
% AdventFrameTop
%----

\newenvironment{AdventFrameTop}
{%
  \begin{mdframed}[
    linecolor=adventgreen!0,
    linewidth=0pt,
    roundcorner=0pt,
    innertopmargin=10pt,
    innerbottommargin=10pt,
    innerleftmargin=10pt,
    innerrightmargin=10pt,
    backgroundcolor=adventgreen!2
  ]%
}
{%
  \end{mdframed}
}

\newcommand{\BeginAdventPage}{}
\newcommand{\EndAdventPage}{}

%----
% AdventTitleBlock
%----

\newcommand{\AdventTitleBlock}[4]{%
  \begin{center}
    {\Large\textcolor{adventred}{\textbf{#1}}}\par\vspace{4pt}%
    \ifx&#2&\else
      {\large\textbf{#2}}\par\vspace{2pt}%
    \fi
    {\Large\textcolor{adventblue}{\textbf{#3}}}\par
    \ifx&#4&\else
      \vspace{2pt}%
      {\normalsize\textbf{#4}}\par%
    \fi
  \end{center}%
}

%----
% AdventKeyInsight
%----

\newcommand{\AdventKeyInsight}[1]{%
  \vspace{0.5em}%
  \noindent\colorbox{adventred!8}{%
    \parbox{\dimexpr\linewidth-2\fboxsep}{%
    \textbf{\textcolor{adventred}{Key Insight.}}~#1%
    }%
  }%
  \vspace{0.5em}%
}

%----
% AdventStarRule
%----

\newcommand{\AdventStarRule}{%
  \vspace{0.3em}%
  \begin{center}
    {\color{adventgold}%
    \rule[0.5ex]{0.25\linewidth}{0.4pt}\;
    $\ast\;\ast\;\ast$\;
    \rule[0.5ex]{0.25\linewidth}{0.4pt}%
    }%
  \end{center}
  \vspace{0.3em}%
}

%----
% AdventClosing
%----

\newcommand{\AdventClosing}[1]{%
  \vspace{0.4em}%
  \begin{center}
    \textcolor{adventgreen}{\emph{#1}}%
  \end{center}
}

%----
% AdventAuthor
%----

\newcommand{\AdventAuthor}{%
  \AddToShipoutPictureFG{%
    \begin{tikzpicture}[remember picture,overlay]
    \node[anchor=south, yshift=2mm] at (current page.south) {%
    \footnotesize Andreas Müller, Kempten University of Applied Sciences, %
    \texttt{andreas.mueller@hs-kempten.de}%
    };
    \end{tikzpicture}%
  }%
}

%----
% AdventInitial
%----

\newcommand{\AdventInitial}[2]{%
  \lettrine[lines=2,lhang=0.1,loversize=0.15]%
    {\textcolor{adventred}{#1}}%
    {#2}%
}

%----
% AdventPageBackground
%----

\newcommand{\AdventPageBackground}{%
  \AddToShipoutPictureBG{%
    \begin{tikzpicture}[remember picture,overlay]
    \draw[adventgreen!80!black, line width=3pt, rounded corners=12pt]
    ($(current page.north west)+(0.8cm,-0.8cm)$)
    rectangle
    ($(current page.south east)+(-0.8cm,0.8cm)$);
    \fill[adventgold]
    ($(current page.north west)+(1.0cm,-1.0cm)$) circle (1.2pt)
    ($(current page.north east)+(-1.0cm,-1.0cm)$) circle (1.2pt)
    ($(current page.south west)+(1.0cm,1.0cm)$) circle (1.2pt)
    ($(current page.south east)+(-1.0cm,1.0cm)$) circle (1.2pt);
    \end{tikzpicture}
  }%
}

%---- AdventSheet macro (1-page, single column) ----
% #1: Date + occasion
% #2: unused
% #3: Main title
% #4: Subtitle
% #5: Key Insight
% #6: Main body content
% #7: Closing statement
\newcommand{\AdventSheet}[7]{%
  \BeginAdventPage
  \vspace*{1cm}
  \begin{AdventFrameTop}
    \AdventTitleBlock{#1}{#2}{#3}{#4}
    \AdventKeyInsight{#5}
  \end{AdventFrameTop}
  \AdventStarRule
  #6%
  \EndAdventPage
  \AdventClosing{#7}%
}

%---- AdventSheetTwoCol macro (two-column hero sheet) ----
% #1: Date + occasion
% #2: unused
% #3: Main title
% #4: Subtitle
% #5: Key Insight
% #6: Main body content (wrapped in multicols{2})
% #7: Closing statement
\newcommand{\AdventSheetTwoCol}[7]{%
  \BeginAdventPage
  \begin{AdventFrameTop}
    \AdventTitleBlock{#1}{#2}{#3}{#4}
    \AdventKeyInsight{#5}
  \end{AdventFrameTop}
  \AdventStarRule
  \begin{multicols}{2}
    #6%
  \end{multicols}
  \EndAdventPage
  \AdventClosing{#7}%
}

\begin{document}

\AdventPageBackground
\AdventAuthor

\AdventSheetTwoCol
  {December 5, 2025} % #1 Date
  {}                 % #2 unused
  {Radius operator: scales as geometric invariants} % #3 Main title
  {From $(\alpha,\beta,\gamma)$ to $(a_0,b_0,c_0)$ and energy hierarchies} % #4 Subtitle
  {The radius operator $R$ is built from the heptagon geometry and the
   octonionic structure. Its spectrum
   $\mathrm{spec}(R)=(a_0,b_0,c_0)$ produces three characteristic
   dimensionless radii. Exponentials of these radii can serve as
   prototypes for the Planck, electroweak and QCD scales. For the first
   time, fundamental energy hierarchies appear as \emph{geometric
   invariants} of an internal operator, not as arbitrary input
   parameters.} % #5 Key Insight
  { % #6 body (two columns)

\section*{From directions to radii}

\AdventInitial{Y}{esterday's} heptagon operator $H_7$ compressed the seven
imaginary directions of the octonions into three eigenvalues
$(\alpha,\beta,\gamma)$. Today we take the next step: we pass from
\emph{angular} information to \emph{radial} information.

Intuitively:

\begin{itemize}
  \item $H_7$ encodes how the seven directions are oriented relative to
        each other.
  \item The radius operator $R$ encodes how far typical internal
        configurations lie from certain preferred centres in this
        7-dimensional structure.
\end{itemize}

The precise construction of $R$ is technical, but its qualitative role is
simple: it measures distance in the internal symmetry atlas defined by the
heptagon and $G_2$.

\section*{Defining the radius operator $R$}

In the internal 8-dimensional space, one can construct an operator $R$
whose definition is constrained by:

\begin{itemize}
  \item $G_2$-invariance (it must respect the octonionic automorphisms),
  \item compatibility with the heptagon structure,
  \item positivity or at least a well-defined spectrum that can be
        interpreted as squared radii.
\end{itemize}

Schematically, one may think of $R$ as a function of the heptagon
operator and related data:

\[
  R = F(H_7),
\]

where $F$ is chosen such that $R$ has only three distinct eigenvalues:

\[
  \mathrm{spec}(R) = \{a_0, b_0, c_0\},
\]

with multiplicities adding up to 8. The triple

\[
  (a_0,b_0,c_0)
\]

is then the \emph{radius spectrum}: three characteristic internal radii
associated with the geometry encoded by the heptagon.

\section*{From radii to energy scales}

In quantum field theory, length scales and energy scales are inversely
related. It is therefore natural to turn dimensionless radii into
dimensionless energy scales by exponentiation:

\[
  \Lambda_i \;\sim\; \exp(a_0),\quad
  \Lambda_j \;\sim\; \exp(b_0),\quad
  \Lambda_k \;\sim\; \exp(c_0),
\]

up to overall normalisations. With a suitable choice of units, one can
associate:

\begin{itemize}
  \item one radius with the \emph{Planck scale},
  \item one with the \emph{electroweak scale},
  \item one with the \emph{QCD/confinement scale}.
\end{itemize}

The key message is not the exact fit (that requires detailed numerics)
but the \emph{structural fact}:

\begin{quote}
  There exist three distinguished internal radii $(a_0,b_0,c_0)$ from
  which three physically relevant energy scales can naturally be
  constructed.
\end{quote}

\section*{Why three scales?}

Empirically, particle physics is organised around three strikingly
different characteristic scales:

\begin{enumerate}
  \item The Planck scale, where gravity becomes comparable to other
        interactions.
  \item The electroweak scale, where $SU(2)_L \times U(1)_Y$ symmetry is
        broken.
  \item The QCD scale, where confinement and chiral symmetry breaking
        dominate.
\end{enumerate}

In the model, this triad is mirrored by the triad $(a_0,b_0,c_0)$:

\begin{itemize}
  \item The number of qualitatively distinct scales is fixed by the
        structure of $R$, not by phenomenological needs.
  \item The relative ordering and separation of these scales can be traced
        back to differences between $a_0$, $b_0$ and $c_0$.
\end{itemize}

This turns a long-standing ``why these three?'' question into a statement
about the eigenstructure of an internal operator.

\section*{Radius operator within the attractor picture}

On the second Advent Sunday (7 December), the calendar will present the
idea of an \emph{attractor} for scales. In that picture:

\begin{itemize}
  \item The triple $(a_0,b_0,c_0)$ defines three preferred radii in the
        internal space.
  \item Renormalisation-group (RG) flow in the physical theory is naturally
        attracted to energy values constructed from these radii.
  \item The observed scales are stable fixed points rather than arbitrary
        initial conditions.
\end{itemize}

The radius operator $R$ is thus the internal, geometric backbone of this
attractor story. Without $R$ and its discrete spectrum, the attractor
mechanism would have nothing to lock onto.

\section*{Conceptual gain from $\mathrm{spec}(R)=(a_0,b_0,c_0)$}

Introducing $R$ and its spectrum brings several conceptual benefits:

\begin{enumerate}
  \item \textbf{Geometric origin of hierarchies:}
    large ratios between energy scales (Planck vs.\ electroweak vs.\ QCD)
    are no longer mere accidents but are linked to differences between
    eigenvalues of a symmetry-constrained operator.
  \item \textbf{Minimality:}
    three radii suffice—no large list of independent scale parameters is
    needed at the fundamental level.
  \item \textbf{Spectral language:}
    scale information is encoded spectrally, aligning with the later use
    of spectral geometry and spectral actions.
\end{enumerate}

This is why the XLS lists the radius operator as the first explicit bridge
from abstract octonionic geometry to physically observed hierarchies.

\small
\begin{thebibliography}{9}

\bibitem{ChamseddineConnesMarcolli2007}
A.~H.~Chamseddine, A.~Connes and M.~Marcolli,
\newblock ``Gravity and the standard model with neutrino mixing,''
\newblock {\em Adv.\ Theor.\ Math.\ Phys.} \textbf{11}, 991--1089 (2007).

\bibitem{Wilson1971}
K.~G.~Wilson,
\newblock ``Renormalization group and critical phenomena,''
\newblock {\em Phys.\ Rev.\ B} \textbf{4}, 3174--3183 (1971).

\bibitem{Internal}
[Internal notes on the radius operator and scale hierarchies:
{\tt chap03\_neu.tex; appK\_neu.tex; konstanten-hierarchie.tex}.]

\end{thebibliography}
\normalsize

  } % end #6 body
  {The radius operator $R$ translates the heptagon geometry into three
   characteristic radii $(a_0,b_0,c_0)$. Exponentials of these radii
   provide natural candidates for the Planck, electroweak and QCD scales,
   turning energy hierarchies into geometric invariants rather than
   arbitrary inputs.} % #7 Closing

\end{document}