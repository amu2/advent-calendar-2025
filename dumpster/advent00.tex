% advent00.tex
% First Advent Sunday: November 30, 2025 – First Light: Octonions, G2 and Triality
% Two-page hero: p.1 technical theme, p.2 one-column series introduction

\documentclass[a4paper,10pt]{article}

\usepackage[utf8]{inputenc}
\usepackage[T1]{fontenc}
\usepackage[english]{babel}
\usepackage{amsmath,amssymb,amsfonts}
\usepackage{xcolor}
\usepackage{tikz}
\usetikzlibrary{calc}
\usepackage[framemethod=TikZ]{mdframed}
\usepackage{titlesec}
\usepackage{lettrine}
\usepackage{eso-pic}
\usepackage{geometry}
\usepackage{multicol}
\usepackage{lmodern}

\geometry{margin=2.0cm}

% advent-layout.tex (corrected for \input usage)

\definecolor{adventred}{HTML}{B3001B}
\definecolor{adventblue}{HTML}{003366}
\definecolor{adventgreen}{HTML}{006633}
\definecolor{adventgold}{HTML}{B59410}

\pagestyle{empty}

\titleformat{\section}
  {\normalfont\large\bfseries\color{adventblue}}{\thesection}{0em}{}
  
\titleformat{\subsection}
  {\normalfont\normalsize\bfseries\color{adventblue}}{\thesubsection}{0em}{}

%----
% AdventFrameTop
%----

\newenvironment{AdventFrameTop}
{%
  \begin{mdframed}[
    linecolor=adventgreen!0,
    linewidth=0pt,
    roundcorner=0pt,
    innertopmargin=10pt,
    innerbottommargin=10pt,
    innerleftmargin=10pt,
    innerrightmargin=10pt,
    backgroundcolor=adventgreen!2
  ]%
}
{%
  \end{mdframed}
}

\newcommand{\BeginAdventPage}{}
\newcommand{\EndAdventPage}{}

%----
% AdventTitleBlock
%----

\newcommand{\AdventTitleBlock}[4]{%
  \begin{center}
    {\Large\textcolor{adventred}{\textbf{#1}}}\par\vspace{4pt}%
    \ifx&#2&\else
      {\large\textbf{#2}}\par\vspace{2pt}%
    \fi
    {\Large\textcolor{adventblue}{\textbf{#3}}}\par
    \ifx&#4&\else
      \vspace{2pt}%
      {\normalsize\textbf{#4}}\par%
    \fi
  \end{center}%
}

%----
% AdventKeyInsight
%----

\newcommand{\AdventKeyInsight}[1]{%
  \vspace{0.5em}%
  \noindent\colorbox{adventred!8}{%
    \parbox{\dimexpr\linewidth-2\fboxsep}{%
    \textbf{\textcolor{adventred}{Key Insight.}}~#1%
    }%
  }%
  \vspace{0.5em}%
}

%----
% AdventStarRule
%----

\newcommand{\AdventStarRule}{%
  \vspace{0.3em}%
  \begin{center}
    {\color{adventgold}%
    \rule[0.5ex]{0.25\linewidth}{0.4pt}\;
    $\ast\;\ast\;\ast$\;
    \rule[0.5ex]{0.25\linewidth}{0.4pt}%
    }%
  \end{center}
  \vspace{0.3em}%
}

%----
% AdventClosing
%----

\newcommand{\AdventClosing}[1]{%
  \vspace{0.4em}%
  \begin{center}
    \textcolor{adventgreen}{\emph{#1}}%
  \end{center}
}

%----
% AdventAuthor
%----

\newcommand{\AdventAuthor}{%
  \AddToShipoutPictureFG{%
    \begin{tikzpicture}[remember picture,overlay]
    \node[anchor=south, yshift=2mm] at (current page.south) {%
    \footnotesize Andreas Müller, Kempten University of Applied Sciences, %
    \texttt{andreas.mueller@hs-kempten.de}%
    };
    \end{tikzpicture}%
  }%
}

%----
% AdventInitial
%----

\newcommand{\AdventInitial}[2]{%
  \lettrine[lines=2,lhang=0.1,loversize=0.15]%
    {\textcolor{adventred}{#1}}%
    {#2}%
}

%----
% AdventPageBackground
%----

\newcommand{\AdventPageBackground}{%
  \AddToShipoutPictureBG{%
    \begin{tikzpicture}[remember picture,overlay]
    \draw[adventgreen!80!black, line width=3pt, rounded corners=12pt]
    ($(current page.north west)+(0.8cm,-0.8cm)$)
    rectangle
    ($(current page.south east)+(-0.8cm,0.8cm)$);
    \fill[adventgold]
    ($(current page.north west)+(1.0cm,-1.0cm)$) circle (1.2pt)
    ($(current page.north east)+(-1.0cm,-1.0cm)$) circle (1.2pt)
    ($(current page.south west)+(1.0cm,1.0cm)$) circle (1.2pt)
    ($(current page.south east)+(-1.0cm,1.0cm)$) circle (1.2pt);
    \end{tikzpicture}
  }%
}

%---- AdventSheet macro (1-page, single column) ----
% #1: Date + occasion
% #2: unused
% #3: Main title
% #4: Subtitle
% #5: Key Insight
% #6: Main body content
% #7: Closing statement
\newcommand{\AdventSheet}[7]{%
  \BeginAdventPage
  \vspace*{1cm}
  \begin{AdventFrameTop}
    \AdventTitleBlock{#1}{#2}{#3}{#4}
    \AdventKeyInsight{#5}
  \end{AdventFrameTop}
  \AdventStarRule
  #6%
  \EndAdventPage
  \AdventClosing{#7}%
}

%---- AdventSheetTwoCol macro (two-column hero sheet) ----
% #1: Date + occasion
% #2: unused
% #3: Main title
% #4: Subtitle
% #5: Key Insight
% #6: Main body content (wrapped in multicols{2})
% #7: Closing statement
\newcommand{\AdventSheetTwoCol}[7]{%
  \BeginAdventPage
  \begin{AdventFrameTop}
    \AdventTitleBlock{#1}{#2}{#3}{#4}
    \AdventKeyInsight{#5}
  \end{AdventFrameTop}
  \AdventStarRule
  \begin{multicols}{2}
    #6%
  \end{multicols}
  \EndAdventPage
  \AdventClosing{#7}%
}

\begin{document}

\AdventPageBackground
\AdventAuthor

%--------------------------------------------------
% Page 1: Two-column hero about O, G2, triality
%--------------------------------------------------

\AdventSheetTwoCol
  {November 30, 2025 \large (First Advent Sunday)} % #1 Date + occasion
  {}                                              % #2 unused
  {First Light: Octonions, $G_2$ and Triality}    % #3 Main title
  {100 years after Heisenberg's matrix mechanics} % #4 Subtitle
  {One hundred years ago, Heisenberg replaced continuous orbits by
   discrete matrices and discovered noncommutativity. In this Advent
   story we go one step further: we anchor physics in an
   eight-dimensional number system --- the octonions $\mathbb{O}$ ---
   whose automorphisms form the exceptional group $G_2$ and whose
   triality-related Spin(8) structure organises an entire generation of
   matter. The surprising claim is that this rigid internal stage is
   already ``almost'' the Standard Model plus gravity.} % #5 Key Insight
  { % #6 main body (two columns)

\section*{From noncommutativity to nonassociativity}

\AdventInitial{I}{n} 1925, Heisenberg's matrix mechanics marked a clean
break with classical intuitions: position and momentum were no longer
numbers but noncommuting operators. This was a first ``algebraic turn'' in
physics. Today, noncommutativity is standard language in quantum theory.

The next turn is less familiar. There exists a unique real division algebra
of dimension eight, the octonions $\mathbb{O}$, which is not only
noncommutative but also \emph{nonassociative}. Multiplying three octonions
depends on how we place the brackets. At first sight this seems like a
mathematical curiosity, far removed from physics.

However, nonassociativity comes with a remarkable compensation: the
octonions still admit a multiplicative norm,
\[
  N(xy) \;=\; N(x)\,N(y),
\]
and their automorphism group is the smallest exceptional Lie group $G_2$.
This rigid internal symmetry makes $\mathbb{O}$ a natural candidate for a
hidden layer beneath the familiar complex Hilbert spaces of quantum
theory.

\section*{The octonionic stage}

The real vector space underlying the octonions is $\mathbb{R}^8$. As a
stage for physics it brings several features at once:

\begin{itemize}
  \item An 8-dimensional structure that can host vectors and spinors of
        $\mathrm{Spin}(8)$, the double cover of $\mathrm{SO}(8)$.
  \item A distinguished subgroup $G_2 \subset \mathrm{SO}(8)$ that fixes the
        multiplication table of $\mathbb{O}$.
  \item A triality symmetry of $\mathrm{Spin}(8)$ that permutes its three
        eight-dimensional irreducible representations:
        vector, left-handed spinor, right-handed spinor.
\end{itemize}

Physically, this means that:

\begin{itemize}
  \item Internal degrees of freedom can be arranged in three correlated
        eight-component blocks.
  \item Rotations in the internal space can mix these blocks in a highly
        constrained way.
\end{itemize}

Later in the Advent calendar, these three blocks will be read as three
generations of fermions, and specific subgroups will be identified with
$SU(3)\times SU(2)\times U(1)$.

\section*{$G_2$ as guardian of the multiplication table}

The group $G_2$ can be defined as the set of all linear maps
$g:\mathbb{O}\to\mathbb{O}$ that preserve octonionic multiplication:
\[
  g(xy) \;=\; g(x)\,g(y)
  \quad \text{for all } x,y\in\mathbb{O}.
\]
This makes $G_2$ the symmetry group of the multiplication table. In a
physical setting, $G_2$-compatible transformations are those that respect
the hidden octonionic structure of the internal space.

The existence of $G_2$ has two important consequences:

\begin{itemize}
  \item It restricts which internal rotations are ``legal'' if we want to
        keep the multiplication law intact.
  \item It provides a natural environment for embeddings of familiar gauge
        groups. Subgroups of $G_2$ and related structures can house
        $SU(3)$-like and $SU(2)\times U(1)$-like symmetries.
\end{itemize}

In this sense, the octonionic stage with $G_2$ symmetry is already a
candidate for the internal symmetry space of the Standard Model.

\section*{Triality and the seed of generations}

The group $\mathrm{Spin}(8)$ acts on three eight-dimensional
representations:
\[
  V_8,\quad S_8^+,\quad S_8^-.
\]
These correspond to vectors, left-handed spinors and right-handed spinors.
An exceptional outer automorphism permutes these three representations. This
is the triality symmetry.

For our purposes, triality can be read as a structural reason for the
recurrence of the number three in particle physics:

\begin{itemize}
  \item There is \emph{one} internal eight-dimensional block, but it admits
        three coherent readings: $V_8$, $S_8^+$, $S_8^-$.
  \item Later we will see how these three readings unfold into three
        generations of quarks and leptons when embedded into a larger
        exceptional algebra.
\end{itemize}

In this way, the abstract triality of $\mathrm{Spin}(8)$ becomes a seed for
family replication.

\section*{Why start the Advent story here?}

Starting the Advent calendar with octonions, $G_2$ and triality is not an
exercise in exotic mathematics for its own sake. It sets up three themes
that will run through the entire series:

\begin{enumerate}
  \item \textbf{Rigidity:} Exceptional structures like $\mathbb{O}$ and
        $G_2$ leave little room for arbitrary choices. This rigidity is a
        feature if we seek explanations rather than fits.

  \item \textbf{Hidden simplicity:} Behind the zoo of fields and parameters
        in the Standard Model there might be a much smaller set of algebraic
        building blocks.

  \item \textbf{Algebra as geometry:} Nonassociative multiplication and
        its automorphisms can be interpreted as a kind of curved internal
        geometry, on par with spacetime curvature in general relativity.
\end{enumerate}

\small
\begin{thebibliography}{9}

\bibitem{Heisenberg1925}
W.~Heisenberg,
\newblock ``Über quantentheoretische Umdeutung kinematischer und
mechanischer Beziehungen,''
\newblock {\em Z.\ Phys.} \textbf{33}, 879--893 (1925).

\bibitem{Baez2002}
J.~C.~Baez,
\newblock ``The octonions,''
\newblock {\em Bull.\ Amer.\ Math.\ Soc.} \textbf{39}, 145--205 (2002).

\bibitem{GurseyTze1996}
F.~Gürsey and H.~C.~Tze,
\newblock {\em On the Role of Division, Jordan and Related Algebras in Particle
Physics},
\newblock World Scientific, 1996.

\end{thebibliography}
\normalsize

  } % end #6 body
  {One century after matrix mechanics, we explore a universe whose hidden stage is an exceptional eight-dimensional number system.} % #7 Closing



\end{document}