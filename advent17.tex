% advent17.tex
% December 17, 2025 – Spectral geometry: bridge to gravity

\documentclass[a4paper,10pt]{article}

\usepackage[utf8]{inputenc}
\usepackage[T1]{fontenc}
\usepackage[english]{babel}
\usepackage{amsmath,amssymb,amsfonts}
\usepackage{xcolor}
\usepackage{tikz}
\usetikzlibrary{calc}
\usepackage[framemethod=TikZ]{mdframed}
\usepackage{titlesec}
\usepackage{lettrine}
\usepackage{eso-pic}
\usepackage{geometry}
\usepackage{multicol}
\usepackage{lmodern}

\geometry{margin=2.0cm}

% advent-layout.tex (corrected for \input usage)

\definecolor{adventred}{HTML}{B3001B}
\definecolor{adventblue}{HTML}{003366}
\definecolor{adventgreen}{HTML}{006633}
\definecolor{adventgold}{HTML}{B59410}

\pagestyle{empty}

\titleformat{\section}
  {\normalfont\large\bfseries\color{adventblue}}{\thesection}{0em}{}
  
\titleformat{\subsection}
  {\normalfont\normalsize\bfseries\color{adventblue}}{\thesubsection}{0em}{}

%----
% AdventFrameTop
%----

\newenvironment{AdventFrameTop}
{%
  \begin{mdframed}[
    linecolor=adventgreen!0,
    linewidth=0pt,
    roundcorner=0pt,
    innertopmargin=10pt,
    innerbottommargin=10pt,
    innerleftmargin=10pt,
    innerrightmargin=10pt,
    backgroundcolor=adventgreen!2
  ]%
}
{%
  \end{mdframed}
}

\newcommand{\BeginAdventPage}{}
\newcommand{\EndAdventPage}{}

%----
% AdventTitleBlock
%----

\newcommand{\AdventTitleBlock}[4]{%
  \begin{center}
    {\Large\textcolor{adventred}{\textbf{#1}}}\par\vspace{4pt}%
    \ifx&#2&\else
      {\large\textbf{#2}}\par\vspace{2pt}%
    \fi
    {\Large\textcolor{adventblue}{\textbf{#3}}}\par
    \ifx&#4&\else
      \vspace{2pt}%
      {\normalsize\textbf{#4}}\par%
    \fi
  \end{center}%
}

%----
% AdventKeyInsight
%----

\newcommand{\AdventKeyInsight}[1]{%
  \vspace{0.5em}%
  \noindent\colorbox{adventred!8}{%
    \parbox{\dimexpr\linewidth-2\fboxsep}{%
    \textbf{\textcolor{adventred}{Key Insight.}}~#1%
    }%
  }%
  \vspace{0.5em}%
}

%----
% AdventStarRule
%----

\newcommand{\AdventStarRule}{%
  \vspace{0.3em}%
  \begin{center}
    {\color{adventgold}%
    \rule[0.5ex]{0.25\linewidth}{0.4pt}\;
    $\ast\;\ast\;\ast$\;
    \rule[0.5ex]{0.25\linewidth}{0.4pt}%
    }%
  \end{center}
  \vspace{0.3em}%
}

%----
% AdventClosing
%----

\newcommand{\AdventClosing}[1]{%
  \vspace{0.4em}%
  \begin{center}
    \textcolor{adventgreen}{\emph{#1}}%
  \end{center}
}

%----
% AdventAuthor
%----

\newcommand{\AdventAuthor}{%
  \AddToShipoutPictureFG{%
    \begin{tikzpicture}[remember picture,overlay]
    \node[anchor=south, yshift=2mm] at (current page.south) {%
    \footnotesize Andreas Müller, Kempten University of Applied Sciences, %
    \texttt{andreas.mueller@hs-kempten.de}%
    };
    \end{tikzpicture}%
  }%
}

%----
% AdventInitial
%----

\newcommand{\AdventInitial}[2]{%
  \lettrine[lines=2,lhang=0.1,loversize=0.15]%
    {\textcolor{adventred}{#1}}%
    {#2}%
}

%----
% AdventPageBackground
%----

\newcommand{\AdventPageBackground}{%
  \AddToShipoutPictureBG{%
    \begin{tikzpicture}[remember picture,overlay]
    \draw[adventgreen!80!black, line width=3pt, rounded corners=12pt]
    ($(current page.north west)+(0.8cm,-0.8cm)$)
    rectangle
    ($(current page.south east)+(-0.8cm,0.8cm)$);
    \fill[adventgold]
    ($(current page.north west)+(1.0cm,-1.0cm)$) circle (1.2pt)
    ($(current page.north east)+(-1.0cm,-1.0cm)$) circle (1.2pt)
    ($(current page.south west)+(1.0cm,1.0cm)$) circle (1.2pt)
    ($(current page.south east)+(-1.0cm,1.0cm)$) circle (1.2pt);
    \end{tikzpicture}
  }%
}

%---- AdventSheet macro (1-page, single column) ----
% #1: Date + occasion
% #2: unused
% #3: Main title
% #4: Subtitle
% #5: Key Insight
% #6: Main body content
% #7: Closing statement
\newcommand{\AdventSheet}[7]{%
  \BeginAdventPage
  \vspace*{1cm}
  \begin{AdventFrameTop}
    \AdventTitleBlock{#1}{#2}{#3}{#4}
    \AdventKeyInsight{#5}
  \end{AdventFrameTop}
  \AdventStarRule
  #6%
  \EndAdventPage
  \AdventClosing{#7}%
}

%---- AdventSheetTwoCol macro (two-column hero sheet) ----
% #1: Date + occasion
% #2: unused
% #3: Main title
% #4: Subtitle
% #5: Key Insight
% #6: Main body content (wrapped in multicols{2})
% #7: Closing statement
\newcommand{\AdventSheetTwoCol}[7]{%
  \BeginAdventPage
  \begin{AdventFrameTop}
    \AdventTitleBlock{#1}{#2}{#3}{#4}
    \AdventKeyInsight{#5}
  \end{AdventFrameTop}
  \AdventStarRule
  \begin{multicols}{2}
    #6%
  \end{multicols}
  \EndAdventPage
  \AdventClosing{#7}%
}

\begin{document}

\AdventPageBackground
\AdventAuthor

\AdventSheetTwoCol
  {December 17, 2025} % #1 Date
  {}                  % #2 unused
  {Spectral geometry: a bridge to gravity} % #3 Main title
  {When a Dirac spectrum encodes both spacetime and octonions} % #4 Subtitle
  {To pull gravity into the octonionic story, one needs a framework in
   which spacetime geometry and internal exceptional structure are encoded
   in a single object. Spectral geometry provides exactly that: a Dirac
   operator $D$ whose spectrum knows about curvature, gauge fields and
   matter. A spectral action
   $$S_{\mathrm{spec}} = \mathrm{Tr}\,f(D^2/\Lambda^2)$$
   then generates the Einstein--Hilbert term, cosmological constant and
   gauge dynamics in one stroke. This is the natural bridge between the
   octonionic operator toolbox and gravitational physics.} % #5 Key Insight
  { % #6 body (two columns)

\section*{From metrics to spectra}

\AdventInitial{I}{n} Riemannian geometry, a metric $g_{\mu\nu}$ defines
distances and curvature. Spectral geometry starts from a different object:
a Dirac operator $D$ acting on spinors. Remarkably, much of the geometric
information can be recovered from the spectrum of $D$:

\begin{itemize}
  \item The eigenvalues of $D$ encode the volume and curvature of the
        manifold.
  \item Heat-kernel coefficients of $D^2$ reproduce scalar curvature,
        cosmological constant terms and higher curvature invariants.
\end{itemize}

The slogan is:

\begin{quote}
  ``Geometry is what you can hear from the spectrum of $D$.''
\end{quote}

\section*{The spectral action principle}

Chamseddine and Connes proposed to build the fundamental action of physics
from the spectrum of $D$:
\[
  S_{\mathrm{spec}} = \mathrm{Tr}\,f(D^2/\Lambda^2),
\]
where $f$ is a positive cut-off function and $\Lambda$ a large mass scale.

The heat-kernel expansion of this trace yields schematically
\[
  S_{\mathrm{spec}}
  \sim \int d^4x\,\sqrt{|g|}
       \bigl(
         \Lambda^4 a_0
       + \Lambda^2 a_2 R
       + a_4 (R^2, F_{\mu\nu}F^{\mu\nu}, \ldots)
       + \cdots
       \bigr),
\]
where $R$ is the scalar curvature, $F_{\mu\nu}$ are gauge field strengths
and the $a_n$ are Seeley--DeWitt coefficients built from the geometry and
internal data.

Thus, from a single spectral expression one recovers:

\begin{itemize}
  \item the cosmological constant term,
  \item the Einstein--Hilbert term,
  \item gauge kinetic terms,
  \item and higher-order corrections.
\end{itemize}

\section*{Adding internal structure}

In noncommutative geometry, one extends the usual spinor space by an
internal finite-dimensional Hilbert space $\mathcal{H}_F$ carrying the
internal degrees of freedom (gauge charges, generations, etc.). The full
Hilbert space becomes
\[
  \mathcal{H} = L^2(\text{spinors on }M) \otimes \mathcal{H}_F,
\]
and the Dirac operator factorises as
\[
  D = D_M \otimes \mathbf{1}
    + \gamma^5 \otimes D_F,
\]
where $D_M$ is the spacetime Dirac operator and $D_F$ encodes the internal
structure.

For the Standard Model, a specific choice of $D_F$ reproduces the known
gauge group and particle content via the spectral action. For the
octonionic model, $D_F$ should encode:

\begin{itemize}
  \item the octonionic/Albert structure $H_3(\mathbb{O})$,
  \item the rotor/compressor operator toolbox,
  \item triality and $G_2$/$F_4$ symmetries.
\end{itemize}

\section*{How octonions could enter $D$}

From the perspective of this Advent calendar, we already have the
ingredients for $D_F$:

\begin{itemize}
  \item Compressors and rotors acting on the internal space of one
        generation.
  \item The mass map $\Pi(\langle H\rangle)$ as a candidate for the
        fermionic mass block in $D_F$.
  \item Gauge couplings read from rotor commutators, suggestive of the
        gauge-covariant part of $D$.
\end{itemize}

The spectral-geometry programme for the octonionic model can be stated in
one sentence:

\begin{quote}
  Build a finite Dirac operator $D_F$ from the existing octonionic
  operator toolbox, then feed it into the spectral action and see what
  gravitational and gauge sector emerges.
\end{quote}

If successful, this would:

\begin{itemize}
  \item Derive gravitational couplings (and possibly $\kappa=m_p/m_P$)
        from the same internal data that set masses and couplings.
  \item Turn the ``missing gravity corner'' of the previous day into a
        computed part of the spectrum of $D$.
\end{itemize}

\section*{What is realistic to expect}

The spectral-geometry approach is ambitious but not magic. Realistically,
one might hope for:

\begin{itemize}
  \item \textbf{Structural results:} identification of which combinations
        of internal invariants control the effective Newton constant and
        cosmological constant in the spectral action.
  \item \textbf{Order-of-magnitude estimates:} showing that the same
        operator scales that fix internal hierarchies also produce a
        plausible separation between proton, electroweak and Planck
        scales.
  \item \textbf{Consistency checks:} demonstrating that the octonionic
        $D_F$ reproduces the known gauge content and anomaly structure in
        the low-energy limit.
\end{itemize}

A full, detailed computation of all spectral coefficients is a long-term
project, but the direction is clear.

\section*{Why this is the natural next step}

The spectral-day message is simple:

\begin{enumerate}
  \item We already think in terms of operators (rotors, compressors, mass
        maps) on an exceptional internal space.
  \item Spectral geometry says that the \emph{spectrum} of a combined
        Dirac operator captures both internal and spacetime geometry.
  \item The spectral action then turns this spectrum into an effective
        action with gravity, gauge fields and matter.
\end{enumerate}

In other words, spectral geometry is not an add-on but the natural language
in which the octonionic model can talk to gravity. It is the mathematical
framework in which yesterday's ``missing corner'' can, in principle, be
filled.

\small
\begin{thebibliography}{9}

\bibitem{ConnesChamseddine1997}
A.~Connes and A.~H.~Chamseddine,
\newblock ``The spectral action principle,''
\newblock {\em Commun.\ Math.\ Phys.} \textbf{186}, 731--750 (1997).

\bibitem{ChamseddineConnesMarcolli2007}
A.~H.~Chamseddine, A.~Connes and M.~Marcolli,
\newblock ``Gravity and the standard model with neutrino mixing,''
\newblock {\em Adv.\ Theor.\ Math.\ Phys.} \textbf{11}, 991--1089 (2007).

\bibitem{Internal}
[Internal notes on spectral geometry and gravity:
{\tt appE\_neu.tex; chap20\_neu.tex}.]

\end{thebibliography}
\normalsize

  } % end #6 body
  {Spectral geometry offers exactly what the octonionic model needs next:
   a Dirac operator whose spectrum unifies internal exceptional structure
   with spacetime curvature and gravity.} % #7 Closing

\end{document}