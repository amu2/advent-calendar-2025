% advent08.tex
% December 8, 2025 – Nonassociativity as origin of hierarchies

\documentclass[a4paper,10pt]{article}

\usepackage[utf8]{inputenc}
\usepackage[T1]{fontenc}
\usepackage[english]{babel}
\usepackage{amsmath,amssymb,amsfonts}
\usepackage{xcolor}
\usepackage{tikz}
\usetikzlibrary{calc}
\usepackage[framemethod=TikZ]{mdframed}
\usepackage{titlesec}
\usepackage{lettrine}
\usepackage{eso-pic}
\usepackage{geometry}
\usepackage{multicol}
\usepackage{lmodern}

\geometry{margin=2.0cm}

% advent-layout.tex (corrected for \input usage)

\definecolor{adventred}{HTML}{B3001B}
\definecolor{adventblue}{HTML}{003366}
\definecolor{adventgreen}{HTML}{006633}
\definecolor{adventgold}{HTML}{B59410}

\pagestyle{empty}

\titleformat{\section}
  {\normalfont\large\bfseries\color{adventblue}}{\thesection}{0em}{}
  
\titleformat{\subsection}
  {\normalfont\normalsize\bfseries\color{adventblue}}{\thesubsection}{0em}{}

%----
% AdventFrameTop
%----

\newenvironment{AdventFrameTop}
{%
  \begin{mdframed}[
    linecolor=adventgreen!0,
    linewidth=0pt,
    roundcorner=0pt,
    innertopmargin=10pt,
    innerbottommargin=10pt,
    innerleftmargin=10pt,
    innerrightmargin=10pt,
    backgroundcolor=adventgreen!2
  ]%
}
{%
  \end{mdframed}
}

\newcommand{\BeginAdventPage}{}
\newcommand{\EndAdventPage}{}

%----
% AdventTitleBlock
%----

\newcommand{\AdventTitleBlock}[4]{%
  \begin{center}
    {\Large\textcolor{adventred}{\textbf{#1}}}\par\vspace{4pt}%
    \ifx&#2&\else
      {\large\textbf{#2}}\par\vspace{2pt}%
    \fi
    {\Large\textcolor{adventblue}{\textbf{#3}}}\par
    \ifx&#4&\else
      \vspace{2pt}%
      {\normalsize\textbf{#4}}\par%
    \fi
  \end{center}%
}

%----
% AdventKeyInsight
%----

\newcommand{\AdventKeyInsight}[1]{%
  \vspace{0.5em}%
  \noindent\colorbox{adventred!8}{%
    \parbox{\dimexpr\linewidth-2\fboxsep}{%
    \textbf{\textcolor{adventred}{Key Insight.}}~#1%
    }%
  }%
  \vspace{0.5em}%
}

%----
% AdventStarRule
%----

\newcommand{\AdventStarRule}{%
  \vspace{0.3em}%
  \begin{center}
    {\color{adventgold}%
    \rule[0.5ex]{0.25\linewidth}{0.4pt}\;
    $\ast\;\ast\;\ast$\;
    \rule[0.5ex]{0.25\linewidth}{0.4pt}%
    }%
  \end{center}
  \vspace{0.3em}%
}

%----
% AdventClosing
%----

\newcommand{\AdventClosing}[1]{%
  \vspace{0.4em}%
  \begin{center}
    \textcolor{adventgreen}{\emph{#1}}%
  \end{center}
}

%----
% AdventAuthor
%----

\newcommand{\AdventAuthor}{%
  \AddToShipoutPictureFG{%
    \begin{tikzpicture}[remember picture,overlay]
    \node[anchor=south, yshift=2mm] at (current page.south) {%
    \footnotesize Andreas Müller, Kempten University of Applied Sciences, %
    \texttt{andreas.mueller@hs-kempten.de}%
    };
    \end{tikzpicture}%
  }%
}

%----
% AdventInitial
%----

\newcommand{\AdventInitial}[2]{%
  \lettrine[lines=2,lhang=0.1,loversize=0.15]%
    {\textcolor{adventred}{#1}}%
    {#2}%
}

%----
% AdventPageBackground
%----

\newcommand{\AdventPageBackground}{%
  \AddToShipoutPictureBG{%
    \begin{tikzpicture}[remember picture,overlay]
    \draw[adventgreen!80!black, line width=3pt, rounded corners=12pt]
    ($(current page.north west)+(0.8cm,-0.8cm)$)
    rectangle
    ($(current page.south east)+(-0.8cm,0.8cm)$);
    \fill[adventgold]
    ($(current page.north west)+(1.0cm,-1.0cm)$) circle (1.2pt)
    ($(current page.north east)+(-1.0cm,-1.0cm)$) circle (1.2pt)
    ($(current page.south west)+(1.0cm,1.0cm)$) circle (1.2pt)
    ($(current page.south east)+(-1.0cm,1.0cm)$) circle (1.2pt);
    \end{tikzpicture}
  }%
}

%---- AdventSheet macro (1-page, single column) ----
% #1: Date + occasion
% #2: unused
% #3: Main title
% #4: Subtitle
% #5: Key Insight
% #6: Main body content
% #7: Closing statement
\newcommand{\AdventSheet}[7]{%
  \BeginAdventPage
  \vspace*{1cm}
  \begin{AdventFrameTop}
    \AdventTitleBlock{#1}{#2}{#3}{#4}
    \AdventKeyInsight{#5}
  \end{AdventFrameTop}
  \AdventStarRule
  #6%
  \EndAdventPage
  \AdventClosing{#7}%
}

%---- AdventSheetTwoCol macro (two-column hero sheet) ----
% #1: Date + occasion
% #2: unused
% #3: Main title
% #4: Subtitle
% #5: Key Insight
% #6: Main body content (wrapped in multicols{2})
% #7: Closing statement
\newcommand{\AdventSheetTwoCol}[7]{%
  \BeginAdventPage
  \begin{AdventFrameTop}
    \AdventTitleBlock{#1}{#2}{#3}{#4}
    \AdventKeyInsight{#5}
  \end{AdventFrameTop}
  \AdventStarRule
  \begin{multicols}{2}
    #6%
  \end{multicols}
  \EndAdventPage
  \AdventClosing{#7}%
}

\begin{document}

\AdventPageBackground
\AdventAuthor

\AdventSheetTwoCol
  {December 8, 2025} % #1 Date
  {}                 % #2 unused
  {Nonassociativity as the source of hierarchies} % #3 Main title
  {When $(ab)c\neq a(bc)$ turns into mass and scale gaps} % #4 Subtitle
  {Octonions are nonassociative: in general $(ab)c\neq a(bc)$. Far from a
   nuisance, this failure of associativity can be measured by the
   \emph{associator}, and its norm feeds directly into the size of
   hierarchies. In the octonionic model, the huge gaps between electron
   and top mass, or between Planck and electroweak scales, are not
   independent miracles: they are controlled by how strongly the internal
   multiplication fails to be associative in selected directions.} % #5 Key Insight
  { % #6 body (two columns)

\section*{Associativity lost, structure gained}

\AdventInitial{I}{n} the complex numbers and quaternions we have
\[
  (ab)c = a(bc)
  \quad\text{for all }a,b,c.
\]
For octonions $\mathbb{O}$ this is no longer true. The deviation is
captured by the \emph{associator}
\[
  [a,b,c] := (ab)c - a(bc).
\]
Nonassociativity means $[a,b,c]\neq0$ for suitable triples $(a,b,c)$.

At first sight this looks like a technical complication. But in an
operator-based model, $[a,b,c]$ is a structured quantity whose norm can
be used as a measure of ``how curved'' the internal multiplication is in a
given region of the algebra.

\section*{Associator norms as hierarchy seeds}

The key idea is simple:

\begin{quote}
  Strong nonassociativity (large $||[a,b,c]||$) correlates with
  \emph{large} hierarchies; near-associative directions correspond to
  \emph{small} gaps.
\end{quote}

Schematically, for suitable triples $(a,b,c)$ associated with sectors
of the internal space, one can write
\[
  \text{hierarchy factor} \;\sim\;
  \exp\!\bigl(\gamma\,\|[a,b,c]\|\bigr),
\]
with a model-dependent constant $\gamma$. Large associator norms then
naturally give exponentials of order $10^{10}$ or $10^{30}$—exactly the
kind of huge ratios seen between Planck and electroweak scales, or between
electron and top mass.

\section*{Flat vs.\ curved internal directions}

Not all directions in $\mathbb{O}$ are equally nonassociative:

\begin{itemize}
  \item Quaternion subalgebras inside $\mathbb{O}$ are associative:
        $[a,b,c]=0$ whenever $a,b,c$ lie in the same quaternionic
        subalgebra. These are ``flat'' directions.
  \item Triples that genuinely probe the full octonionic structure
        typically have $[a,b,c]\neq0$. These are ``curved'' directions.
\end{itemize}

This suggests a qualitative picture:

\begin{itemize}
  \item Light fermions and low-energy scales live predominantly in almost
        quaternionic (nearly associative) directions.
  \item Heavy fermions and high-energy scales probe strongly octonionic
        (strongly nonassociative) directions.
\end{itemize}

The associator becomes a geometric dial between ``light'' and ``heavy''.

\section*{From internal curvature to physical gaps}

In curved spacetime geometry, curvature scalars control focusing of
geodesics and tidal forces. In the octonionic internal geometry, the
associator norm plays an analogous role:

\begin{itemize}
  \item It tells us how violently internal products deviate from the
        naive, associative expectation.
  \item Through the operator toolbox (rotors, compressors, radius
        operator), this deviation feeds into eigenvalue spectra and hence
        into physical scales and masses.
\end{itemize}

Symbolically one can write
\[
  m_{\text{heavy}}/m_{\text{light}}
   \;\sim\; F\bigl(\|[a,b,c]\|\bigr),
\]
for some monotone function $F$ determined by the detailed embedding in
$H_3(\mathbb{O})$.

\section*{Why this matters conceptually}

The nonassociativity day adds an important layer to the Advent story:

\begin{enumerate}
  \item It explains why large hierarchies are \emph{allowed} and
        \emph{natural} in an octonionic setting: they are the rule, not
        the exception, once associativity is dropped.
  \item It connects an abstract algebraic property—failure of
        associativity—to directly observable quantities: mass ratios and
        scale separations.
  \item It sharpens the contrast with associative models, where such large
        gaps often have to be enforced by hand via fine-tuned potentials
        or additional symmetries.
\end{enumerate}

If future work can make this link between associator norms and concrete
hierarchies numerically sharp, nonassociativity would move from a curious
mathematical footnote to a central player in explaining why the physical
world has the extreme scales we observe.

\small
\begin{thebibliography}{9}

\bibitem{Schafer1966}
R.~D.~Schafer,
\newblock {\em An Introduction to Nonassociative Algebras},
\newblock Academic Press, 1966.

\bibitem{Baez2002}
J.~C.~Baez,
\newblock ``The octonions,''
\newblock {\em Bull.\ Amer.\ Math.\ Soc.} \textbf{39}, 145--205 (2002).

\bibitem{Internal}
[Internal notes on hierarchy patterns and associators, see
{\tt chap02\_neu.tex; chap05\_neu.tex}.]

\end{thebibliography}
\normalsize

  } % end #6 body
  {In an octonionic world, huge hierarchies do not come from fine-tuning
   potentials—they follow the size of the associator: how far internal
   multiplication strays from associativity.} % #7 Closing

\end{document}