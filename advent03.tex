% advent03.tex
% December 3, 2025 – G2 as the minimal exceptional symmetry

\documentclass[a4paper,10pt]{article}

\usepackage[utf8]{inputenc}
\usepackage[T1]{fontenc}
\usepackage[english]{babel}
\usepackage{amsmath,amssymb,amsfonts}
\usepackage{xcolor}
\usepackage{tikz}
\usetikzlibrary{calc}
\usepackage[framemethod=TikZ]{mdframed}
\usepackage{titlesec}
\usepackage{lettrine}
\usepackage{eso-pic}
\usepackage{geometry}
\usepackage{multicol}
\usepackage{lmodern}
\usepackage{hyperref}

\geometry{margin=2.0cm}

% advent-layout.tex (corrected for \input usage)

\definecolor{adventred}{HTML}{B3001B}
\definecolor{adventblue}{HTML}{003366}
\definecolor{adventgreen}{HTML}{006633}
\definecolor{adventgold}{HTML}{B59410}

\pagestyle{empty}

\titleformat{\section}
  {\normalfont\large\bfseries\color{adventblue}}{\thesection}{0em}{}
  
\titleformat{\subsection}
  {\normalfont\normalsize\bfseries\color{adventblue}}{\thesubsection}{0em}{}

%----
% AdventFrameTop
%----

\newenvironment{AdventFrameTop}
{%
  \begin{mdframed}[
    linecolor=adventgreen!0,
    linewidth=0pt,
    roundcorner=0pt,
    innertopmargin=10pt,
    innerbottommargin=10pt,
    innerleftmargin=10pt,
    innerrightmargin=10pt,
    backgroundcolor=adventgreen!2
  ]%
}
{%
  \end{mdframed}
}

\newcommand{\BeginAdventPage}{}
\newcommand{\EndAdventPage}{}

%----
% AdventTitleBlock
%----

\newcommand{\AdventTitleBlock}[4]{%
  \begin{center}
    {\Large\textcolor{adventred}{\textbf{#1}}}\par\vspace{4pt}%
    \ifx&#2&\else
      {\large\textbf{#2}}\par\vspace{2pt}%
    \fi
    {\Large\textcolor{adventblue}{\textbf{#3}}}\par
    \ifx&#4&\else
      \vspace{2pt}%
      {\normalsize\textbf{#4}}\par%
    \fi
  \end{center}%
}

%----
% AdventKeyInsight
%----

\newcommand{\AdventKeyInsight}[1]{%
  \vspace{0.5em}%
  \noindent\colorbox{adventred!8}{%
    \parbox{\dimexpr\linewidth-2\fboxsep}{%
    \textbf{\textcolor{adventred}{Key Insight.}}~#1%
    }%
  }%
  \vspace{0.5em}%
}

%----
% AdventStarRule
%----

\newcommand{\AdventStarRule}{%
  \vspace{0.3em}%
  \begin{center}
    {\color{adventgold}%
    \rule[0.5ex]{0.25\linewidth}{0.4pt}\;
    $\ast\;\ast\;\ast$\;
    \rule[0.5ex]{0.25\linewidth}{0.4pt}%
    }%
  \end{center}
  \vspace{0.3em}%
}

%----
% AdventClosing
%----

\newcommand{\AdventClosing}[1]{%
  \vspace{0.4em}%
  \begin{center}
    \textcolor{adventgreen}{\emph{#1}}%
  \end{center}
}

%----
% AdventAuthor
%----

\newcommand{\AdventAuthor}{%
  \AddToShipoutPictureFG{%
    \begin{tikzpicture}[remember picture,overlay]
    \node[anchor=south, yshift=2mm] at (current page.south) {%
    \footnotesize Andreas Müller, Kempten University of Applied Sciences, %
    \texttt{andreas.mueller@hs-kempten.de}%
    };
    \end{tikzpicture}%
  }%
}

%----
% AdventInitial
%----

\newcommand{\AdventInitial}[2]{%
  \lettrine[lines=2,lhang=0.1,loversize=0.15]%
    {\textcolor{adventred}{#1}}%
    {#2}%
}

%----
% AdventPageBackground
%----

\newcommand{\AdventPageBackground}{%
  \AddToShipoutPictureBG{%
    \begin{tikzpicture}[remember picture,overlay]
    \draw[adventgreen!80!black, line width=3pt, rounded corners=12pt]
    ($(current page.north west)+(0.8cm,-0.8cm)$)
    rectangle
    ($(current page.south east)+(-0.8cm,0.8cm)$);
    \fill[adventgold]
    ($(current page.north west)+(1.0cm,-1.0cm)$) circle (1.2pt)
    ($(current page.north east)+(-1.0cm,-1.0cm)$) circle (1.2pt)
    ($(current page.south west)+(1.0cm,1.0cm)$) circle (1.2pt)
    ($(current page.south east)+(-1.0cm,1.0cm)$) circle (1.2pt);
    \end{tikzpicture}
  }%
}

%---- AdventSheet macro (1-page, single column) ----
% #1: Date + occasion
% #2: unused
% #3: Main title
% #4: Subtitle
% #5: Key Insight
% #6: Main body content
% #7: Closing statement
\newcommand{\AdventSheet}[7]{%
  \BeginAdventPage
  \vspace*{1cm}
  \begin{AdventFrameTop}
    \AdventTitleBlock{#1}{#2}{#3}{#4}
    \AdventKeyInsight{#5}
  \end{AdventFrameTop}
  \AdventStarRule
  #6%
  \EndAdventPage
  \AdventClosing{#7}%
}

%---- AdventSheetTwoCol macro (two-column hero sheet) ----
% #1: Date + occasion
% #2: unused
% #3: Main title
% #4: Subtitle
% #5: Key Insight
% #6: Main body content (wrapped in multicols{2})
% #7: Closing statement
\newcommand{\AdventSheetTwoCol}[7]{%
  \BeginAdventPage
  \begin{AdventFrameTop}
    \AdventTitleBlock{#1}{#2}{#3}{#4}
    \AdventKeyInsight{#5}
  \end{AdventFrameTop}
  \AdventStarRule
  \begin{multicols}{2}
    #6%
  \end{multicols}
  \EndAdventPage
  \AdventClosing{#7}%
}

\begin{document}

\AdventPageBackground
\AdventAuthor

\AdventSheetTwoCol
  {December 3, 2025} % #1 Date
  {}                 % #2 unused
  {G\textsubscript{2} as the minimal exceptional symmetry} % #3 Main title
  {The 14-dimensional gatekeeper of the octonions} % #4 Subtitle
  {G\textsubscript{2} is the smallest exceptional Lie group and the full
   automorphism group of the octonions. Every map in G\textsubscript{2}
   preserves octonionic multiplication and the norm. In the model, this
   makes G\textsubscript{2} the \emph{gatekeeper} of the internal
   structure: any internal operator, symmetry or interaction must respect
   G\textsubscript{2} invariance. Today we meet G\textsubscript{2} as the
   minimal exceptional symmetry from which the larger exceptional group
   F\textsubscript{4} will later emerge.} % #5 Key Insight
  { % #6 body (two columns)

\section*{What is G\texorpdfstring{$_2$}{G2}?}

\AdventInitial{T}{he} group $G_2$ can be defined in many equivalent ways.
For the octonionic story, the most natural is:

\[
  G_2 = \mathrm{Aut}(\mathbb{O}),
\]

the group of all linear transformations of $\mathbb{O}$ that preserve the
octonionic product and the norm. It is a 14-dimensional, compact,
connected, simply-connected Lie group and the smallest of the five
exceptional Lie groups.

Concretely:

\begin{itemize}
  \item $G_2$ preserves the multiplication table of the seven imaginary
        units $e_1,\dots,e_7$.
  \item It preserves the standard norm $|x|^2 = x\bar{x}$.
  \item It acts transitively on the unit sphere of imaginary octonions,
        with stabiliser isomorphic to $SU(3)$.
\end{itemize}

In other words, $G_2$ is the full continuous symmetry group of the
octonionic number system itself.

\section*{Why ``minimal exceptional'' matters}

As a Lie group, $G_2$ is:

\begin{itemize}
  \item too small to host all Standard Model symmetries directly,
  \item but large enough to control the essential nonassociative structure
        of $\mathbb{O}$,
  \item and exceptional—meaning it does not fit into the infinite $A_n$,
        $B_n$, $C_n$, $D_n$ series.
\end{itemize}

This makes $G_2$ an ideal starting point:

\begin{itemize}
  \item It is restrictive enough to strongly constrain internal operators.
  \item It is flexible enough to embed subgroups that resemble
        $SU(3)_C\times SU(2)_L\times U(1)_Y$ in appropriate ways.
  \item It naturally sits inside the larger exceptional group
        $F_4$, the automorphism group of the Albert algebra
        $H_3(\mathbb{O})$.
\end{itemize}

\section*{G\texorpdfstring{$_2$}{G2} as a gatekeeper of allowed operators}

In the model, internal operators (heptagon operator, radius operator,
rotors, compressors) are not arbitrary matrices; they must be compatible
with the $G_2$-structure. Informally:

\begin{quote}
  If an operator would break $G_2$ in an uncontrolled way, it is not part
  of the fundamental toolbox.
\end{quote}

This has two important consequences:

\begin{enumerate}
  \item \textbf{Restricted parameter space:}
    many couplings and mass terms that are allowed in a generic
    field-theory Lagrangian are simply forbidden by $G_2$.
  \item \textbf{Natural subgroups:}
    gauge groups that actually appear (or approximate) in low-energy
    physics are precisely those that can be embedded in $G_2$ (and later
    in $F_4$) in a structurally compatible way.
\end{enumerate}

G\textsubscript{2} thus serves as a first filter between ``any algebraic
construction on $\mathbb{R}^8$'' and ``constructions that respect the
octonionic number system''.

\section*{From G\texorpdfstring{$_2$}{G2} to F\texorpdfstring{$_4$}{F4}}

Later in the calendar, the Albert algebra $H_3(\mathbb{O})$ will appear,
and with it the larger exceptional group $F_4$:

\[
  F_4 = \mathrm{Aut}\big(H_3(\mathbb{O})\big).
\]

The relationship is hierarchical:

\begin{itemize}
  \item $G_2$ controls the algebra of $\mathbb{O}$ itself.
  \item $H_3(\mathbb{O})$ builds $3\times3$ Hermitian matrices over
        $\mathbb{O}$.
  \item $F_4$ controls the automorphisms of this larger Jordan algebra.
\end{itemize}

From the perspective of the calendar:

\begin{itemize}
  \item early days: $G_2$ and triality structure the octonionic stage,
  \item middle days: $F_4$ organises the symmetry atlas on the Albert
        algebra,
  \item later days: potentials and equilibria on this atlas fix physical
        scales and constants.
\end{itemize}

G\textsubscript{2} is the first rung on this exceptional ladder.

\section*{Conceptual gain from G\texorpdfstring{$_2$}{G2}}

Putting $G_2$ at the base of the internal symmetry story has clear
advantages:

\begin{enumerate}
  \item \textbf{Uniqueness:} There is only one real division algebra with
    7 imaginary units, and only one connected Lie group that preserves it:
    $G_2$. This is about as far from ``model building by choice'' as one
    can get.
  \item \textbf{Rigidity:} Once octonions are chosen, $G_2$ is fixed.
    Internal symmetries are no longer free groups to be dialled but are
    inherited from this starting point.
  \item \textbf{Roadmap:} The inclusion $G_2\subset F_4$ provides a clear
    path from basic number system to full symmetry atlas.
\end{enumerate}

In this sense, $G_2$ is not just an exotic group in a classification
table; it is the minimal exceptional guardian of the octonionic world.

\small
\begin{thebibliography}{9}

\bibitem{Engel1900}
F.~Engel,
\newblock ``Ein neues, dem linearen Komplexe analoges Gebilde,''
\newblock {\em Ber.\ Verh.\ K\"onigl.\ S\"achs.\ Ges.\ Wiss.\ Leipzig}
\textbf{52}, 63--74 (1900).

\bibitem{Bryant1987}
R.~L.~Bryant,
\newblock ``Metrics with exceptional holonomy,''
\newblock {\em Ann.\ Math.} \textbf{126}, 525--576 (1987).

\bibitem{Baez2002}
J.~C.~Baez,
\newblock ``The octonions,''
\newblock {\em Bull.\ Amer.\ Math.\ Soc.} \textbf{39}, 145--205 (2002).

\bibitem{Internal}
[Internal notes on $G_2$ and its role in the symmetry atlas:
{\tt chap02\_neu.tex; appB\_neu.tex; oktonionen-basen.tex}.]

\end{thebibliography}
\normalsize

  } % end #6 body
  {G\textsubscript{2} is the 14-dimensional automorphism group of the
   octonions and the minimal exceptional symmetry. In the model it acts as
   a gatekeeper: only operators and symmetries compatible with
   G\textsubscript{2} are admitted into the internal stage on which all
   later structures are built.} % #7 Closing

\end{document}