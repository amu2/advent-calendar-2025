% advent03.tex
% December 3, 2025 – G2 as the minimal exceptional symmetry

\documentclass[a4paper,10pt]{article}

\usepackage[utf8]{inputenc}
\usepackage[T1]{fontenc}
\usepackage[english]{babel}
\usepackage{amsmath,amssymb,amsfonts}
\usepackage{xcolor}
\usepackage{tikz}
\usetikzlibrary{calc}
\usepackage[framemethod=TikZ]{mdframed}
\usepackage{titlesec}
\usepackage{lettrine}
\usepackage{eso-pic}
\usepackage{geometry}
\usepackage{multicol}
\usepackage{lmodern}

\geometry{margin=2.0cm}

% advent-layout.tex (corrected for \input usage)

\definecolor{adventred}{HTML}{B3001B}
\definecolor{adventblue}{HTML}{003366}
\definecolor{adventgreen}{HTML}{006633}
\definecolor{adventgold}{HTML}{B59410}

\pagestyle{empty}

\titleformat{\section}
  {\normalfont\large\bfseries\color{adventblue}}{\thesection}{0em}{}
  
\titleformat{\subsection}
  {\normalfont\normalsize\bfseries\color{adventblue}}{\thesubsection}{0em}{}

%----
% AdventFrameTop
%----

\newenvironment{AdventFrameTop}
{%
  \begin{mdframed}[
    linecolor=adventgreen!0,
    linewidth=0pt,
    roundcorner=0pt,
    innertopmargin=10pt,
    innerbottommargin=10pt,
    innerleftmargin=10pt,
    innerrightmargin=10pt,
    backgroundcolor=adventgreen!2
  ]%
}
{%
  \end{mdframed}
}

\newcommand{\BeginAdventPage}{}
\newcommand{\EndAdventPage}{}

%----
% AdventTitleBlock
%----

\newcommand{\AdventTitleBlock}[4]{%
  \begin{center}
    {\Large\textcolor{adventred}{\textbf{#1}}}\par\vspace{4pt}%
    \ifx&#2&\else
      {\large\textbf{#2}}\par\vspace{2pt}%
    \fi
    {\Large\textcolor{adventblue}{\textbf{#3}}}\par
    \ifx&#4&\else
      \vspace{2pt}%
      {\normalsize\textbf{#4}}\par%
    \fi
  \end{center}%
}

%----
% AdventKeyInsight
%----

\newcommand{\AdventKeyInsight}[1]{%
  \vspace{0.5em}%
  \noindent\colorbox{adventred!8}{%
    \parbox{\dimexpr\linewidth-2\fboxsep}{%
    \textbf{\textcolor{adventred}{Key Insight.}}~#1%
    }%
  }%
  \vspace{0.5em}%
}

%----
% AdventStarRule
%----

\newcommand{\AdventStarRule}{%
  \vspace{0.3em}%
  \begin{center}
    {\color{adventgold}%
    \rule[0.5ex]{0.25\linewidth}{0.4pt}\;
    $\ast\;\ast\;\ast$\;
    \rule[0.5ex]{0.25\linewidth}{0.4pt}%
    }%
  \end{center}
  \vspace{0.3em}%
}

%----
% AdventClosing
%----

\newcommand{\AdventClosing}[1]{%
  \vspace{0.4em}%
  \begin{center}
    \textcolor{adventgreen}{\emph{#1}}%
  \end{center}
}

%----
% AdventAuthor
%----

\newcommand{\AdventAuthor}{%
  \AddToShipoutPictureFG{%
    \begin{tikzpicture}[remember picture,overlay]
    \node[anchor=south, yshift=2mm] at (current page.south) {%
    \footnotesize Andreas Müller, Kempten University of Applied Sciences, %
    \texttt{andreas.mueller@hs-kempten.de}%
    };
    \end{tikzpicture}%
  }%
}

%----
% AdventInitial
%----

\newcommand{\AdventInitial}[2]{%
  \lettrine[lines=2,lhang=0.1,loversize=0.15]%
    {\textcolor{adventred}{#1}}%
    {#2}%
}

%----
% AdventPageBackground
%----

\newcommand{\AdventPageBackground}{%
  \AddToShipoutPictureBG{%
    \begin{tikzpicture}[remember picture,overlay]
    \draw[adventgreen!80!black, line width=3pt, rounded corners=12pt]
    ($(current page.north west)+(0.8cm,-0.8cm)$)
    rectangle
    ($(current page.south east)+(-0.8cm,0.8cm)$);
    \fill[adventgold]
    ($(current page.north west)+(1.0cm,-1.0cm)$) circle (1.2pt)
    ($(current page.north east)+(-1.0cm,-1.0cm)$) circle (1.2pt)
    ($(current page.south west)+(1.0cm,1.0cm)$) circle (1.2pt)
    ($(current page.south east)+(-1.0cm,1.0cm)$) circle (1.2pt);
    \end{tikzpicture}
  }%
}

%---- AdventSheet macro (1-page, single column) ----
% #1: Date + occasion
% #2: unused
% #3: Main title
% #4: Subtitle
% #5: Key Insight
% #6: Main body content
% #7: Closing statement
\newcommand{\AdventSheet}[7]{%
  \BeginAdventPage
  \vspace*{1cm}
  \begin{AdventFrameTop}
    \AdventTitleBlock{#1}{#2}{#3}{#4}
    \AdventKeyInsight{#5}
  \end{AdventFrameTop}
  \AdventStarRule
  #6%
  \EndAdventPage
  \AdventClosing{#7}%
}

%---- AdventSheetTwoCol macro (two-column hero sheet) ----
% #1: Date + occasion
% #2: unused
% #3: Main title
% #4: Subtitle
% #5: Key Insight
% #6: Main body content (wrapped in multicols{2})
% #7: Closing statement
\newcommand{\AdventSheetTwoCol}[7]{%
  \BeginAdventPage
  \begin{AdventFrameTop}
    \AdventTitleBlock{#1}{#2}{#3}{#4}
    \AdventKeyInsight{#5}
  \end{AdventFrameTop}
  \AdventStarRule
  \begin{multicols}{2}
    #6%
  \end{multicols}
  \EndAdventPage
  \AdventClosing{#7}%
}

\begin{document}

\AdventPageBackground
\AdventAuthor

\AdventSheetTwoCol
  {December 3, 2025} % #1 Date
  {}    % #2 unused
  {G2 as the minimal exceptional symmetry} % #3 Main title
  {The 14-dimensional gatekeeper of the octonions}    % #4 Subtitle
  {G\textsubscript{2} is the smallest exceptional Lie group and the full
   automorphism group of the octonions. Every map in G\textsubscript{2}
   preserves octonionic multiplication and the norm. In the model, this
   makes G\textsubscript{2} the \emph{gatekeeper} of the internal
   structure: any internal operator, symmetry or interaction must respect
   G\textsubscript{2} invariance. Today we meet G\textsubscript{2} as the
   minimal exceptional symmetry from which the larger exceptional group
   F\textsubscript{4} will later emerge.} % #5 Key Insight
  { % #6 body (two columns)

\AdventInitial{E}{very} number system has a symmetry group—the set of transformations that preserve its essential structure. For the real numbers $\mathbb{R}$, the symmetry is almost trivial: only multiplication by $\pm1$. For the complex numbers $\mathbb{C}$, it is the unit circle $U(1)$ of  complex numbers of modulus one, isomorphic to rotations in the plane and representable by real $2 \times 2$ rotation matrices.
For the quaternions $\mathbb{H}$, it's the 3-sphere of unit quaternions, which is isomorphic to the special unitary group $SU(2)$ of complex $2 \times 2$ matrices, as we saw on Dec 1.



What about the octonions? The symmetry group of the octonions—the transformations that preserve both the multiplication table and the norm—is called $G_2$. It's a 14-dimensional Lie group, and it's the smallest of the five exceptional Lie groups $G_2, F_4, E_6, E_7, E_8$. ``Exceptional'' means it doesn't fit into the infinite families of classical groups like $SU(n+1)=A_{n}$, $SO(2n+1)=B_n$, $SO(2n)=D_n$, or $Sp(n)=C_n$ (special unitary, special orthogonal or symplectic groups). It's a one-of-a-kind structure.

Why does this matter for physics? Because $G_2$ acts as a gatekeeper. If we're building a model where internal degrees of freedom live in an octonionic space, then any operator, any symmetry, any interaction must respect the $G_2$ structure. You can't just write down arbitrary matrices and hope they make sense — they have to be compatible with the rigid multiplication rules of the octonions.

This is a feature, not a bug. In conventional field theory, we have enormous freedom to choose gauge groups, representations, and couplings. This freedom is both a blessing and a curse: it allows us to fit data, but it doesn't explain why nature chose one particular set of parameters over another. With $G_2$ as the starting point, much of that freedom disappears. The structure is forced on us by the choice of octonions.

Here's a concrete example: $G_2$ contains $SU(3)$ as a subgroup. In fact, if you pick any imaginary octonion and ask which transformations leave it fixed, you get a copy of $SU(3)$. This is not a coincidence — it's a hint that the color symmetry of the strong force might be a natural subgroup of the octonionic automorphism group.

But $G_2$ is not large enough to contain the full Standard Model gauge group $SU(3) \times SU(2) \times U(1)$ directly. For that, we'll need to move to a larger structure: the Albert algebra $H_3(\mathbb{O})$ and its automorphism group $F_4$, which we'll meet in the coming days. $G_2$ is the first rung on the exceptional ladder — the minimal exceptional symmetry that controls the octonionic stage itself.

Think of today's sheet as introducing the guardian of the internal space. $G_2$ is not just an exotic group in a classification table. It's the symmetry that makes the octonions work as a number system, and in the model we're building, it's the symmetry that constrains which internal structures are allowed and which are forbidden.

Let us see how $G_2$ guards the octonionic multiplication table.

\section*{What is $G_2$?}

The group $G_2$ can be defined in many equivalent ways.
For the octonionic story, the most natural is:

\[
  G_2 = \mathrm{Aut}(\mathbb{O}),
\]
the group of all real-linear transformations of $\mathbb{O}$ that preserve
the octonionic product (and hence the standard norm). It is a 14-dimensional,
compact, connected, simply connected simple Lie group and the smallest of the
five exceptional Lie groups.

Concretely:
\begin{itemize}
    \item $G_2$ preserves the multiplication rules among the seven imaginary units $e_1,\dots,e_7$.
    \item It preserves the standard norm $|x|^2 = x\bar{x}$.
    \item It acts transitively on the unit sphere of imaginary octonions,
          with stabiliser isomorphic to $SU(3)$.
\end{itemize}



Equivalently, one can view $G_2$ as the subgroup of $SO(7)$ that preserves
both the Euclidean inner product and a distinguished $3$-form (or,
equivalently, a cross product) on the $7$-dimensional space $\mathbb{R}^7$ of imaginary
octonions (those with vanishing real coordinates).


In representation-theoretic terms, $G_2$ has rank~$2$, with two smallest
non-trivial representations:
\begin{itemize}
    \item the $7$-dimensional fundamental representation on the imaginary
          octonions, and
    \item the $14$-dimensional adjoint representation on its Lie algebra
          $\mathfrak{g}_2$.
\end{itemize}
Under the $SU(3)$ subgroup that fixes a chosen imaginary unit, these
representations decompose as
\[
  7 \;\cong\; 1 \oplus 3 \oplus \bar{3}, \qquad
  14 \;\cong\; 8 \oplus 3 \oplus \bar{3},
\]
a pattern that already hints at colour-like structures.

In other words, $G_2$ is the full continuous symmetry group of the
octonionic number system itself.

\section*{Why ``minimal exceptional'' matters}

As a Lie group, $G_2$ is:

\begin{itemize}
    \item too small to host all Standard Model symmetries directly,
    \item but large enough to control the essential nonassociative structure
        of $\mathbb{O}$,
    \item and exceptional — meaning it does not fit into the infinite $A_n$,
        $B_n$, $C_n$, $D_n$ series.
\end{itemize}
This makes $G_2$ an ideal starting point:

\begin{itemize}
    \item It is restrictive enough to strongly constrain internal operators.
    \item It is flexible enough to embed subgroups that resemble the Standard Model structure
        $SU(3)_C\times SU(2)_L\times U(1)_Y$ in appropriate ways.
    \item It naturally sits inside the larger exceptional group
        $F_4$, the automorphism group of the Albert algebra
        $H_3(\mathbb{O})$.
\end{itemize}

\section*{G$_2$ as a gatekeeper of allowed operators}

In the model, internal operators (heptagon operator, radius operator,
rotors, compressors) are not arbitrary matrices; they must be compatible
with the $G_2$ structure. Informally:

\begin{quote}
  If an operator would break $G_2$ in an uncontrolled way, it is not part
  of the fundamental toolbox.
\end{quote}
This has two important consequences:

\begin{enumerate}
    \item \textbf{Restricted parameter space:}
    many couplings and mass terms that are allowed in a generic
    field-theory Lagrangian are simply forbidden by $G_2$.
    \item \textbf{Natural subgroups:}
    gauge groups that actually appear (or approximate) in low-energy
    physics are required to arise as structurally compatible subgroups of $G_2$ (and later
    in $F_4$).
\end{enumerate}

$G_2$ thus serves as a first filter between ``any algebraic
construction on $\mathbb{R}^8$'' and ``constructions that respect the
octonionic number system''.

\section*{From G$_2$ to F$_4$}

Later in the calendar, the Albert algebra $H_3(\mathbb{O})$ will appear,
and with it the larger exceptional group $F_4$:

\[
  F_4 = \mathrm{Aut}\big(H_3(\mathbb{O})\big).
\]
The relationship is hierarchical:

\begin{itemize}
    \item $G_2$ controls the algebra of $\mathbb{O}$ itself.
    \item $H_3(\mathbb{O})$ builds $3\times3$ Hermitian matrices over
        $\mathbb{O}$.
    \item $F_4$ controls the automorphisms of this larger Jordan algebra $H_3(\mathbb{O})$.
\end{itemize}
%From the perspective of the calendar:
%
%\begin{itemize}
%    \item early days: $G_2$ and triality structure the octonionic stage,
%    \item middle days: $F_4$ organises the symmetry atlas on the Albert
%        algebra,
%    \item later days: potentials and equilibria on this atlas fix physical
%        scales and constants.
%\end{itemize}
%
%G2 is the first rung on this exceptional ladder.

%\section*{Conceptual gain from G$_2$}
%
%Putting $G_2$ at the base of the internal symmetry story has clear
%advantages:
%
%\begin{enumerate}
%    \item \textbf{Uniqueness:} There is only one real division algebra with
%    7 imaginary units, and only one connected Lie group that preserves it:
%    $G_2$. This is about as far from ``model building by choice'' as one
%    can get.
%    \item \textbf{Rigidity:} Once octonions are chosen, $G_2$ is fixed.
%    Internal symmetries are no longer free groups to be dialled but are
%    inherited from this starting point.
%    \item \textbf{Roadmap:} The inclusion $G_2\subset F_4$ provides a clear
%    path from basic number system to full symmetry atlas.
%\end{enumerate}
%
%In this sense, $G_2$ is not just an exotic group in a classification
%table; it is the minimal exceptional guardian of the octonionic world.




\small
\begin{thebibliography}{9}


\bibitem{Engel1900}
F.~Engel, ``Ein neues, dem linearen Komplexe analoges Gebilde,'' Ber.\ Verh.\ K\"onigl.\ S\"achs.\ Ges.\ Wiss.\ Leipzig 52, 63--74 (1900).


\bibitem{Bryant1987}
R.~L.~Bryant, ``Metrics with exceptional holonomy,'' Ann.\ Math. 126, 525--576 (1987).


\bibitem{Baez2002}
J.~C.~Baez, ``The octonions,'' Bull.\ Amer.\ Math.\ Soc. 39, 145--205 (2002).

%
%\bibitem{Internal}
%[Internal notes on $G_2$ and its role in the symmetry atlas: chap02\_neu.tex; appB\_neu.tex; oktonionen-basen.tex.] 


\end{thebibliography}
\normalsize

  } % end #6 body
  {G\textsubscript{2} is the 14-dimensional automorphism group of the
   octonions and the minimal exceptional symmetry. In the model it acts as
   a gatekeeper: only operators and symmetries compatible with
   G\textsubscript{2} are admitted into the internal stage on which all
   later structures are built.} % #7 Closing

\end{document}