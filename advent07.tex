% advent07.tex
% Second Advent Sunday: December 7, 2025 – Operator toolbox

\documentclass[a4paper,10pt]{article}

\usepackage[utf8]{inputenc}
\usepackage[T1]{fontenc}
\usepackage[english]{babel}
\usepackage{amsmath,amssymb,amsfonts}
\usepackage{xcolor}
\usepackage{tikz}
\usetikzlibrary{calc}
\usepackage[framemethod=TikZ]{mdframed}
\usepackage{titlesec}
\usepackage{lettrine}
\usepackage{eso-pic}
\usepackage{geometry}
\usepackage{multicol}
\usepackage{lmodern}

\geometry{margin=2.0cm}

% advent-layout.tex (corrected for \input usage)

\definecolor{adventred}{HTML}{B3001B}
\definecolor{adventblue}{HTML}{003366}
\definecolor{adventgreen}{HTML}{006633}
\definecolor{adventgold}{HTML}{B59410}

\pagestyle{empty}

\titleformat{\section}
  {\normalfont\large\bfseries\color{adventblue}}{\thesection}{0em}{}
  
\titleformat{\subsection}
  {\normalfont\normalsize\bfseries\color{adventblue}}{\thesubsection}{0em}{}

%----
% AdventFrameTop
%----

\newenvironment{AdventFrameTop}
{%
  \begin{mdframed}[
    linecolor=adventgreen!0,
    linewidth=0pt,
    roundcorner=0pt,
    innertopmargin=10pt,
    innerbottommargin=10pt,
    innerleftmargin=10pt,
    innerrightmargin=10pt,
    backgroundcolor=adventgreen!2
  ]%
}
{%
  \end{mdframed}
}

\newcommand{\BeginAdventPage}{}
\newcommand{\EndAdventPage}{}

%----
% AdventTitleBlock
%----

\newcommand{\AdventTitleBlock}[4]{%
  \begin{center}
    {\Large\textcolor{adventred}{\textbf{#1}}}\par\vspace{4pt}%
    \ifx&#2&\else
      {\large\textbf{#2}}\par\vspace{2pt}%
    \fi
    {\Large\textcolor{adventblue}{\textbf{#3}}}\par
    \ifx&#4&\else
      \vspace{2pt}%
      {\normalsize\textbf{#4}}\par%
    \fi
  \end{center}%
}

%----
% AdventKeyInsight
%----

\newcommand{\AdventKeyInsight}[1]{%
  \vspace{0.5em}%
  \noindent\colorbox{adventred!8}{%
    \parbox{\dimexpr\linewidth-2\fboxsep}{%
    \textbf{\textcolor{adventred}{Key Insight.}}~#1%
    }%
  }%
  \vspace{0.5em}%
}

%----
% AdventStarRule
%----

\newcommand{\AdventStarRule}{%
  \vspace{0.3em}%
  \begin{center}
    {\color{adventgold}%
    \rule[0.5ex]{0.25\linewidth}{0.4pt}\;
    $\ast\;\ast\;\ast$\;
    \rule[0.5ex]{0.25\linewidth}{0.4pt}%
    }%
  \end{center}
  \vspace{0.3em}%
}

%----
% AdventClosing
%----

\newcommand{\AdventClosing}[1]{%
  \vspace{0.4em}%
  \begin{center}
    \textcolor{adventgreen}{\emph{#1}}%
  \end{center}
}

%----
% AdventAuthor
%----

\newcommand{\AdventAuthor}{%
  \AddToShipoutPictureFG{%
    \begin{tikzpicture}[remember picture,overlay]
    \node[anchor=south, yshift=2mm] at (current page.south) {%
    \footnotesize Andreas Müller, Kempten University of Applied Sciences, %
    \texttt{andreas.mueller@hs-kempten.de}%
    };
    \end{tikzpicture}%
  }%
}

%----
% AdventInitial
%----

\newcommand{\AdventInitial}[2]{%
  \lettrine[lines=2,lhang=0.1,loversize=0.15]%
    {\textcolor{adventred}{#1}}%
    {#2}%
}

%----
% AdventPageBackground
%----

\newcommand{\AdventPageBackground}{%
  \AddToShipoutPictureBG{%
    \begin{tikzpicture}[remember picture,overlay]
    \draw[adventgreen!80!black, line width=3pt, rounded corners=12pt]
    ($(current page.north west)+(0.8cm,-0.8cm)$)
    rectangle
    ($(current page.south east)+(-0.8cm,0.8cm)$);
    \fill[adventgold]
    ($(current page.north west)+(1.0cm,-1.0cm)$) circle (1.2pt)
    ($(current page.north east)+(-1.0cm,-1.0cm)$) circle (1.2pt)
    ($(current page.south west)+(1.0cm,1.0cm)$) circle (1.2pt)
    ($(current page.south east)+(-1.0cm,1.0cm)$) circle (1.2pt);
    \end{tikzpicture}
  }%
}

%---- AdventSheet macro (1-page, single column) ----
% #1: Date + occasion
% #2: unused
% #3: Main title
% #4: Subtitle
% #5: Key Insight
% #6: Main body content
% #7: Closing statement
\newcommand{\AdventSheet}[7]{%
  \BeginAdventPage
  \vspace*{1cm}
  \begin{AdventFrameTop}
    \AdventTitleBlock{#1}{#2}{#3}{#4}
    \AdventKeyInsight{#5}
  \end{AdventFrameTop}
  \AdventStarRule
  #6%
  \EndAdventPage
  \AdventClosing{#7}%
}

%---- AdventSheetTwoCol macro (two-column hero sheet) ----
% #1: Date + occasion
% #2: unused
% #3: Main title
% #4: Subtitle
% #5: Key Insight
% #6: Main body content (wrapped in multicols{2})
% #7: Closing statement
\newcommand{\AdventSheetTwoCol}[7]{%
  \BeginAdventPage
  \begin{AdventFrameTop}
    \AdventTitleBlock{#1}{#2}{#3}{#4}
    \AdventKeyInsight{#5}
  \end{AdventFrameTop}
  \AdventStarRule
  \begin{multicols}{2}
    #6%
  \end{multicols}
  \EndAdventPage
  \AdventClosing{#7}%
}

\begin{document}

\AdventPageBackground
\AdventAuthor

\AdventSheetTwoCol
  {December 7, 2025 \large (Second Advent Sunday)} % #1 Date + occasion
  {}                                               % #2 unused
  {Heptagon, Radii and Attractor}                  % #3 Main title
  {The rotor/compressor toolbox for all observables} % #4 Subtitle
  {In the octonionic model, all observable quantities—couplings, masses,
   mixings and scales—are traced back to invariants of two operator
   families on the internal space: antisymmetric \emph{rotors} (forces)
   and symmetric \emph{compressors} (masses and mixings). The heptagon
   operator encodes the seven imaginary octonion directions; the radius
   operator $R$ with spectrum $(a_0,b_0,c_0)$ defines an attractor
   mechanism for fundamental scales. Together they form a minimal but
   sufficient operator toolbox: nothing else is added by hand.} % #5 Key Insight
  { % #6 body (two columns)

\section*{From algebra to operators}

\AdventInitial{O}{ctonions} and their automorphism group $G_2$ give us a
rigid internal stage. But physical predictions are not read directly from
the multiplication table; they arise from \emph{operators} acting on the
internal space. In this model, two operator families play the central role:

\begin{itemize}
  \item \textbf{Rotors} — antisymmetric operators $G_a$ generating internal
        rotations: they encode forces and couplings.
  \item \textbf{Compressors} — symmetric operators $C$ with real spectra:
        they encode masses and mixing patterns.
\end{itemize}

Once this toolbox is in place, every later sheet becomes a story about
eigenvalues, eigenvectors, commutators and norms of these operators.

\section*{The heptagon operator: seven directions, three invariants}

The seven imaginary octonion units are arranged on the Fano-plane
heptagon. Instead of handling them one by one, the model packages them
into a single \emph{heptagon operator} $H_7$ acting on the internal space:
\[
  H_7 \;=\; \sum_{i=1}^7 c_i E_i,
\]
where the $E_i$ encode the seven directions and the $c_i$ are fixed by the
vacuum and $G_2$-symmetry.

Despite being built from seven directions, $H_7$ has only three independent
eigenvalues,
\[
  \mathrm{Spec}(H_7) = (\alpha,\beta,\gamma),
\]
which are invariants of the $G_2$-orbit of $H_7$. These three numbers will
reappear as seeds for:

\begin{itemize}
  \item gauge couplings (fine-structure, strong coupling, weak mixing),
  \item relative positions of flavour sectors,
  \item and parts of the hierarchy structure.
\end{itemize}

\section*{The radius operator and three fundamental scales}

From the heptagon structure one constructs a \emph{radius operator} $R$.
It measures how far internal directions sit from preferred axes. Its
spectrum is
\[
  \mathrm{Spec}(R) = (a_0,b_0,c_0),
  \quad a_0 > b_0 > c_0.
\]

Exponentials of these radii define three characteristic energy scales:
\[
  E_{\text{Planck}} \sim e^{a_0},\qquad
  E_{\text{EW}}     \sim e^{b_0},\qquad
  E_{\text{QCD}}    \sim e^{c_0}.
\]
Thus, the familiar hierarchy of Planck, electroweak and QCD scales is
encoded in a few geometric invariants of $R$ rather than inserted as three
independent inputs.

\section*{Rotors: forces from commutator norms}

Rotors are antisymmetric operators $G_a$ generating continuous internal
symmetries. Their commutators measure how two internal directions fail to
commute. The squared norm of a commutator,
\[
  \bigl\|[G_a,G_b]\bigr\|^2,
\]
acts as the prototype for a coupling constant. Symbolically,
\[
  \alpha \sim \bigl\|[G_{\text{em}},G_{\text{ref}}]\bigr\|^2,\quad
  \alpha_s \sim \bigl\|[G_{\text{color}},G_{\text{ref}}]\bigr\|^2,\quad
  \sin^2\theta_W \sim \bigl\|[G_{\text{weak}},G_{\text{ref}}]\bigr\|^2.
\]

Choosing different rotor pairs recovers different interactions. In this
view, a ``strong'' force is literally a large commutator norm in the
internal algebra; a ``weak'' one corresponds to nearly commuting rotors.

\section*{Compressors: masses and mixings as spectra}

Compressors are symmetric operators with real eigenvalues. The most
important example is the mass map $\Pi(\langle H\rangle)$ constructed from
a vacuum configuration $\langle H\rangle \in H_3(\mathbb{O})$:
\[
  \Pi(\langle H\rangle)\Psi = m\,\Psi.
\]

Its eigenvalues $m$ provide prototype fermion masses; its eigenvectors
define the associated mass eigenstates. Additional compressors act in
flavour subspaces; misalignment between their eigenbases produces mixing
matrices:

\begin{itemize}
  \item down- vs. up-quark compressors $\Rightarrow$ CKM matrix,
  \item charged-lepton vs. neutrino compressors $\Rightarrow$ PMNS matrix.
\end{itemize}

Masses and mixings thus share a common origin: they are different ways of
reading the same symmetric operators.

\section*{Attractor behaviour of scales and hierarchies}

The combination of radius operator and compressors suggests an
\emph{attractor} picture:

\begin{itemize}
  \item The spectrum $(a_0,b_0,c_0)$ singles out preferred scales.
  \item Renormalisation-group flow tends to pull effective parameters
        towards these scales.
  \item Small deformations of $\langle H\rangle$ move eigenvalues, but
        certain patterns (like the large gap between Planck and EW) remain
        stable.
\end{itemize}

Hierarchies thus become \emph{fixed points} of a dynamical process in
operator space, not arbitrary distances between hand-picked numbers.

\section*{Why this toolbox is minimal and sufficient}

What makes the toolbox attractive is its balance between simplicity and
power:

\begin{itemize}
  \item It is \emph{minimal}: no independent Yukawa matrices, no
        unmotivated extra symmetries, no long list of unrelated constants.
  \item It is \emph{sufficient}: in principle, all quantities that enter
        phenomenology—masses, mixing angles, couplings, scales—can be
        expressed in terms of the invariants of these operators.
\end{itemize}

The remaining Advent days flesh out this claim with concrete examples:
from specific couplings (fine-structure, strong, Weinberg angle) to
numerical mass prototypes and flavour structures.

\small
\begin{thebibliography}{9}

\bibitem{DrayManogue1999}
T.~Dray and C.~A.~Manogue,
\newblock {\em The Geometry of the Octonions},
\newblock World Scientific, 1999.

\bibitem{Dixon1994}
G.~M.~Dixon,
\newblock {\em Division Algebras: Octonions, Quaternions, Complex Numbers and
the Algebraic Design of Physics},
\newblock Kluwer, 1994.

\bibitem{ChamseddineConnesMarcolli2007}
A.~H.~Chamseddine, A.~Connes and M.~Marcolli,
\newblock ``Gravity and the standard model with neutrino mixing,''
\newblock {\em Adv.\ Theor.\ Math.\ Phys.} \textbf{11}, 991--1089 (2007).

\end{thebibliography}
\normalsize

  } % end #6 body
  {One toolbox, two operator families: rotors and compressors turn octonionic geometry into concrete numbers for couplings, masses and scales.} % #7 Closing

\end{document}