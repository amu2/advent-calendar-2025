% advent21.tex
% Fourth Advent Sunday: December 21, 2025 – AQFT variant & F4 potential

\documentclass[a4paper,10pt]{article}

\usepackage[utf8]{inputenc}
\usepackage[T1]{fontenc}
\usepackage[english]{babel}
\usepackage{amsmath,amssymb,amsfonts}
\usepackage{xcolor}
\usepackage{tikz}
\usetikzlibrary{calc}
\usepackage[framemethod=TikZ]{mdframed}
\usepackage{titlesec}
\usepackage{lettrine}
\usepackage{eso-pic}
\usepackage{geometry}
\usepackage{multicol}
\usepackage{lmodern}

\geometry{margin=2.0cm}

% advent-layout.tex (corrected for \input usage)

\definecolor{adventred}{HTML}{B3001B}
\definecolor{adventblue}{HTML}{003366}
\definecolor{adventgreen}{HTML}{006633}
\definecolor{adventgold}{HTML}{B59410}

\pagestyle{empty}

\titleformat{\section}
  {\normalfont\large\bfseries\color{adventblue}}{\thesection}{0em}{}
  
\titleformat{\subsection}
  {\normalfont\normalsize\bfseries\color{adventblue}}{\thesubsection}{0em}{}

%----
% AdventFrameTop
%----

\newenvironment{AdventFrameTop}
{%
  \begin{mdframed}[
    linecolor=adventgreen!0,
    linewidth=0pt,
    roundcorner=0pt,
    innertopmargin=10pt,
    innerbottommargin=10pt,
    innerleftmargin=10pt,
    innerrightmargin=10pt,
    backgroundcolor=adventgreen!2
  ]%
}
{%
  \end{mdframed}
}

\newcommand{\BeginAdventPage}{}
\newcommand{\EndAdventPage}{}

%----
% AdventTitleBlock
%----

\newcommand{\AdventTitleBlock}[4]{%
  \begin{center}
    {\Large\textcolor{adventred}{\textbf{#1}}}\par\vspace{4pt}%
    \ifx&#2&\else
      {\large\textbf{#2}}\par\vspace{2pt}%
    \fi
    {\Large\textcolor{adventblue}{\textbf{#3}}}\par
    \ifx&#4&\else
      \vspace{2pt}%
      {\normalsize\textbf{#4}}\par%
    \fi
  \end{center}%
}

%----
% AdventKeyInsight
%----

\newcommand{\AdventKeyInsight}[1]{%
  \vspace{0.5em}%
  \noindent\colorbox{adventred!8}{%
    \parbox{\dimexpr\linewidth-2\fboxsep}{%
    \textbf{\textcolor{adventred}{Key Insight.}}~#1%
    }%
  }%
  \vspace{0.5em}%
}

%----
% AdventStarRule
%----

\newcommand{\AdventStarRule}{%
  \vspace{0.3em}%
  \begin{center}
    {\color{adventgold}%
    \rule[0.5ex]{0.25\linewidth}{0.4pt}\;
    $\ast\;\ast\;\ast$\;
    \rule[0.5ex]{0.25\linewidth}{0.4pt}%
    }%
  \end{center}
  \vspace{0.3em}%
}

%----
% AdventClosing
%----

\newcommand{\AdventClosing}[1]{%
  \vspace{0.4em}%
  \begin{center}
    \textcolor{adventgreen}{\emph{#1}}%
  \end{center}
}

%----
% AdventAuthor
%----

\newcommand{\AdventAuthor}{%
  \AddToShipoutPictureFG{%
    \begin{tikzpicture}[remember picture,overlay]
    \node[anchor=south, yshift=2mm] at (current page.south) {%
    \footnotesize Andreas Müller, Kempten University of Applied Sciences, %
    \texttt{andreas.mueller@hs-kempten.de}%
    };
    \end{tikzpicture}%
  }%
}

%----
% AdventInitial
%----

\newcommand{\AdventInitial}[2]{%
  \lettrine[lines=2,lhang=0.1,loversize=0.15]%
    {\textcolor{adventred}{#1}}%
    {#2}%
}

%----
% AdventPageBackground
%----

\newcommand{\AdventPageBackground}{%
  \AddToShipoutPictureBG{%
    \begin{tikzpicture}[remember picture,overlay]
    \draw[adventgreen!80!black, line width=3pt, rounded corners=12pt]
    ($(current page.north west)+(0.8cm,-0.8cm)$)
    rectangle
    ($(current page.south east)+(-0.8cm,0.8cm)$);
    \fill[adventgold]
    ($(current page.north west)+(1.0cm,-1.0cm)$) circle (1.2pt)
    ($(current page.north east)+(-1.0cm,-1.0cm)$) circle (1.2pt)
    ($(current page.south west)+(1.0cm,1.0cm)$) circle (1.2pt)
    ($(current page.south east)+(-1.0cm,1.0cm)$) circle (1.2pt);
    \end{tikzpicture}
  }%
}

%---- AdventSheet macro (1-page, single column) ----
% #1: Date + occasion
% #2: unused
% #3: Main title
% #4: Subtitle
% #5: Key Insight
% #6: Main body content
% #7: Closing statement
\newcommand{\AdventSheet}[7]{%
  \BeginAdventPage
  \vspace*{1cm}
  \begin{AdventFrameTop}
    \AdventTitleBlock{#1}{#2}{#3}{#4}
    \AdventKeyInsight{#5}
  \end{AdventFrameTop}
  \AdventStarRule
  #6%
  \EndAdventPage
  \AdventClosing{#7}%
}

%---- AdventSheetTwoCol macro (two-column hero sheet) ----
% #1: Date + occasion
% #2: unused
% #3: Main title
% #4: Subtitle
% #5: Key Insight
% #6: Main body content (wrapped in multicols{2})
% #7: Closing statement
\newcommand{\AdventSheetTwoCol}[7]{%
  \BeginAdventPage
  \begin{AdventFrameTop}
    \AdventTitleBlock{#1}{#2}{#3}{#4}
    \AdventKeyInsight{#5}
  \end{AdventFrameTop}
  \AdventStarRule
  \begin{multicols}{2}
    #6%
  \end{multicols}
  \EndAdventPage
  \AdventClosing{#7}%
}

\begin{document}

\AdventPageBackground
\AdventAuthor

\AdventSheetTwoCol
  {December 21, 2025 \large (Fourth Advent Sunday)} % #1 Date + occasion
  {}                                               % #2 unused
  {AQFT variant and $F_4$ potential}               % #3 Main title
  {Quantisation and electroweak scale equilibrium} % #4 Subtitle
  {An octonionic model of particle physics must pass two hard tests:
   it has to admit a mathematically clean quantisation, and it has to
   explain why the electroweak scale sits where it does. Both issues
   meet on this fourth Advent Sunday. An algebraic quantum field theory
   (AQFT) formulation replaces Hilbert-space dogma by local algebras on
   the octonionic stage, while an $F_4$-symmetric potential fixes the
   electroweak scale $Y_S$ as a true equilibrium quantity:
   $$Y_S^2 = -\frac{\mu^2}{2(\lambda+\kappa c)}.$$} % #5 Key Insight
  { % #6 body (two columns)

\section*{QFT without Hilbert-space dogma}

\AdventInitial{S}{tandard} quantum field theory is usually presented as a
story about fields on spacetime, quantised on a Hilbert space and expanded
in creation and annihilation operators. For highly nontrivial internal
structures like octonions and the Albert algebra, this picture is more a
hindrance than a help.

Algebraic quantum field theory (AQFT) offers a different starting point:

\begin{itemize}
  \item The fundamental objects are \emph{local *-algebras}
        $\mathcal{A}(\mathcal{O})$ assigned to spacetime regions
        $\mathcal{O}$.
  \item States are positive linear functionals
        $\omega: \mathcal{A}(\mathcal{O}) \to \mathbb{C}$, not vectors in a
        pre-chosen Hilbert space.
  \item Dynamics and symmetries act as automorphisms of the net
        $\mathcal{O} \mapsto \mathcal{A}(\mathcal{O})$.
\end{itemize}

This fits the octonionic framework naturally:

\begin{itemize}
  \item The internal degrees of freedom live in nonassociative structures;
        their observables sit more comfortably in operator algebras than in
        a single global Hilbert basis.
  \item Locality and covariance are built into the net of algebras, not
        into a particular field representation.
\end{itemize}

\section*{Local algebras on an octonionic stage}

In the present model, the internal octonionic/Albert structure augments
ordinary spacetime. A minimal AQFT-compatible picture is:

\begin{itemize}
  \item To each spacetime region $\mathcal{O}$ we assign a local algebra
        $\mathcal{A}(\mathcal{O})$ generated by
        \begin{itemize}
          \item rotor operators (internal symmetry generators),
          \item compressor-derived fields (mass/mixing structures),
          \item and possibly additional scalar modes related to the
                Jordan potential.
        \end{itemize}
  \item The net $\mathcal{O}\mapsto\mathcal{A}(\mathcal{O})$ satisfies the
        Haag–Kastler axioms: isotony, locality (commutativity at spacelike
        separation), covariance under the relevant spacetime symmetries.
\end{itemize}

Quantisation in this picture means: we specify the algebra and its
relations, then look at representations (states) that realise this net on
Hilbert spaces \emph{a posteriori}. This reverses the usual logic:

\begin{itemize}
  \item Instead of quantising classical fields, we start from an operator
        algebra informed by octonionic geometry.
  \item Hilbert spaces appear only as GNS completions of chosen states, not
        as primary input.
\end{itemize}

\section*{The $F_4$-symmetric potential}

Beyond the kinematics of local algebras, we need dynamics and vacuum
structure. The central object is a Jordan element $H \in H_3(\mathbb{O})$
and a potential $V_J(H)$ that is invariant under an $F_4$ symmetry acting
on the Albert algebra.

A typical $F_4$-invariant potential has the schematic form
\[
  V_J(H) \;=\; \mu^2\,\mathrm{Tr}(H^2)
                + \lambda\,\mathrm{Tr}(H^4)
                + \kappa\,c\,(\det H)
                + \cdots,
\]
where the invariants (trace, determinant, higher Jordan invariants) are
organised such that $F_4$ acts as the symmetry group of the potential.

The electroweak order parameter $Y_S$ is then read as a particular
component or invariant of the vacuum expectation value $\langle H\rangle$.
Minimising $V_J$ with respect to this degree of freedom leads to a
condition of the form
\[
  \frac{\partial V_J}{\partial Y_S} = 0
  \quad\Longrightarrow\quad
  Y_S^2 = -\frac{\mu^2}{2(\lambda+\kappa c)}.
\]

\section*{Electroweak scale as equilibrium quantity}

The formula
\[
  Y_S^2 = -\frac{\mu^2}{2(\lambda+\kappa c)}
\]
should be read structurally, not numerically. It expresses three key
points:

\begin{enumerate}
  \item The electroweak scale is not inserted by hand; it is a
        \emph{minimum} of an $F_4$-symmetric potential.
  \item The same invariants that control the structure of the operator
        toolbox (rotors, compressors, radius operator) also enter the
        coefficients $(\mu^2,\lambda,\kappa c)$.
  \item Small changes in these coefficients lead to controlled shifts in
        $Y_S$, not to arbitrary values: the scale is a stable equilibrium
        of the internal geometry.
\end{enumerate}

In other words, the 246 GeV scale is reinterpreted as an equilibrium
quantity of the exceptional internal space, analogous to how a crystal
lattice spacing is an equilibrium of an atomic potential, not an external
parameter.

\section*{Connecting AQFT and the potential}

The AQFT and potential pictures are not separate stories:

\begin{itemize}
  \item The potential $V_J(H)$ determines the vacuum configuration
        $\langle H\rangle$, hence the compressor spectra and the operator
        content of the local algebras $\mathcal{A}(\mathcal{O})$.
  \item Conversely, the dynamics of fields in AQFT—encoded in the local
        algebras and their time evolution—are constrained by the same
        internal symmetries ($G_2$, $F_4$) that shaped the potential.
\end{itemize}

This dual view is typical of modern mathematical physics:

\begin{itemize}
  \item On the \emph{operator side}, we quantise via local algebras and
        states, avoiding early commitment to particular Fock spaces.
  \item On the \emph{potential side}, we see vacuum structure and symmetry
        breaking as consequences of exceptional invariants.
\end{itemize}

\section*{What remains open and what is gained}

The fourth Advent Sunday is intentionally honest about open questions:

\begin{itemize}
  \item The gravitational sector is not yet fully encoded; the ratio
        $\kappa = m_p/m_P$ still enters as an external constant, to be
        addressed by a future spectral-gravity analysis.
  \item Renormalisation and nonperturbative effects in an octonionic AQFT
        remain technically challenging.
\end{itemize}

But we gain two important things:

\begin{enumerate}
  \item A quantisation framework—AQFT—that is robust enough to house
        nonassociative internal structures.
  \item A structural explanation of the electroweak scale as an equilibrium
        value of an $F_4$-symmetric potential, linked tightly to the rest
        of the operator toolbox.
\end{enumerate}

\small
\begin{thebibliography}{9}

\bibitem{Haag1996}
R.~Haag,
\newblock {\em Local Quantum Physics},
\newblock Springer, 1996.

\bibitem{HaagKastler1964}
R.~Haag and D.~Kastler,
\newblock ``An algebraic approach to quantum field theory,''
\newblock {\em J.\ Math.\ Phys.} \textbf{5}, 848--861 (1964).

\bibitem{Freund1985}
P.~G.~O.~Freund,
\newblock {\em Introduction to Supersymmetry},
\newblock Cambridge University Press, 1985. % used here as reference for F4/SUSY-style potentials

\bibitem{ConnesChamseddine}
A.~Connes and A.~H.~Chamseddine,
\newblock ``The spectral action principle,''
\newblock {\em Commun.\ Math.\ Phys.} \textbf{186}, 731--750 (1997).

\end{thebibliography}
\normalsize

  } % end #6 body
  {Octonionic AQFT and an $F_4$-symmetric potential turn the electroweak scale
   from a fitted number into an equilibrium of the internal exceptional
   geometry.} % #7 Closing

\end{document}