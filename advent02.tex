% advent02.tex
% December 2, 2025 – Triality: three equal 8D representations

\documentclass[a4paper,10pt]{article}

\usepackage[utf8]{inputenc}
\usepackage[T1]{fontenc}
\usepackage[english]{babel}
\usepackage{amsmath,amssymb,amsfonts}
\usepackage{xcolor}
\usepackage{tikz}
\usetikzlibrary{calc}
\usepackage[framemethod=TikZ]{mdframed}
\usepackage{titlesec}
\usepackage{lettrine}
\usepackage{eso-pic}
\usepackage{geometry}
\usepackage{multicol}
\usepackage{lmodern}

\geometry{margin=2.0cm}

% advent-layout.tex (corrected for \input usage)

\definecolor{adventred}{HTML}{B3001B}
\definecolor{adventblue}{HTML}{003366}
\definecolor{adventgreen}{HTML}{006633}
\definecolor{adventgold}{HTML}{B59410}

\pagestyle{empty}

\titleformat{\section}
  {\normalfont\large\bfseries\color{adventblue}}{\thesection}{0em}{}
  
\titleformat{\subsection}
  {\normalfont\normalsize\bfseries\color{adventblue}}{\thesubsection}{0em}{}

%----
% AdventFrameTop
%----

\newenvironment{AdventFrameTop}
{%
  \begin{mdframed}[
    linecolor=adventgreen!0,
    linewidth=0pt,
    roundcorner=0pt,
    innertopmargin=10pt,
    innerbottommargin=10pt,
    innerleftmargin=10pt,
    innerrightmargin=10pt,
    backgroundcolor=adventgreen!2
  ]%
}
{%
  \end{mdframed}
}

\newcommand{\BeginAdventPage}{}
\newcommand{\EndAdventPage}{}

%----
% AdventTitleBlock
%----

\newcommand{\AdventTitleBlock}[4]{%
  \begin{center}
    {\Large\textcolor{adventred}{\textbf{#1}}}\par\vspace{4pt}%
    \ifx&#2&\else
      {\large\textbf{#2}}\par\vspace{2pt}%
    \fi
    {\Large\textcolor{adventblue}{\textbf{#3}}}\par
    \ifx&#4&\else
      \vspace{2pt}%
      {\normalsize\textbf{#4}}\par%
    \fi
  \end{center}%
}

%----
% AdventKeyInsight
%----

\newcommand{\AdventKeyInsight}[1]{%
  \vspace{0.5em}%
  \noindent\colorbox{adventred!8}{%
    \parbox{\dimexpr\linewidth-2\fboxsep}{%
    \textbf{\textcolor{adventred}{Key Insight.}}~#1%
    }%
  }%
  \vspace{0.5em}%
}

%----
% AdventStarRule
%----

\newcommand{\AdventStarRule}{%
  \vspace{0.3em}%
  \begin{center}
    {\color{adventgold}%
    \rule[0.5ex]{0.25\linewidth}{0.4pt}\;
    $\ast\;\ast\;\ast$\;
    \rule[0.5ex]{0.25\linewidth}{0.4pt}%
    }%
  \end{center}
  \vspace{0.3em}%
}

%----
% AdventClosing
%----

\newcommand{\AdventClosing}[1]{%
  \vspace{0.4em}%
  \begin{center}
    \textcolor{adventgreen}{\emph{#1}}%
  \end{center}
}

%----
% AdventAuthor
%----

\newcommand{\AdventAuthor}{%
  \AddToShipoutPictureFG{%
    \begin{tikzpicture}[remember picture,overlay]
    \node[anchor=south, yshift=2mm] at (current page.south) {%
    \footnotesize Andreas Müller, Kempten University of Applied Sciences, %
    \texttt{andreas.mueller@hs-kempten.de}%
    };
    \end{tikzpicture}%
  }%
}

%----
% AdventInitial
%----

\newcommand{\AdventInitial}[2]{%
  \lettrine[lines=2,lhang=0.1,loversize=0.15]%
    {\textcolor{adventred}{#1}}%
    {#2}%
}

%----
% AdventPageBackground
%----

\newcommand{\AdventPageBackground}{%
  \AddToShipoutPictureBG{%
    \begin{tikzpicture}[remember picture,overlay]
    \draw[adventgreen!80!black, line width=3pt, rounded corners=12pt]
    ($(current page.north west)+(0.8cm,-0.8cm)$)
    rectangle
    ($(current page.south east)+(-0.8cm,0.8cm)$);
    \fill[adventgold]
    ($(current page.north west)+(1.0cm,-1.0cm)$) circle (1.2pt)
    ($(current page.north east)+(-1.0cm,-1.0cm)$) circle (1.2pt)
    ($(current page.south west)+(1.0cm,1.0cm)$) circle (1.2pt)
    ($(current page.south east)+(-1.0cm,1.0cm)$) circle (1.2pt);
    \end{tikzpicture}
  }%
}

%---- AdventSheet macro (1-page, single column) ----
% #1: Date + occasion
% #2: unused
% #3: Main title
% #4: Subtitle
% #5: Key Insight
% #6: Main body content
% #7: Closing statement
\newcommand{\AdventSheet}[7]{%
  \BeginAdventPage
  \vspace*{1cm}
  \begin{AdventFrameTop}
    \AdventTitleBlock{#1}{#2}{#3}{#4}
    \AdventKeyInsight{#5}
  \end{AdventFrameTop}
  \AdventStarRule
  #6%
  \EndAdventPage
  \AdventClosing{#7}%
}

%---- AdventSheetTwoCol macro (two-column hero sheet) ----
% #1: Date + occasion
% #2: unused
% #3: Main title
% #4: Subtitle
% #5: Key Insight
% #6: Main body content (wrapped in multicols{2})
% #7: Closing statement
\newcommand{\AdventSheetTwoCol}[7]{%
  \BeginAdventPage
  \begin{AdventFrameTop}
    \AdventTitleBlock{#1}{#2}{#3}{#4}
    \AdventKeyInsight{#5}
  \end{AdventFrameTop}
  \AdventStarRule
  \begin{multicols}{2}
    #6%
  \end{multicols}
  \EndAdventPage
  \AdventClosing{#7}%
}

\begin{document}

\AdventPageBackground
\AdventAuthor

\AdventSheetTwoCol
  {December 2, 2025} % #1 Date
  {}                 % #2 unused
  {Triality: three equal 8D representations} % #3 Main title
  {When space, matter and antimatter share the same dimension} % #4 Subtitle
  {Spin(8) has a unique feature: \emph{triality}. It admits three
   equivalent 8-dimensional representations—one vector representation
   $V_8$ and two chiral spinor representations $S_8^+$ and $S_8^-$. This
   is not a decorative curiosity: it provides a geometric template for
   three kinds of ``8-dimensional stuff'' that can later be read as
   space, matter and antimatter. Today we meet triality as a symmetry
   between representations; later days will turn it into the organising
   principle behind three fermion generations.} % #5 Key Insight
  { % #6 body (two columns)

\section*{Three eights instead of one}

\AdventInitial{I}{n} most Lie groups, vector and spinor representations
look different and behave differently. Spin(8) is special: it has

\[
  V_8,\qquad S_8^+,\qquad S_8^-,
\]

three irreducible 8-dimensional representations that are related by a
nontrivial outer automorphism group of order 6. This automorphism group
acts by permuting $(V_8, S_8^+, S_8^-)$; its $\mathbb{Z}_3$-part is called
\emph{triality}:

\[
  S:\quad V_8 \;\longrightarrow\; S_8^+ 
         \;\longrightarrow\; S_8^- 
         \;\longrightarrow\; V_8.
\]

Geometrically, the three representations are on equal footing: no one is
more fundamental than the others.

\section*{Why triality matters for an octonionic model}

Octonions $\mathbb{O}$ naturally support an 8-dimensional real
representation. In the model described by this calendar, one uses:

\begin{itemize}
  \item an 8D ``vector-like'' role associated with internal space
        directions,
  \item two 8D ``spinor-like'' roles associated with chiral matter and
        antimatter sectors.
\end{itemize}

Triality then becomes the statement that there is an underlying symmetry
relating these three roles. It is the reason why it is natural to package
internal degrees of freedom in blocks of size 8 and why it is not absurd
to think of space-like and matter-like sectors as different faces of the
same algebraic coin.

\section*{From three 8D reps to three generations}

The XLS plan for the calendar reserves a later day (12 December) for the
statement:

\begin{quote}
  ``Three fermion generations mirror the three triality
  representations.''
\end{quote}

Today we prepare that statement conceptually:

\begin{itemize}
  \item If there is a symmetry that permutes $(V_8, S_8^+, S_8^-)$,
        it is natural to try to attach one 8D ``copy'' of the internal
        structure to each of them.
  \item In particle language, this suggests three families of fermions
        with identical gauge quantum numbers but different ``triality
        label''.
  \item The existence of \emph{exactly} three such representations
        motivates the existence of \emph{exactly} three generations, not
        one or four.
\end{itemize}

The details (how charges and masses are assigned in each block) are left
to the later flavor and generation days. The conceptual point is simple:

\begin{quote}
  Three fermion generations are not an arbitrary count; they reflect a
  built-in threefold symmetry of Spin(8).
\end{quote}

\section*{Link to octonions and G$_2$}

Spin(8) and octonions are tightly linked:

\begin{itemize}
  \item The group $G_2$ is the automorphism group of the octonions
        $\mathbb{O}$.
  \item Spin(8) acts on $\mathbb{O}$ in ways compatible with this
        $G_2$-structure.
  \item Triality is a statement about how vector and spinor actions can be
        interchanged without breaking the internal octonionic structure.
\end{itemize}

In the broader project, this becomes one ingredient in the
\emph{symmetry atlas}:

\begin{itemize}
  \item $G_2$ as the minimal exceptional symmetry (tomorrow),
  \item $F_4$ as the automorphism group of the Albert algebra $H_3(\mathbb{O})$,
  \item triality as the bridge between space-like and spinor-like sectors.
\end{itemize}

\section*{Conceptual gain from triality}

What is gained by taking triality seriously?

\begin{enumerate}
  \item \textbf{Symmetry-based multiplicity:}
    the multiplicity ``three'' is not an afterthought but a symmetry
    consequence.
  \item \textbf{Unified treatment of sectors:}
    space-like, matter-like and antimatter-like sectors share the same
    dimension and are related by an actual group action.
  \item \textbf{Constraints for model building:}
    any attempt to modify the number of generations must explain how
    triality is broken or extended.
\end{enumerate}

Instead of asking ``Why three generations?'' as a bare phenomenological
question, the model asks a more geometric one:

\begin{quote}
  How does Spin(8) triality appear inside the octonionic/Albert
  structure, and how does it force a threefold replication of internal
  degrees of freedom?
\end{quote}

\small
\begin{thebibliography}{9}

\bibitem{Cartan1925}
E.~Cartan,
\newblock ``Le principe de dualit\'e et la th\'eorie des groupes simples et
semi-simples,''
\newblock {\em Bull.\ Sci.\ Math.} \textbf{49}, 361--374 (1925).

\bibitem{Baez2002}
J.~C.~Baez,
\newblock ``The octonions,''
\newblock {\em Bull.\ Amer.\ Math.\ Soc.} \textbf{39}, 145--205 (2002).

\bibitem{Internal}
[Internal notes on triality and its role in generation structure:
{\tt chap01\_neu.tex; appA\_neu.tex; appN\_neu.tex}.]

\end{thebibliography}
\normalsize

  } % end #6 body
  {Triality of Spin(8) provides three equal 8D representations. In an
   octonionic model, this becomes the natural algebraic source of the
   threefold replication of fermion content that we observe as
   ``three generations''.} % #7 Closing

\end{document}