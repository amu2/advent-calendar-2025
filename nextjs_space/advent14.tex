% advent14.tex
% Hero day: December 14, 2025 – CKM and PMNS

\documentclass[a4paper,10pt]{article}

\usepackage[utf8]{inputenc}
\usepackage[T1]{fontenc}
\usepackage[english]{babel}
\usepackage{amsmath,amssymb,amsfonts}
\usepackage{xcolor}
\usepackage{tikz}
\usetikzlibrary{calc}
\usepackage[framemethod=TikZ]{mdframed}
\usepackage{titlesec}
\usepackage{lettrine}
\usepackage{eso-pic}
\usepackage{geometry}
\usepackage{multicol}
\usepackage{lmodern}

\geometry{margin=2.0cm}

% advent-layout.tex (corrected for \input usage)

\definecolor{adventred}{HTML}{B3001B}
\definecolor{adventblue}{HTML}{003366}
\definecolor{adventgreen}{HTML}{006633}
\definecolor{adventgold}{HTML}{B59410}

\pagestyle{empty}

\titleformat{\section}
  {\normalfont\large\bfseries\color{adventblue}}{\thesection}{0em}{}
  
\titleformat{\subsection}
  {\normalfont\normalsize\bfseries\color{adventblue}}{\thesubsection}{0em}{}

%----
% AdventFrameTop
%----

\newenvironment{AdventFrameTop}
{%
  \begin{mdframed}[
    linecolor=adventgreen!0,
    linewidth=0pt,
    roundcorner=0pt,
    innertopmargin=10pt,
    innerbottommargin=10pt,
    innerleftmargin=10pt,
    innerrightmargin=10pt,
    backgroundcolor=adventgreen!2
  ]%
}
{%
  \end{mdframed}
}

\newcommand{\BeginAdventPage}{}
\newcommand{\EndAdventPage}{}

%----
% AdventTitleBlock
%----

\newcommand{\AdventTitleBlock}[4]{%
  \begin{center}
    {\Large\textcolor{adventred}{\textbf{#1}}}\par\vspace{4pt}%
    \ifx&#2&\else
      {\large\textbf{#2}}\par\vspace{2pt}%
    \fi
    {\Large\textcolor{adventblue}{\textbf{#3}}}\par
    \ifx&#4&\else
      \vspace{2pt}%
      {\normalsize\textbf{#4}}\par%
    \fi
  \end{center}%
}

%----
% AdventKeyInsight
%----

\newcommand{\AdventKeyInsight}[1]{%
  \vspace{0.5em}%
  \noindent\colorbox{adventred!8}{%
    \parbox{\dimexpr\linewidth-2\fboxsep}{%
    \textbf{\textcolor{adventred}{Key Insight.}}~#1%
    }%
  }%
  \vspace{0.5em}%
}

%----
% AdventStarRule
%----

\newcommand{\AdventStarRule}{%
  \vspace{0.3em}%
  \begin{center}
    {\color{adventgold}%
    \rule[0.5ex]{0.25\linewidth}{0.4pt}\;
    $\ast\;\ast\;\ast$\;
    \rule[0.5ex]{0.25\linewidth}{0.4pt}%
    }%
  \end{center}
  \vspace{0.3em}%
}

%----
% AdventClosing
%----

\newcommand{\AdventClosing}[1]{%
  \vspace{0.4em}%
  \begin{center}
    \textcolor{adventgreen}{\emph{#1}}%
  \end{center}
}

%----
% AdventAuthor
%----

\newcommand{\AdventAuthor}{%
  \AddToShipoutPictureFG{%
    \begin{tikzpicture}[remember picture,overlay]
    \node[anchor=south, yshift=2mm] at (current page.south) {%
    \footnotesize Andreas Müller, Kempten University of Applied Sciences, %
    \texttt{andreas.mueller@hs-kempten.de}%
    };
    \end{tikzpicture}%
  }%
}

%----
% AdventInitial
%----

\newcommand{\AdventInitial}[2]{%
  \lettrine[lines=2,lhang=0.1,loversize=0.15]%
    {\textcolor{adventred}{#1}}%
    {#2}%
}

%----
% AdventPageBackground
%----

\newcommand{\AdventPageBackground}{%
  \AddToShipoutPictureBG{%
    \begin{tikzpicture}[remember picture,overlay]
    \draw[adventgreen!80!black, line width=3pt, rounded corners=12pt]
    ($(current page.north west)+(0.8cm,-0.8cm)$)
    rectangle
    ($(current page.south east)+(-0.8cm,0.8cm)$);
    \fill[adventgold]
    ($(current page.north west)+(1.0cm,-1.0cm)$) circle (1.2pt)
    ($(current page.north east)+(-1.0cm,-1.0cm)$) circle (1.2pt)
    ($(current page.south west)+(1.0cm,1.0cm)$) circle (1.2pt)
    ($(current page.south east)+(-1.0cm,1.0cm)$) circle (1.2pt);
    \end{tikzpicture}
  }%
}

%---- AdventSheet macro (1-page, single column) ----
% #1: Date + occasion
% #2: unused
% #3: Main title
% #4: Subtitle
% #5: Key Insight
% #6: Main body content
% #7: Closing statement
\newcommand{\AdventSheet}[7]{%
  \BeginAdventPage
  \vspace*{1cm}
  \begin{AdventFrameTop}
    \AdventTitleBlock{#1}{#2}{#3}{#4}
    \AdventKeyInsight{#5}
  \end{AdventFrameTop}
  \AdventStarRule
  #6%
  \EndAdventPage
  \AdventClosing{#7}%
}

%---- AdventSheetTwoCol macro (two-column hero sheet) ----
% #1: Date + occasion
% #2: unused
% #3: Main title
% #4: Subtitle
% #5: Key Insight
% #6: Main body content (wrapped in multicols{2})
% #7: Closing statement
\newcommand{\AdventSheetTwoCol}[7]{%
  \BeginAdventPage
  \begin{AdventFrameTop}
    \AdventTitleBlock{#1}{#2}{#3}{#4}
    \AdventKeyInsight{#5}
  \end{AdventFrameTop}
  \AdventStarRule
  \begin{multicols}{2}
    #6%
  \end{multicols}
  \EndAdventPage
  \AdventClosing{#7}%
}

\begin{document}

\AdventPageBackground
\AdventAuthor

\AdventSheetTwoCol
  {December 14, 2025}      % #1 Date + occasion
  {}                                           % #2 unused
  {CKM and PMNS from an Exceptional Atlas}     % #3 Main title
  {Mixing matrices as transition charts on $H_3(\mathbb{O})$} % #4 Subtitle
  {Quark and lepton mixings are usually encoded in two unitary matrices,
   CKM and PMNS, introduced as phenomenological parameters. In the
   octonionic model, these matrices are reinterpreted as transition maps
   between different eigenbases of operators on the Albert algebra
   $H_3(\mathbb{O})$. The pattern of mixings is constrained by how the
   mass map, flavor projectors and compressor operators fit together in
   the exceptional geometry.}                   % #5 Key Insight
  { % #6 main body (two columns)

\section*{Two mysterious unitary matrices}

\AdventInitial{T}{he} Standard Model describes quark and lepton mixings by
two unitary matrices:
\[
  V_{\text{CKM}} \quad\text{for quarks},\qquad
  U_{\text{PMNS}} \quad\text{for leptons}.
\]
Empirically, $V_{\text{CKM}}$ is close to the identity with small off-diagonal
elements, while $U_{\text{PMNS}}$ shows large mixings between neutrino
flavors. In the usual formulation, both matrices are introduced by hand
when diagonalising Yukawa matrices. They are indispensable in
phenomenology, but their structure appears accidental.

The octonionic model offers a different viewpoint. It treats mixing angles
and phases not as arbitrary complex numbers, but as geometric data attached
to how different preferred bases in the exceptional internal space
intersect.

\section*{The internal atlas picture}

The internal degrees of freedom live in the Albert algebra $H_3(\mathbb{O})$,
with automorphism group $F_4$. On this space, several distinguished
operators are defined:

\begin{itemize}
  \item The \emph{mass map} $\Pi(H)$, built from a vacuum configuration
        $\langle H\rangle$, whose eigenvectors encode mass eigenstates.
  \item \emph{Flavor projectors} which single out quark-like and
        lepton-like subspaces.
  \item \emph{Compressor} and \emph{rotor} operators that encode hierarchies
        and couplings.
\end{itemize}

Different physical questions prefer different eigenbases:

\begin{itemize}
  \item Gauge interactions ``see'' eigenbases aligned with flavor projectors.
  \item Mass measurements ``see'' eigenbases of $\Pi(H)$.
\end{itemize}

The resulting misalignment between these bases is what phenomenologists
call ``mixing''. In the geometric language, it is nothing but a transition
map between charts on the internal space.

\section*{CKM as a transition map in the quark sector}

Focus first on the quark sector. Consider two orthonormal bases of the
relevant internal subspace:

\begin{itemize}
  \item $\{\,|d_i\rangle\,\}$, $i=1,2,3$, adapted to flavor (interaction)
        eigenstates: these diagonalise the couplings to $W^\pm$ and gluons.
  \item $\{\,|d_i'\rangle\,\}$, $i=1,2,3$, adapted to mass eigenstates:
        these diagonalise the restriction of $\Pi(H)$ to the down-type quark
        sector.
\end{itemize}

By definition, there exists a unitary matrix $V_{\text{CKM}}$ such that
\[
  |d_i\rangle \;=\; \sum_j (V_{\text{CKM}})_{ij}\,|d_j'\rangle.
\]
In the octonionic setting, both bases arise from eigenproblems for
operators that live in the same exceptional algebra and are constrained by
its structure. As a result:

\begin{itemize}
  \item The allowed pattern of overlaps $(V_{\text{CKM}})_{ij}$ is not
        arbitrary; some entries are naturally suppressed or enhanced.
  \item Small quark mixing angles become a consequence of how the quark
        flavor projectors and the mass map almost commute within the
        relevant subspace of $H_3(\mathbb{O})$.
\end{itemize}

The hierarchical structure of $V_{\text{CKM}}$ thus reflects an almost
compatible choice of charts in the internal atlas.

\section*{PMNS and the special role of neutrinos}

For leptons, the situation is different. The PMNS matrix connects charged
lepton mass eigenstates to neutrino flavor eigenstates. In many models,
neutrinos are Majorana particles with a mass matrix that has a different
origin from the charged lepton masses.

In the octonionic model, this difference is encoded in the way neutrino
directions sit inside $H_3(\mathbb{O})$:

\begin{itemize}
  \item Neutrino-like internal directions may be associated with subspaces
        that are more strongly affected by nonassociative chains in the
        octonionic structure.
  \item The corresponding projectors onto neutrino subspaces can fail to
        commute more strongly with the mass map and with certain rotor
        operators.
\end{itemize}

As a result, the overlap matrix $U_{\text{PMNS}}$ between flavor and mass
eigenbases for leptons can naturally have large off-diagonal entries: the
relevant operators pick eigenbases that are more misaligned than in the
quark case.

\section*{Qualitative patterns from exceptional constraints}

The goal of the octonionic approach is not to predict every mixing angle
and phase numerically from first principles in a single stroke. Instead, it
aims to explain why certain \emph{patterns} appear robust:

\begin{enumerate}
  \item \textbf{Quark mixings are small:} quark flavor projectors and the
        quark part of $\Pi(H)$ are nearly simultaneously diagonalisable
        within the exceptional algebra, leading to small mixing angles.
  \item \textbf{Lepton mixings are large:} leptonic projectors, especially
        in the neutrino sector, have a more frustrated relation to the mass
        map and rotor operators, leading to larger natural mixings.
  \item \textbf{Correlations between sectors:} because all these operators
        live in the same $H_3(\mathbb{O})$ and share the same vacuum
        configuration, patterns in the CKM and PMNS matrices are not
        independent: a change in the internal geometry affects both.
\end{enumerate}

This is reminiscent of how coordinate changes on a curved manifold cannot
be chosen independently in overlapping charts.

\section*{Towards numerical explorations}

Once a concrete vacuum configuration $\langle H\rangle$ and explicit forms
for the mass map, projectors and compressors are chosen, the model becomes
numerical:

\begin{itemize}
  \item One can diagonalise the relevant operators in $H_3(\mathbb{O})$ and
        compute the induced mixing matrices.
  \item One can study how these matrices change under deformations of
        $\langle H\rangle$ or of the internal rotor structure.
  \item One can search for attractor configurations that yield mixing
        patterns close to the observed CKM and PMNS matrices.
\end{itemize}

In this sense, the exceptional geometry provides a \emph{parameter space}
for mixings that is much more constrained than a generic set of complex
$3\times3$ unitary matrices.

\small
\begin{thebibliography}{9}

\bibitem{Cabibbo1963}
N.~Cabibbo,
\newblock ``Unitary symmetry and leptonic decays,''
\newblock {\em Phys.\ Rev.\ Lett.} \textbf{10}, 531--533 (1963).

\bibitem{KobayashiMaskawa1973}
M.~Kobayashi and T.~Maskawa,
\newblock ``CP violation in the renormalizable theory of weak interaction,''
\newblock {\em Prog.\ Theor.\ Phys.} \textbf{49}, 652--657 (1973).

\bibitem{MakiNakagawaSakata1962}
Z.~Maki, M.~Nakagawa and S.~Sakata,
\newblock ``Remarks on the unified model of elementary particles,''
\newblock {\em Prog.\ Theor.\ Phys.} \textbf{28}, 870--880 (1962).

\bibitem{GurseyTze1996}
F.~Gürsey and H.~C.~Tze,
\newblock {\em On the Role of Division, Jordan and Related Algebras in Particle
Physics},
\newblock World Scientific, 1996.

\end{thebibliography}
\normalsize

  } % end #6 body
  {CKM and PMNS are read as transition maps between preferred bases in an exceptional internal space, not as arbitrary phenomenological matrices.} % #7 Closing

\end{document}