% advent15.tex
% Hero day: December 15, 2025 – Numerical prototypes (Ersatzbank)

\documentclass[a4paper,10pt]{article}

\usepackage[utf8]{inputenc}
\usepackage[T1]{fontenc}
\usepackage[english]{babel}
\usepackage{amsmath,amssymb,amsfonts}
\usepackage{xcolor}
\usepackage{tikz}
\usetikzlibrary{calc}
\usepackage[framemethod=TikZ]{mdframed}
\usepackage{titlesec}
\usepackage{lettrine}
\usepackage{eso-pic}
\usepackage{geometry}
\usepackage{multicol}
\usepackage{lmodern}

\geometry{margin=2.0cm}

% advent-layout.tex (corrected for \input usage)

\definecolor{adventred}{HTML}{B3001B}
\definecolor{adventblue}{HTML}{003366}
\definecolor{adventgreen}{HTML}{006633}
\definecolor{adventgold}{HTML}{B59410}

\pagestyle{empty}

\titleformat{\section}
  {\normalfont\large\bfseries\color{adventblue}}{\thesection}{0em}{}
  
\titleformat{\subsection}
  {\normalfont\normalsize\bfseries\color{adventblue}}{\thesubsection}{0em}{}

%----
% AdventFrameTop
%----

\newenvironment{AdventFrameTop}
{%
  \begin{mdframed}[
    linecolor=adventgreen!0,
    linewidth=0pt,
    roundcorner=0pt,
    innertopmargin=10pt,
    innerbottommargin=10pt,
    innerleftmargin=10pt,
    innerrightmargin=10pt,
    backgroundcolor=adventgreen!2
  ]%
}
{%
  \end{mdframed}
}

\newcommand{\BeginAdventPage}{}
\newcommand{\EndAdventPage}{}

%----
% AdventTitleBlock
%----

\newcommand{\AdventTitleBlock}[4]{%
  \begin{center}
    {\Large\textcolor{adventred}{\textbf{#1}}}\par\vspace{4pt}%
    \ifx&#2&\else
      {\large\textbf{#2}}\par\vspace{2pt}%
    \fi
    {\Large\textcolor{adventblue}{\textbf{#3}}}\par
    \ifx&#4&\else
      \vspace{2pt}%
      {\normalsize\textbf{#4}}\par%
    \fi
  \end{center}%
}

%----
% AdventKeyInsight
%----

\newcommand{\AdventKeyInsight}[1]{%
  \vspace{0.5em}%
  \noindent\colorbox{adventred!8}{%
    \parbox{\dimexpr\linewidth-2\fboxsep}{%
    \textbf{\textcolor{adventred}{Key Insight.}}~#1%
    }%
  }%
  \vspace{0.5em}%
}

%----
% AdventStarRule
%----

\newcommand{\AdventStarRule}{%
  \vspace{0.3em}%
  \begin{center}
    {\color{adventgold}%
    \rule[0.5ex]{0.25\linewidth}{0.4pt}\;
    $\ast\;\ast\;\ast$\;
    \rule[0.5ex]{0.25\linewidth}{0.4pt}%
    }%
  \end{center}
  \vspace{0.3em}%
}

%----
% AdventClosing
%----

\newcommand{\AdventClosing}[1]{%
  \vspace{0.4em}%
  \begin{center}
    \textcolor{adventgreen}{\emph{#1}}%
  \end{center}
}

%----
% AdventAuthor
%----

\newcommand{\AdventAuthor}{%
  \AddToShipoutPictureFG{%
    \begin{tikzpicture}[remember picture,overlay]
    \node[anchor=south, yshift=2mm] at (current page.south) {%
    \footnotesize Andreas Müller, Kempten University of Applied Sciences, %
    \texttt{andreas.mueller@hs-kempten.de}%
    };
    \end{tikzpicture}%
  }%
}

%----
% AdventInitial
%----

\newcommand{\AdventInitial}[2]{%
  \lettrine[lines=2,lhang=0.1,loversize=0.15]%
    {\textcolor{adventred}{#1}}%
    {#2}%
}

%----
% AdventPageBackground
%----

\newcommand{\AdventPageBackground}{%
  \AddToShipoutPictureBG{%
    \begin{tikzpicture}[remember picture,overlay]
    \draw[adventgreen!80!black, line width=3pt, rounded corners=12pt]
    ($(current page.north west)+(0.8cm,-0.8cm)$)
    rectangle
    ($(current page.south east)+(-0.8cm,0.8cm)$);
    \fill[adventgold]
    ($(current page.north west)+(1.0cm,-1.0cm)$) circle (1.2pt)
    ($(current page.north east)+(-1.0cm,-1.0cm)$) circle (1.2pt)
    ($(current page.south west)+(1.0cm,1.0cm)$) circle (1.2pt)
    ($(current page.south east)+(-1.0cm,1.0cm)$) circle (1.2pt);
    \end{tikzpicture}
  }%
}

%---- AdventSheet macro (1-page, single column) ----
% #1: Date + occasion
% #2: unused
% #3: Main title
% #4: Subtitle
% #5: Key Insight
% #6: Main body content
% #7: Closing statement
\newcommand{\AdventSheet}[7]{%
  \BeginAdventPage
  \vspace*{1cm}
  \begin{AdventFrameTop}
    \AdventTitleBlock{#1}{#2}{#3}{#4}
    \AdventKeyInsight{#5}
  \end{AdventFrameTop}
  \AdventStarRule
  #6%
  \EndAdventPage
  \AdventClosing{#7}%
}

%---- AdventSheetTwoCol macro (two-column hero sheet) ----
% #1: Date + occasion
% #2: unused
% #3: Main title
% #4: Subtitle
% #5: Key Insight
% #6: Main body content (wrapped in multicols{2})
% #7: Closing statement
\newcommand{\AdventSheetTwoCol}[7]{%
  \BeginAdventPage
  \begin{AdventFrameTop}
    \AdventTitleBlock{#1}{#2}{#3}{#4}
    \AdventKeyInsight{#5}
  \end{AdventFrameTop}
  \AdventStarRule
  \begin{multicols}{2}
    #6%
  \end{multicols}
  \EndAdventPage
  \AdventClosing{#7}%
}

\begin{document}

\AdventPageBackground
\AdventAuthor

\AdventSheetTwoCol
  {December 15, 2025 \large (Numerical prototypes)} % #1 Date + occasion
  {}                                                % #2 unused
  {First Numerical Prototypes from $H_3(\mathbb{O})$} % #3 Main title
  {Simple vacua, concrete spectra}                  % #4 Subtitle
  {The octonionic model is not only an abstract Playground of exceptional
   algebras. Already for very simple vacuum configurations $\langle H\rangle$
   in the Albert algebra $H_3(\mathbb{O})$ one can compute explicit
   eigenvalue multiplets of the mass map $\Pi(\langle H\rangle)$. These
   multiplets form ``numerical prototypes'' for fermion mass spectra:
   banded, hierarchical, and organised in patterns that resemble quark and
   lepton families.}                                % #5 Key Insight
  { % #6 main body (two columns)

\section*{From symbols to numbers}

\AdventInitial{S}{o} far, the exceptional structures have been presented in
a largely symbolic way: octonions, the Albert algebra $H_3(\mathbb{O})$, the
mass map $\Pi(H)$ and its relation to vacuum configurations $\langle H\rangle$.
At some point, however, the model must leave the realm of pure algebra and
produce numbers that can be compared, at least qualitatively, to particle
physics.

The good news is that this can already be done with remarkably simple
choices of $\langle H\rangle$. By freezing most degrees of freedom and
retaining only a few dominant internal directions, one obtains toy vacua
that are simple enough to diagonalise exactly, but rich enough to display
hierarchical spectra.

\section*{Diagonal prototype vacua}

The cleanest starting point is a diagonal configuration in $H_3(\mathbb{O})$:
\[
  \langle H\rangle
  \;=\;
  \begin{pmatrix}
    \lambda_1 & 0         & 0 \\
    0         & \lambda_2 & 0 \\
    0         & 0         & \lambda_3
  \end{pmatrix},
  \qquad
  \lambda_i \in \mathbb{R}.
\]
Such a vacuum preserves a maximal amount of internal symmetry: off-diagonal
octonionic entries are set to zero, and only three real parameters remain.
Despite this simplicity, the induced mass map
\[
  \Pi(\langle H\rangle) : \mathcal{H}_{\text{int}} \longrightarrow \mathcal{H}_{\text{int}}
\]
already exhibits a nontrivial eigenvalue structure.

In the eigenvalue problem
\[
  \Pi(\langle H\rangle)\,\Psi = m\,\Psi,
\]
the internal Hilbert space $\mathcal{H}_{\text{int}}$ decomposes into
sectors that correspond to quark-like and lepton-like states, coloured and
colourless states, left- and right-handed components, and so on. Each
sector contributes a set of eigenvalues $m$.

For suitable choices of $(\lambda_1,\lambda_2,\lambda_3)$ one observes:

\begin{itemize}
  \item Eigenvalues cluster into a few \emph{bands}, separated by gaps.
  \item Within each band, degeneracies and small splittings appear, induced
        by the residual internal symmetry.
  \item The overall pattern can be read as ``light, medium, heavy'' families,
        evocative of the observed hierarchy between generations.
\end{itemize}

\section*{Off-diagonal perturbations and splitting patterns}

To go beyond the simplest diagonal picture, one can add controlled
off-diagonal octonionic entries to $\langle H\rangle$:
\[
  \langle H\rangle
  \;=\;
  \begin{pmatrix}
    \lambda_1 & x         & 0 \\
    \bar{x}   & \lambda_2 & y \\
    0         & \bar{y}   & \lambda_3
  \end{pmatrix},
  \qquad
  x,y \in \mathbb{O}.
\]
Even when $x$ and $y$ are chosen along a single imaginary octonion unit,
the eigenvalue structure of $\Pi(\langle H\rangle)$ changes qualitatively:

\begin{itemize}
  \item Degenerate bands split into sub-bands.
  \item Some states are pushed up or down in mass, mimicking the effect of
        mixing between different internal directions.
  \item Eigenvectors acquire nontrivial support across different
        triality-related components, foreshadowing the mixing phenomena
        associated with CKM and PMNS matrices.
\end{itemize}

These controlled deformations are the ``numerical workhorses'' of the
model: they allow us to scan how the spectrum responds when the vacuum
is moved inside $H_3(\mathbb{O})$.

\section*{Prototype spectra and their interpretation}

The resulting eigenvalue sets are not yet to be interpreted as precise
predictions for Standard Model fermion masses. Instead, they serve as
\emph{prototypes} that illustrate what the exceptional machinery tends to
produce generically:

\begin{enumerate}
  \item \textbf{Banded hierarchies:} eigenvalues group into bands with
        ratios that are naturally large or small, depending on how many
        internal directions contribute coherently.
  \item \textbf{Sector-dependent patterns:} quark-like and lepton-like
        sectors display different splitting patterns, reflecting their
        different embedding in the internal algebra.
  \item \textbf{Robustness under deformation:} certain qualitative
        hierarchies persist under a wide range of small changes in
        $(\lambda_1,\lambda_2,\lambda_3,x,y)$, suggesting that they are
        structural rather than fine-tuned.
\end{enumerate}

From this perspective, the complicated observed mass spectrum is seen as
one point in a structured space of possible spectra generated by the
octonionic geometry.

\section*{The role of the Ersatzbank}

Within the larger Advent project, these numerical prototypes form an
``Ersatzbank'' of ready-made examples:

\begin{itemize}
  \item They can be used to replace more philosophical or qualitative
        discussion days by concrete numerical sheets.
  \item They demonstrate to the reader that the abstract constructions
        genuinely lead to computable spectra with clear patterns.
  \item They provide anchor points for later, more refined fits where
        the parameters of $\langle H\rangle$ and the definition of
        $\Pi(H)$ are tuned to approximate real-world data.
\end{itemize}

In teaching or outreach contexts, such an Ersatzbank is particularly
valuable: it allows one to show explicit tables of ``eigenvalues from an
exceptional universe'' without requiring the audience to follow every
technical step.

\section*{An invitation to explore}

Technically, numerical exploration of $\Pi(\langle H\rangle)$ is
straightforward: once a concrete basis for $H_3(\mathbb{O})$ and the
relevant operators is chosen, the problem reduces to diagonalising large
but structured matrices. Modern linear algebra libraries can handle this
comfortably.

Conceptually, the challenge is to interpret the resulting patterns in a
way that respects both the algebraic symmetries and the phenomenological
constraints. The prototypes presented here are first steps in that
direction: they show that the exceptional machinery is capable of
producing spectra that look qualitatively like what one expects from
quark and lepton masses, without any artificial ``by hand'' hierarchies
inserted into the Lagrangian.

\small
\begin{thebibliography}{9}

\bibitem{GurseyTze1996}
F.~Gürsey and H.~C.~Tze,
\newblock {\em On the Role of Division, Jordan and Related Algebras in Particle
Physics},
\newblock World Scientific, 1996.

\bibitem{Baez2002}
J.~C.~Baez,
\newblock ``The octonions,''
\newblock {\em Bull.\ Amer.\ Math.\ Soc.} \textbf{39}, 145--205 (2002).

\bibitem{ConnesMarcolli2008}
A.~Connes and M.~Marcolli,
\newblock {\em Noncommutative Geometry, Quantum Fields and Motives},
\newblock American Mathematical Society, 2008.

\end{thebibliography}
\normalsize

  } % end #6 body
  {Even very simple vacua in $H_3(\mathbb{O})$ already yield banded, hierarchical spectra: numerical prototypes of a world built from exceptional geometry.} % #7 Closing

\end{document}