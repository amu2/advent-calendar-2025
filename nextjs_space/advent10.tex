% advent10.tex
% December 10, 2025 – Weinberg angle and couplings from rotor norms

\documentclass[a4paper,10pt]{article}

\usepackage[utf8]{inputenc}
\usepackage[T1]{fontenc}
\usepackage[english]{babel}
\usepackage{amsmath,amssymb,amsfonts}
\usepackage{xcolor}
\usepackage{tikz}
\usetikzlibrary{calc}
\usepackage[framemethod=TikZ]{mdframed}
\usepackage{titlesec}
\usepackage{lettrine}
\usepackage{eso-pic}
\usepackage{geometry}
\usepackage{multicol}
\usepackage{lmodern}

\geometry{margin=2.0cm}

% advent-layout.tex (corrected for \input usage)

\definecolor{adventred}{HTML}{B3001B}
\definecolor{adventblue}{HTML}{003366}
\definecolor{adventgreen}{HTML}{006633}
\definecolor{adventgold}{HTML}{B59410}

\pagestyle{empty}

\titleformat{\section}
  {\normalfont\large\bfseries\color{adventblue}}{\thesection}{0em}{}
  
\titleformat{\subsection}
  {\normalfont\normalsize\bfseries\color{adventblue}}{\thesubsection}{0em}{}

%----
% AdventFrameTop
%----

\newenvironment{AdventFrameTop}
{%
  \begin{mdframed}[
    linecolor=adventgreen!0,
    linewidth=0pt,
    roundcorner=0pt,
    innertopmargin=10pt,
    innerbottommargin=10pt,
    innerleftmargin=10pt,
    innerrightmargin=10pt,
    backgroundcolor=adventgreen!2
  ]%
}
{%
  \end{mdframed}
}

\newcommand{\BeginAdventPage}{}
\newcommand{\EndAdventPage}{}

%----
% AdventTitleBlock
%----

\newcommand{\AdventTitleBlock}[4]{%
  \begin{center}
    {\Large\textcolor{adventred}{\textbf{#1}}}\par\vspace{4pt}%
    \ifx&#2&\else
      {\large\textbf{#2}}\par\vspace{2pt}%
    \fi
    {\Large\textcolor{adventblue}{\textbf{#3}}}\par
    \ifx&#4&\else
      \vspace{2pt}%
      {\normalsize\textbf{#4}}\par%
    \fi
  \end{center}%
}

%----
% AdventKeyInsight
%----

\newcommand{\AdventKeyInsight}[1]{%
  \vspace{0.5em}%
  \noindent\colorbox{adventred!8}{%
    \parbox{\dimexpr\linewidth-2\fboxsep}{%
    \textbf{\textcolor{adventred}{Key Insight.}}~#1%
    }%
  }%
  \vspace{0.5em}%
}

%----
% AdventStarRule
%----

\newcommand{\AdventStarRule}{%
  \vspace{0.3em}%
  \begin{center}
    {\color{adventgold}%
    \rule[0.5ex]{0.25\linewidth}{0.4pt}\;
    $\ast\;\ast\;\ast$\;
    \rule[0.5ex]{0.25\linewidth}{0.4pt}%
    }%
  \end{center}
  \vspace{0.3em}%
}

%----
% AdventClosing
%----

\newcommand{\AdventClosing}[1]{%
  \vspace{0.4em}%
  \begin{center}
    \textcolor{adventgreen}{\emph{#1}}%
  \end{center}
}

%----
% AdventAuthor
%----

\newcommand{\AdventAuthor}{%
  \AddToShipoutPictureFG{%
    \begin{tikzpicture}[remember picture,overlay]
    \node[anchor=south, yshift=2mm] at (current page.south) {%
    \footnotesize Andreas Müller, Kempten University of Applied Sciences, %
    \texttt{andreas.mueller@hs-kempten.de}%
    };
    \end{tikzpicture}%
  }%
}

%----
% AdventInitial
%----

\newcommand{\AdventInitial}[2]{%
  \lettrine[lines=2,lhang=0.1,loversize=0.15]%
    {\textcolor{adventred}{#1}}%
    {#2}%
}

%----
% AdventPageBackground
%----

\newcommand{\AdventPageBackground}{%
  \AddToShipoutPictureBG{%
    \begin{tikzpicture}[remember picture,overlay]
    \draw[adventgreen!80!black, line width=3pt, rounded corners=12pt]
    ($(current page.north west)+(0.8cm,-0.8cm)$)
    rectangle
    ($(current page.south east)+(-0.8cm,0.8cm)$);
    \fill[adventgold]
    ($(current page.north west)+(1.0cm,-1.0cm)$) circle (1.2pt)
    ($(current page.north east)+(-1.0cm,-1.0cm)$) circle (1.2pt)
    ($(current page.south west)+(1.0cm,1.0cm)$) circle (1.2pt)
    ($(current page.south east)+(-1.0cm,1.0cm)$) circle (1.2pt);
    \end{tikzpicture}
  }%
}

%---- AdventSheet macro (1-page, single column) ----
% #1: Date + occasion
% #2: unused
% #3: Main title
% #4: Subtitle
% #5: Key Insight
% #6: Main body content
% #7: Closing statement
\newcommand{\AdventSheet}[7]{%
  \BeginAdventPage
  \vspace*{1cm}
  \begin{AdventFrameTop}
    \AdventTitleBlock{#1}{#2}{#3}{#4}
    \AdventKeyInsight{#5}
  \end{AdventFrameTop}
  \AdventStarRule
  #6%
  \EndAdventPage
  \AdventClosing{#7}%
}

%---- AdventSheetTwoCol macro (two-column hero sheet) ----
% #1: Date + occasion
% #2: unused
% #3: Main title
% #4: Subtitle
% #5: Key Insight
% #6: Main body content (wrapped in multicols{2})
% #7: Closing statement
\newcommand{\AdventSheetTwoCol}[7]{%
  \BeginAdventPage
  \begin{AdventFrameTop}
    \AdventTitleBlock{#1}{#2}{#3}{#4}
    \AdventKeyInsight{#5}
  \end{AdventFrameTop}
  \AdventStarRule
  \begin{multicols}{2}
    #6%
  \end{multicols}
  \EndAdventPage
  \AdventClosing{#7}%
}

\begin{document}

\AdventPageBackground
\AdventAuthor

\AdventSheetTwoCol
  {December 10, 2025} % #1 Date
  {}                  % #2 unused
  {Weinberg angle and couplings from rotor norms} % #3 Main title
  {When $\alpha$, $\alpha_s$ and $\sin^2\theta_W$ are geometric overlaps} % #4 Subtitle
  {In the Standard Model, the fine-structure constant $\alpha$, the strong
   coupling $\alpha_s$ and the Weinberg angle $\theta_W$ are independent
   running parameters. In the octonionic rotor picture, they have a common
   geometric origin: they are read off from norms and mutual angles of
   commutators in the internal operator algebra. The electroweak mixing
   angle $\theta_W$ becomes a literal angle between two rotor directions
   that define hypercharge and weak isospin inside the exceptional
   stage.} % #5 Key Insight
  { % #6 body (two columns)

\section*{Rotor generators for internal forces}

\AdventInitial{T}{he} internal symmetries of one generation are encoded by
rotor-like operators $G_a$ acting on the octonionic internal space. In
this setting,

\begin{itemize}
  \item color $SU(3)_C$ corresponds to one set of rotor directions,
  \item weak $SU(2)_L$ to another set,
  \item hypercharge $U(1)_Y$ to a particular combination of internal
        rotations.
\end{itemize}

The basic data are the commutators
\[
  [G_a,G_b],
\]
whose norms and mutual angles in operator space reflect the structure
constants and coupling strengths of the effective gauge theory.

\section*{Couplings as norms in operator space}

Schematically, one defines an inner product on the space of rotor
operators, for example via a trace on the internal Hilbert space:
\[
  \langle A,B\rangle := \mathrm{Tr}(A^\dagger B)
  \quad\text{(up to normalisation).}
\]

With this structure, the effective gauge couplings can be associated with
the sizes of commutators:

\begin{itemize}
  \item the electromagnetic coupling $\alpha$ with a suitable abelian
        combination of rotors,
  \item the strong coupling $\alpha_s$ with the norm of $SU(3)$ commutator
        directions,
  \item the weak coupling $g$ with the norm of $SU(2)$ rotors.
\end{itemize}

Symbolically,
\[
  \alpha \sim \|\,[Q,Q']\,\|^2,\qquad
  \alpha_s \sim \|[T_a,T_b]\|^2,
\]
where $Q$ is an electromagnetic charge operator and $T_a$ are color
generators. The proportionality constants depend on normalisation
conventions, but the qualitative statement is: \emph{couplings measure
the non-commutativity of appropriate rotor directions}.

\section*{Weinberg angle as an internal angle}

Electroweak unification mixes weak isospin $SU(2)_L$ and hypercharge
$U(1)_Y$ into the physical photon $A_\mu$ and $Z_\mu$ boson. In the
octonionic rotor picture, this mixing is literally an angle in operator
space.

Let $G_W$ denote the (properly normalised) weak-isospin rotor in the
relevant direction and $G_Y$ the hypercharge rotor constructed from the
internal algebra. Then one can define an angle $\theta$ by
\[
  \cos\theta =
  \frac{\langle G_W, G_Y\rangle}
       {\|G_W\|\,\|G_Y\|}.
\]

Up to renormalisation effects, this geometric angle is identified with the
Weinberg angle $\theta_W$, and the usual electroweak relations
\[
  e = g\sin\theta_W,\qquad
  g' = g\tan\theta_W
\]
are reinterpreted as relations between norms and inner products of rotor
directions.

\section*{Relations among $\alpha$, $\alpha_s$ and $\sin^2\theta_W$}

Because all three quantities are read from the same operator space, they
are not arbitrary:

\begin{itemize}
  \item The relative normalisation of $SU(3)$, $SU(2)$ and $U(1)$ generators
        is fixed by the representation of the exceptional algebra.
  \item This fixes ratios of norms like
        $\|T_a\|^2 : \|G_W\|^2 : \|G_Y\|^2$.
  \item Consequently, at an appropriate reference scale, the couplings
        are correlated.
\end{itemize}

Schematisch:
\[
  \alpha : \alpha_s : \frac{1}{\sin^2\theta_W}
  \;\sim\;
  \|Q\|^2 : \|T_a\|^2 : \frac{\|G_W\|^2}{\|G_Y\|^2},
\]
mit allen Größen aus demselben Operatorraum gelesen. Laufende mit der
Energie kommt zusätzlich durch Renormierungsgruppeneffekte; die
Ausgangswerte sind jedoch geometrisch eingeengt.

\section*{Was das konzeptionell ändert}

Die Weinberg-Winkel-Tag soll weniger eine neue Zahl liefern als den
Blickwinkel ändern:

\begin{enumerate}
  \item $\alpha$, $\alpha_s$ und $\sin^2\theta_W$ sind keine völlig
        unabhängigen Parameter, sondern verschiedene Projektionen derselben
        internen Operatorgeometrie.
  \item Der Weinberg-Winkel wird zu dem, was sein Name verspricht:
        einem \emph{Winkel} zwischen zwei ausgezeichneten Rotorrichtungen.
  \item Mögliche Relationen zwischen Kopplungen sind keine Zufälle, sondern
        Fingerabdrücke der Einbettung von
        $SU(3)_C\times SU(2)_L\times U(1)_Y$ in die Ausnahmegeometrie.
\end{enumerate}

Wenn spätere Rechnungen zeigen, dass die experimentell gemessenen Werte
von $\alpha$, $\alpha_s$ und $\sin^2\theta_W$ sich gut als Rotor-Normen
und -Winkel eines konkreten Oktaven-/Albert-Embeddings rekonstruieren
lassen, wäre das ein starkes Indiz dafür, dass „innere Geometrie“ mehr ist
als eine Metapher.

\small
\begin{thebibliography}{9}

\bibitem{Weinberg1967}
S.~Weinberg,
\newblock ``A model of leptons,''
\newblock {\em Phys.\ Rev.\ Lett.} \textbf{19}, 1264--1266 (1967).

\bibitem{Furey2018}
C.~Furey,
\newblock ``$SU(3)_C\times SU(2)_L \times U(1)_Y$ from division algebras,''
\newblock {\em Phys.\ Lett.\ B} \textbf{785}, 84--89 (2018).

\bibitem{Internal}
[Internal notes on rotor norms and couplings:
{\tt arxiv-const.tex; appK\_neu.tex}.]

\end{thebibliography}
\normalsize

  } % end #6 body
  {In the rotor picture, couplings and the Weinberg angle are not
   arbitrary constants but norms and mutual angles of commutators in the
   exceptional internal operator space.} % #7 Closing

\end{document}