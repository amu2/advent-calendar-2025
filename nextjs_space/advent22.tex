% advent22.tex
% December 22, 2025 – Constants as equilibrium values of exceptional geometry

\documentclass[a4paper,10pt]{article}

\usepackage[utf8]{inputenc}
\usepackage[T1]{fontenc}
\usepackage[english]{babel}
\usepackage{amsmath,amssymb,amsfonts}
\usepackage{xcolor}
\usepackage{tikz}
\usetikzlibrary{calc}
\usepackage[framemethod=TikZ]{mdframed}
\usepackage{titlesec}
\usepackage{lettrine}
\usepackage{eso-pic}
\usepackage{geometry}
\usepackage{multicol}
\usepackage{lmodern}

\geometry{margin=2.0cm}

% advent-layout.tex (corrected for \input usage)

\definecolor{adventred}{HTML}{B3001B}
\definecolor{adventblue}{HTML}{003366}
\definecolor{adventgreen}{HTML}{006633}
\definecolor{adventgold}{HTML}{B59410}

\pagestyle{empty}

\titleformat{\section}
  {\normalfont\large\bfseries\color{adventblue}}{\thesection}{0em}{}
  
\titleformat{\subsection}
  {\normalfont\normalsize\bfseries\color{adventblue}}{\thesubsection}{0em}{}

%----
% AdventFrameTop
%----

\newenvironment{AdventFrameTop}
{%
  \begin{mdframed}[
    linecolor=adventgreen!0,
    linewidth=0pt,
    roundcorner=0pt,
    innertopmargin=10pt,
    innerbottommargin=10pt,
    innerleftmargin=10pt,
    innerrightmargin=10pt,
    backgroundcolor=adventgreen!2
  ]%
}
{%
  \end{mdframed}
}

\newcommand{\BeginAdventPage}{}
\newcommand{\EndAdventPage}{}

%----
% AdventTitleBlock
%----

\newcommand{\AdventTitleBlock}[4]{%
  \begin{center}
    {\Large\textcolor{adventred}{\textbf{#1}}}\par\vspace{4pt}%
    \ifx&#2&\else
      {\large\textbf{#2}}\par\vspace{2pt}%
    \fi
    {\Large\textcolor{adventblue}{\textbf{#3}}}\par
    \ifx&#4&\else
      \vspace{2pt}%
      {\normalsize\textbf{#4}}\par%
    \fi
  \end{center}%
}

%----
% AdventKeyInsight
%----

\newcommand{\AdventKeyInsight}[1]{%
  \vspace{0.5em}%
  \noindent\colorbox{adventred!8}{%
    \parbox{\dimexpr\linewidth-2\fboxsep}{%
    \textbf{\textcolor{adventred}{Key Insight.}}~#1%
    }%
  }%
  \vspace{0.5em}%
}

%----
% AdventStarRule
%----

\newcommand{\AdventStarRule}{%
  \vspace{0.3em}%
  \begin{center}
    {\color{adventgold}%
    \rule[0.5ex]{0.25\linewidth}{0.4pt}\;
    $\ast\;\ast\;\ast$\;
    \rule[0.5ex]{0.25\linewidth}{0.4pt}%
    }%
  \end{center}
  \vspace{0.3em}%
}

%----
% AdventClosing
%----

\newcommand{\AdventClosing}[1]{%
  \vspace{0.4em}%
  \begin{center}
    \textcolor{adventgreen}{\emph{#1}}%
  \end{center}
}

%----
% AdventAuthor
%----

\newcommand{\AdventAuthor}{%
  \AddToShipoutPictureFG{%
    \begin{tikzpicture}[remember picture,overlay]
    \node[anchor=south, yshift=2mm] at (current page.south) {%
    \footnotesize Andreas Müller, Kempten University of Applied Sciences, %
    \texttt{andreas.mueller@hs-kempten.de}%
    };
    \end{tikzpicture}%
  }%
}

%----
% AdventInitial
%----

\newcommand{\AdventInitial}[2]{%
  \lettrine[lines=2,lhang=0.1,loversize=0.15]%
    {\textcolor{adventred}{#1}}%
    {#2}%
}

%----
% AdventPageBackground
%----

\newcommand{\AdventPageBackground}{%
  \AddToShipoutPictureBG{%
    \begin{tikzpicture}[remember picture,overlay]
    \draw[adventgreen!80!black, line width=3pt, rounded corners=12pt]
    ($(current page.north west)+(0.8cm,-0.8cm)$)
    rectangle
    ($(current page.south east)+(-0.8cm,0.8cm)$);
    \fill[adventgold]
    ($(current page.north west)+(1.0cm,-1.0cm)$) circle (1.2pt)
    ($(current page.north east)+(-1.0cm,-1.0cm)$) circle (1.2pt)
    ($(current page.south west)+(1.0cm,1.0cm)$) circle (1.2pt)
    ($(current page.south east)+(-1.0cm,1.0cm)$) circle (1.2pt);
    \end{tikzpicture}
  }%
}

%---- AdventSheet macro (1-page, single column) ----
% #1: Date + occasion
% #2: unused
% #3: Main title
% #4: Subtitle
% #5: Key Insight
% #6: Main body content
% #7: Closing statement
\newcommand{\AdventSheet}[7]{%
  \BeginAdventPage
  \vspace*{1cm}
  \begin{AdventFrameTop}
    \AdventTitleBlock{#1}{#2}{#3}{#4}
    \AdventKeyInsight{#5}
  \end{AdventFrameTop}
  \AdventStarRule
  #6%
  \EndAdventPage
  \AdventClosing{#7}%
}

%---- AdventSheetTwoCol macro (two-column hero sheet) ----
% #1: Date + occasion
% #2: unused
% #3: Main title
% #4: Subtitle
% #5: Key Insight
% #6: Main body content (wrapped in multicols{2})
% #7: Closing statement
\newcommand{\AdventSheetTwoCol}[7]{%
  \BeginAdventPage
  \begin{AdventFrameTop}
    \AdventTitleBlock{#1}{#2}{#3}{#4}
    \AdventKeyInsight{#5}
  \end{AdventFrameTop}
  \AdventStarRule
  \begin{multicols}{2}
    #6%
  \end{multicols}
  \EndAdventPage
  \AdventClosing{#7}%
}

\begin{document}

\AdventPageBackground
\AdventAuthor

\AdventSheetTwoCol
  {December 22, 2025} % #1 Date
  {}                  % #2 unused
  {Constants as equilibrium values of exceptional geometry} % #3 Main title
  {From ``input parameters'' to attractors and minima}      % #4 Subtitle
  {In the textbook Standard Model, parameters like the fine-structure
   constant $\alpha$, the strong coupling $\alpha_s$, the Weinberg angle
   $\theta_W$, fermion masses and mixing angles appear as independent
   inputs. In the octonionic/Albert framework, many of them can be read as
   \emph{equilibrium values}: minima of an $F_4$-symmetric potential,
   attractor radii of an internal flow, or angles between preferred rotor
   directions. Physical constants become what the internal exceptional
   geometry \emph{settles into}, not arbitrary dials.} % #5 Key Insight
  { % #6 body (two columns)

\section*{Constants vs.\ equilibrium values}

\AdventInitial{I}{n} everyday language, a ``constant of nature'' is a
number that does not change over time. But for theory building there is a
deeper question:

\begin{quote}
  Is the constant a fundamental input, or is it the value that some
  dynamical or geometric system relaxes to?
\end{quote}

In statistical mechanics, equilibrium temperatures and densities are not
fundamental; they are fixed points of microscopic dynamics. The octonionic
model applies this logic to parts of high-energy physics:

\begin{itemize}
  \item Couplings and angles become norms and overlaps in rotor space.
  \item Mass scales become attractor radii of an internal flow.
  \item Order parameters become minima of an $F_4$-invariant potential.
\end{itemize}

\section*{Electroweak scale as a potential minimum}

A central example from the 4th Advent (21 December) is the electroweak
scale $Y_S$. A Jordan potential
\[
  V_J(H) = \mu^2 \,\mathrm{Tr}(H^2)
         + \lambda\,\mathrm{Tr}(H^4)
         + \kappa\,c\,\det H
         + \cdots
\]
on $H_3(\mathbb{O})$, invariant under $F_4$, has a minimum at some vacuum
$\langle H\rangle$. Reading $Y_S$ as a suitable component or invariant of
$\langle H\rangle$ and minimising $V_J$ yields a relation of the type
\[
  Y_S^2 = -\frac{\mu^2}{2(\lambda + \kappa c)}.
\]

Here $Y_S$ is not an arbitrary input; it is the equilibrium value of the
order parameter in an exceptional potential. The observed electroweak
scale is then analogous to a lattice spacing in a crystal: an equilibrium
distance, not a fundamental length.

\section*{Angles and couplings as geometric overlaps}

From 10 December we remember: couplings and the Weinberg angle can be read
from rotor geometry. With an inner product
\[
  \langle A,B\rangle = \mathrm{Tr}(A^\dagger B),
\]
and rotor directions $G_W$ (weak isospin) and $G_Y$ (hypercharge), one
defines
\[
  \cos\theta_W =
  \frac{\langle G_W,G_Y\rangle}
       {\|G_W\|\,\|G_Y\|}.
\]

Similarly, $\alpha$ and $\alpha_s$ track the norms of appropriate
commutators in the internal operator space. Once the embedding of
$SU(3)_C\times SU(2)_L\times U(1)_Y$ into the exceptional algebra is
fixed, the \emph{relative} normalisation of these norms is fixed as well.
Running with energy complicates the picture, but the low-energy values are
no longer independent dials; they are correlated by the underlying
geometry.

In this sense, $\theta_W$ and the ratios between $\alpha$ and $\alpha_s$
are \emph{geometric equilibrium values}: the angles and norms selected by a
particular embedding of the Standard Model group into the exceptional
symmetry.

\section*{Mass hierarchies as attractor radii}

The radius operator and attractor mechanism from earlier days provide
another route to equilibrium values. Internal flows in the space of
configurations—driven by renormalisation-group or effective potential
gradients—develop attractors at characteristic radii
\[
  R \sim R_{\ast,1},\; R_{\ast,2},\;\ldots
\]
corresponding to:

\begin{itemize}
  \item low-energy QCD scale,
  \item electroweak scale,
  \item higher intermediate scales.
\end{itemize}

Fermion masses then fall into bands or shells around these radii, with the
precise eigenvalues determined by the compressor spectra. The huge
hierarchies between these radii (e.g.\ Planck vs.\ electroweak) can be
traced back to nonassociativity (8 December): strong deviations from
associativity produce exponential separations between attractor radii.

Thus, mass ratios are not ``inserted by hand''; they are equilibrium
outcomes of flows on an exceptional internal manifold.

\section*{What remains external}

The equilibrium perspective is powerful but not unlimited. In the current
state of the model:

\begin{itemize}
  \item The gravitational coupling—equivalently the Planck mass $m_P$—is
        still external (16 December).
  \item The ratio $\kappa = m_p/m_P$ is acknowledged as an input, awaiting
        a spectral-geometry derivation.
  \item The cosmological constant is not yet computed from an octonionic
        spectral action.
\end{itemize}

So the slogan ``constants are equilibrium values'' is true for many, but
not yet for all of them. The calendar is explicit about this boundary.

\section*{Conceptual shift: from parameter lists to geometry}

The main conceptual gain of this day is not a new formula but a new
classification:

\begin{enumerate}
  \item \textbf{Equilibrium constants:}
    values fixed by minima of potentials, attractors of flows, or fixed
    geometric overlaps (e.g.\ $Y_S$, many fermion masses, $\theta_W$,
    relative couplings).
  \item \textbf{Spectral constants (future work):}
    values to be derived from the spectrum of a combined Dirac operator in
    spectral geometry (e.g.\ $G$, cosmological constant).
  \item \textbf{Still-external constants:}
    numbers not yet tied to the exceptional structure in the present
    model.
\end{enumerate}

This is a clear upgrade over a flat parameter list: we know which constants
are already inside the exceptional story, which should be pulled in via
spectral geometry, and which ones remain open.

\section*{Why this matters for physics, not just for aesthetics}

From a pragmatic viewpoint, one might say: as long as the model matches
data, who cares whether a parameter is an input or an equilibrium value?
The answer is twofold:

\begin{itemize}
  \item \textbf{Predictivity:} Equilibrium values are constrained: small
        changes in the microscopic setup lead to calculable shifts, not
        arbitrary variations. This opens the door to quantitative tests.
  \item \textbf{Stability:} Equilibrium interpretation explains why
        certain values are robust under perturbations, decoupling of heavy
        fields, or environmental changes.
\end{itemize}

If, step by step, more of the Standard Model parameter list can be turned
into equilibrium values of an exceptional geometry, the notorious
``fine-tuning'' problems change character: we no longer tune against
nothing; we characterise which equilibria of which internal structures are
realised in our Universe.

\small
\begin{thebibliography}{9}

\bibitem{ConnesChamseddine}
A.~Connes and A.~H.~Chamseddine,
\newblock ``The spectral action principle,''
\newblock {\em Commun.\ Math.\ Phys.} \textbf{186}, 731--750 (1997).

\bibitem{GurseyTze1996}
F.~G\"ursey and H.~C.~Tze,
\newblock {\em On the Role of Division, Jordan and Related Algebras in
Particle Physics},
\newblock World Scientific, 1996.

\bibitem{Internal}
[Internal notes on attractors, rotor norms and constants as equilibria:
{\tt arxiv-const.tex; appK\_neu.tex; appE\_neu.tex}.]

\end{thebibliography}
\normalsize

  } % end #6 body
  {In the exceptional picture, many ``constants'' are not arbitrary inputs
   but equilibrium values: minima of $F_4$-symmetric potentials, attractor
   radii and geometric angles in an octonionic internal space.} % #7 Closing

\end{document}