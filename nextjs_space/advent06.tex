% advent-06-12-nikolaus-v2.tex
% Complete standalone document for December 6, 2025 – Nikolaus
% Two-column layout, English text

\documentclass[a4paper,10pt]{article}

% Packages
\usepackage[utf8]{inputenc}
\usepackage[T1]{fontenc}
\usepackage[english]{babel}
\usepackage{amsmath,amssymb,amsfonts}
\usepackage{xcolor}
\usepackage{tikz}
\usetikzlibrary{calc}
\usepackage[framemethod=TikZ]{mdframed}\usepackage{titlesec}
\usepackage{lettrine}
\usepackage{eso-pic}
\usepackage{geometry}
\usepackage{multicol}
\usepackage{lmodern}

% Page geometry
\geometry{
  margin=2.0cm
}

% advent-layout.tex (corrected for \input usage)

\definecolor{adventred}{HTML}{B3001B}
\definecolor{adventblue}{HTML}{003366}
\definecolor{adventgreen}{HTML}{006633}
\definecolor{adventgold}{HTML}{B59410}

\pagestyle{empty}

\titleformat{\section}
  {\normalfont\large\bfseries\color{adventblue}}{\thesection}{0em}{}
  
\titleformat{\subsection}
  {\normalfont\normalsize\bfseries\color{adventblue}}{\thesubsection}{0em}{}

%----
% AdventFrameTop
%----

\newenvironment{AdventFrameTop}
{%
  \begin{mdframed}[
    linecolor=adventgreen!0,
    linewidth=0pt,
    roundcorner=0pt,
    innertopmargin=10pt,
    innerbottommargin=10pt,
    innerleftmargin=10pt,
    innerrightmargin=10pt,
    backgroundcolor=adventgreen!2
  ]%
}
{%
  \end{mdframed}
}

\newcommand{\BeginAdventPage}{}
\newcommand{\EndAdventPage}{}

%----
% AdventTitleBlock
%----

\newcommand{\AdventTitleBlock}[4]{%
  \begin{center}
    {\Large\textcolor{adventred}{\textbf{#1}}}\par\vspace{4pt}%
    \ifx&#2&\else
      {\large\textbf{#2}}\par\vspace{2pt}%
    \fi
    {\Large\textcolor{adventblue}{\textbf{#3}}}\par
    \ifx&#4&\else
      \vspace{2pt}%
      {\normalsize\textbf{#4}}\par%
    \fi
  \end{center}%
}

%----
% AdventKeyInsight
%----

\newcommand{\AdventKeyInsight}[1]{%
  \vspace{0.5em}%
  \noindent\colorbox{adventred!8}{%
    \parbox{\dimexpr\linewidth-2\fboxsep}{%
    \textbf{\textcolor{adventred}{Key Insight.}}~#1%
    }%
  }%
  \vspace{0.5em}%
}

%----
% AdventStarRule
%----

\newcommand{\AdventStarRule}{%
  \vspace{0.3em}%
  \begin{center}
    {\color{adventgold}%
    \rule[0.5ex]{0.25\linewidth}{0.4pt}\;
    $\ast\;\ast\;\ast$\;
    \rule[0.5ex]{0.25\linewidth}{0.4pt}%
    }%
  \end{center}
  \vspace{0.3em}%
}

%----
% AdventClosing
%----

\newcommand{\AdventClosing}[1]{%
  \vspace{0.4em}%
  \begin{center}
    \textcolor{adventgreen}{\emph{#1}}%
  \end{center}
}

%----
% AdventAuthor
%----

\newcommand{\AdventAuthor}{%
  \AddToShipoutPictureFG{%
    \begin{tikzpicture}[remember picture,overlay]
    \node[anchor=south, yshift=2mm] at (current page.south) {%
    \footnotesize Andreas Müller, Kempten University of Applied Sciences, %
    \texttt{andreas.mueller@hs-kempten.de}%
    };
    \end{tikzpicture}%
  }%
}

%----
% AdventInitial
%----

\newcommand{\AdventInitial}[2]{%
  \lettrine[lines=2,lhang=0.1,loversize=0.15]%
    {\textcolor{adventred}{#1}}%
    {#2}%
}

%----
% AdventPageBackground
%----

\newcommand{\AdventPageBackground}{%
  \AddToShipoutPictureBG{%
    \begin{tikzpicture}[remember picture,overlay]
    \draw[adventgreen!80!black, line width=3pt, rounded corners=12pt]
    ($(current page.north west)+(0.8cm,-0.8cm)$)
    rectangle
    ($(current page.south east)+(-0.8cm,0.8cm)$);
    \fill[adventgold]
    ($(current page.north west)+(1.0cm,-1.0cm)$) circle (1.2pt)
    ($(current page.north east)+(-1.0cm,-1.0cm)$) circle (1.2pt)
    ($(current page.south west)+(1.0cm,1.0cm)$) circle (1.2pt)
    ($(current page.south east)+(-1.0cm,1.0cm)$) circle (1.2pt);
    \end{tikzpicture}
  }%
}

%---- AdventSheet macro (1-page, single column) ----
% #1: Date + occasion
% #2: unused
% #3: Main title
% #4: Subtitle
% #5: Key Insight
% #6: Main body content
% #7: Closing statement
\newcommand{\AdventSheet}[7]{%
  \BeginAdventPage
  \vspace*{1cm}
  \begin{AdventFrameTop}
    \AdventTitleBlock{#1}{#2}{#3}{#4}
    \AdventKeyInsight{#5}
  \end{AdventFrameTop}
  \AdventStarRule
  #6%
  \EndAdventPage
  \AdventClosing{#7}%
}

%---- AdventSheetTwoCol macro (two-column hero sheet) ----
% #1: Date + occasion
% #2: unused
% #3: Main title
% #4: Subtitle
% #5: Key Insight
% #6: Main body content (wrapped in multicols{2})
% #7: Closing statement
\newcommand{\AdventSheetTwoCol}[7]{%
  \BeginAdventPage
  \begin{AdventFrameTop}
    \AdventTitleBlock{#1}{#2}{#3}{#4}
    \AdventKeyInsight{#5}
  \end{AdventFrameTop}
  \AdventStarRule
  \begin{multicols}{2}
    #6%
  \end{multicols}
  \EndAdventPage
  \AdventClosing{#7}%
}


\begin{document}

\AdventPageBackground
\AdventAuthor

\AdventSheetTwoCol
  {December 6, 2025 \large (Nikolaus)} % #1 Date + occasion
  {}                                   % #2 unused
  {One Single Equation $D\Psi = 0$}    % #3 Main title
  {Matrix transport instead of a three-theory zoo} % #4 Subtitle
  {Instead of writing three separate sets of field equations for matter (Dirac),
   gauge fields (Yang--Mills) and gravity (Einstein), the octonionic model
   starts from a \emph{single} matrix transport equation on $\mathbb{R}^8$:
   $D\Psi = 0$ with $D = \partial + A$, where the connection $A_\mu \in \mathfrak{so}(8)$
   contains both the spin connection (gravity) and all internal gauge fields.
   The familiar equations reappear as \emph{projections} of this one master equation.} % #5 Key Insight
  { % #6 Main body (two columns)

\section*{From three theories to one equation}

\AdventInitial{T}{oday} we usually write down three logically distinct
structures to describe fundamental physics:
a Dirac equation for matter fields, Yang--Mills equations for the gauge bosons,
and Einstein's field equations for gravitation.
This split is historically grown and pragmatically useful, but it hides the
fact that all three are, at heart, transport equations for some kind of
``spinor data'' along some kind of connection.

The octonionic model takes this observation seriously and elevates it to a
principle: \emph{there is only one transport equation}, written purely in
terms of a matrix-valued connection on an $8$-dimensional real vector space.
Everything else --- matter dynamics, gauge-field dynamics, and effective
gravitational dynamics --- is obtained by projecting this single equation
onto different sectors.

\section*{The central structure}

The starting point is an $\mathfrak{so}(8)$-valued connection $A_\mu$ on
$\mathbb{R}^8$, acting on an $8$-component field $\Psi$:
\[
  D\Psi \;=\; 0,
  \qquad
  D \;=\; \partial + A,
  \qquad
  A_\mu \in \mathfrak{so}(8).
\]

Here:
\begin{itemize}
  \item The \emph{kinematic} part $\partial$ encodes flat $\mathbb{R}^8$
        as the basic stage on which everything lives.
  \item The \emph{connection} $A_\mu$ decomposes into
        \[
          A_\mu \;=\;
          \Gamma_\mu \;\oplus\; A_\mu^{\text{int}},
        \]
        where $\Gamma_\mu$ is recognized as a spin connection (gravity)
        and $A_\mu^{\text{int}}$ as the internal gauge fields associated with
        $SU(3)\times SU(2)\times U(1)$, embedded in the octonionic geometry.
  \item The field $\Psi$ carries both gravitational and internal quantum
        numbers; its components are organized according to the
        octonionic/Spin(8) representation structure.
\end{itemize}

In this picture:
\begin{itemize}
  \item The \textbf{Dirac equation} is the projection of $D\Psi=0$ onto the
        fermionic component of $\Psi$, in a sector where $A_\mu$ is treated
        as a fixed background.
  \item The \textbf{Yang--Mills equations} arise as the compatibility
        conditions (integrability) for $D\Psi=0$, expressed as constraints
        on the curvature $F_{\mu\nu} = [D_\mu,D_\nu]$ in the internal
        directions.
  \item The \textbf{Einstein equations} (or their effective counterpart)
        arise from the curvature components of $\Gamma_\mu$ and from the
        spectral action built from $D^2$ in the gravitational sector.
\end{itemize}

\section*{Physical meaning}

From the perspective of the octonionic model, $D\Psi=0$ is not ``just
another compact notation'' for the Standard Model + gravity. It is a
statement about the \emph{underlying current} in the octonionic/Albert
geometry:

\begin{itemize}
  \item The operator $D$ encodes both the geometric background (through
        $\Gamma_\mu$) and the internal symmetry structure (through
        $A_\mu^{\text{int}}$).
  \item The field $\Psi$ encodes the ``state of the universe'' as a section
        of a bundle whose fibers are built from octonionic representations.
  \item The equation $D\Psi=0$ enforces that this state is covariantly
        constant along all directions in $\mathbb{R}^8$ with respect to $A$.
\end{itemize}

This has several conceptual consequences:

\begin{enumerate}
  \item \textbf{Unification of kinematics and interactions.}
        There is no kinematics ``without'' a connection: as soon as we write
        $D = \partial + A$, gravitational and gauge information are hard-wired
        into the basic notion of a derivative on $\mathbb{R}^8$.
  \item \textbf{Operators before fields.}
        The primary object is the operator $D$, not a list of classical
        fields. Matter, gauge bosons and even gravitation appear as different
        faces of $D$ and its curvature, in line with spectral geometry.
  \item \textbf{Natural home for the octonionic structure.}
        The choice $A_\mu \in \mathfrak{so}(8)$, with its Spin(8) triality
        and $G_2$-compatible substructures, is not an accident: it is the
        unique stage where the octonionic and Albert-algebra data fit
        together into a single transport equation.
\end{enumerate}

\small
\begin{thebibliography}{9}

\bibitem{Connes1994}
A.~Connes,
\newblock {\em Noncommutative Geometry},
\newblock Academic Press, 1994.

\bibitem{Haag1996}
R.~Haag,
\newblock {\em Local Quantum Physics: Fields, Particles, Algebras},
\newblock Springer, 1996.

\end{thebibliography}
\normalsize

  } % end #6
  {One single matrix equation replaces three classical theories as independent starting points.} % #7 Closing
  
\end{document}