% advent04.tex
% December 4, 2025 – Heptagon operator: seven directions, one spectrum

\documentclass[a4paper,10pt]{article}

\usepackage[utf8]{inputenc}
\usepackage[T1]{fontenc}
\usepackage[english]{babel}
\usepackage{amsmath,amssymb,amsfonts}
\usepackage{xcolor}
\usepackage{tikz}
\usetikzlibrary{calc}
\usepackage[framemethod=TikZ]{mdframed}
\usepackage{titlesec}
\usepackage{lettrine}
\usepackage{eso-pic}
\usepackage{geometry}
\usepackage{multicol}
\usepackage{lmodern}

\geometry{margin=2.0cm}

% advent-layout.tex (corrected for \input usage)

\definecolor{adventred}{HTML}{B3001B}
\definecolor{adventblue}{HTML}{003366}
\definecolor{adventgreen}{HTML}{006633}
\definecolor{adventgold}{HTML}{B59410}

\pagestyle{empty}

\titleformat{\section}
  {\normalfont\large\bfseries\color{adventblue}}{\thesection}{0em}{}
  
\titleformat{\subsection}
  {\normalfont\normalsize\bfseries\color{adventblue}}{\thesubsection}{0em}{}

%----
% AdventFrameTop
%----

\newenvironment{AdventFrameTop}
{%
  \begin{mdframed}[
    linecolor=adventgreen!0,
    linewidth=0pt,
    roundcorner=0pt,
    innertopmargin=10pt,
    innerbottommargin=10pt,
    innerleftmargin=10pt,
    innerrightmargin=10pt,
    backgroundcolor=adventgreen!2
  ]%
}
{%
  \end{mdframed}
}

\newcommand{\BeginAdventPage}{}
\newcommand{\EndAdventPage}{}

%----
% AdventTitleBlock
%----

\newcommand{\AdventTitleBlock}[4]{%
  \begin{center}
    {\Large\textcolor{adventred}{\textbf{#1}}}\par\vspace{4pt}%
    \ifx&#2&\else
      {\large\textbf{#2}}\par\vspace{2pt}%
    \fi
    {\Large\textcolor{adventblue}{\textbf{#3}}}\par
    \ifx&#4&\else
      \vspace{2pt}%
      {\normalsize\textbf{#4}}\par%
    \fi
  \end{center}%
}

%----
% AdventKeyInsight
%----

\newcommand{\AdventKeyInsight}[1]{%
  \vspace{0.5em}%
  \noindent\colorbox{adventred!8}{%
    \parbox{\dimexpr\linewidth-2\fboxsep}{%
    \textbf{\textcolor{adventred}{Key Insight.}}~#1%
    }%
  }%
  \vspace{0.5em}%
}

%----
% AdventStarRule
%----

\newcommand{\AdventStarRule}{%
  \vspace{0.3em}%
  \begin{center}
    {\color{adventgold}%
    \rule[0.5ex]{0.25\linewidth}{0.4pt}\;
    $\ast\;\ast\;\ast$\;
    \rule[0.5ex]{0.25\linewidth}{0.4pt}%
    }%
  \end{center}
  \vspace{0.3em}%
}

%----
% AdventClosing
%----

\newcommand{\AdventClosing}[1]{%
  \vspace{0.4em}%
  \begin{center}
    \textcolor{adventgreen}{\emph{#1}}%
  \end{center}
}

%----
% AdventAuthor
%----

\newcommand{\AdventAuthor}{%
  \AddToShipoutPictureFG{%
    \begin{tikzpicture}[remember picture,overlay]
    \node[anchor=south, yshift=2mm] at (current page.south) {%
    \footnotesize Andreas Müller, Kempten University of Applied Sciences, %
    \texttt{andreas.mueller@hs-kempten.de}%
    };
    \end{tikzpicture}%
  }%
}

%----
% AdventInitial
%----

\newcommand{\AdventInitial}[2]{%
  \lettrine[lines=2,lhang=0.1,loversize=0.15]%
    {\textcolor{adventred}{#1}}%
    {#2}%
}

%----
% AdventPageBackground
%----

\newcommand{\AdventPageBackground}{%
  \AddToShipoutPictureBG{%
    \begin{tikzpicture}[remember picture,overlay]
    \draw[adventgreen!80!black, line width=3pt, rounded corners=12pt]
    ($(current page.north west)+(0.8cm,-0.8cm)$)
    rectangle
    ($(current page.south east)+(-0.8cm,0.8cm)$);
    \fill[adventgold]
    ($(current page.north west)+(1.0cm,-1.0cm)$) circle (1.2pt)
    ($(current page.north east)+(-1.0cm,-1.0cm)$) circle (1.2pt)
    ($(current page.south west)+(1.0cm,1.0cm)$) circle (1.2pt)
    ($(current page.south east)+(-1.0cm,1.0cm)$) circle (1.2pt);
    \end{tikzpicture}
  }%
}

%---- AdventSheet macro (1-page, single column) ----
% #1: Date + occasion
% #2: unused
% #3: Main title
% #4: Subtitle
% #5: Key Insight
% #6: Main body content
% #7: Closing statement
\newcommand{\AdventSheet}[7]{%
  \BeginAdventPage
  \vspace*{1cm}
  \begin{AdventFrameTop}
    \AdventTitleBlock{#1}{#2}{#3}{#4}
    \AdventKeyInsight{#5}
  \end{AdventFrameTop}
  \AdventStarRule
  #6%
  \EndAdventPage
  \AdventClosing{#7}%
}

%---- AdventSheetTwoCol macro (two-column hero sheet) ----
% #1: Date + occasion
% #2: unused
% #3: Main title
% #4: Subtitle
% #5: Key Insight
% #6: Main body content (wrapped in multicols{2})
% #7: Closing statement
\newcommand{\AdventSheetTwoCol}[7]{%
  \BeginAdventPage
  \begin{AdventFrameTop}
    \AdventTitleBlock{#1}{#2}{#3}{#4}
    \AdventKeyInsight{#5}
  \end{AdventFrameTop}
  \AdventStarRule
  \begin{multicols}{2}
    #6%
  \end{multicols}
  \EndAdventPage
  \AdventClosing{#7}%
}

\begin{document}

\AdventPageBackground
\AdventAuthor

\AdventSheetTwoCol
  {December 4, 2025} % #1 Date
  {}                 % #2 unused
  {Heptagon operator: seven directions, one spectrum} % #3 Main title
  {From seven imaginary units to three eigenvalues $(\alpha,\beta,\gamma)$} % #4 Subtitle
  {The seven imaginary octonion units can be arranged on a heptagon that
   encodes their multiplication rules. From this data one constructs the
   \emph{heptagon operator} $H_7$, a single linear operator whose
   eigenvalues $(\alpha,\beta,\gamma)$ summarise the internal geometry.
   These three numbers are not free parameters; they are geometric
   invariants that will later feed into coupling constants and mixing
   angles. Today we introduce $H_7$ and its spectrum as a compact
   fingerprint of the sevenfold structure of $\mathbb{O}$.} % #5 Key Insight
  { % #6 body (two columns)

\section*{Seven imaginary units on a heptagon}

\AdventInitial{O}{ctonions} $\mathbb{O}$ have seven imaginary units
$e_1,\dots,e_7$. Their multiplication can be depicted on an oriented
heptagon: each directed edge (and certain chords) carries a triple
$(e_i,e_j,e_k)$ with

\[
  e_i e_j = e_k,\qquad
  e_j e_k = e_i,\qquad
  e_k e_i = e_j,
\]

and reversed order introduces a minus sign. This Fano-type diagram is more
than a mnemonic; it packages the nonassociative multiplication table into
a single geometric picture.

\section*{Constructing the heptagon operator}

From the seven units $e_i$ and the heptagon geometry, one defines a linear
operator $H_7$ on the underlying real 8-dimensional space. Schematic
example:

\[
  H_7 = \sum_{i=1}^7 c_i\,L_{e_i},
\]

where $L_{e_i}:x\mapsto e_i x$ denotes left multiplication by $e_i$, and
the coefficients $c_i$ are chosen to respect the heptagon symmetry (for
instance, equal on edges belonging to the same orbit under $G_2$).

The important facts are:

\begin{itemize}
  \item $H_7$ encodes all seven directions in a single operator.
  \item $H_7$ is constrained by $G_2$ symmetry, so its spectrum is highly
        structured.
\end{itemize}

Details of the precise coefficients are not needed on this page; they are
spelled out in the technical notes.

\section*{Spectrum: three eigenvalues $(\alpha,\beta,\gamma)$}

A key feature of $H_7$ is that its eigenspectrum is much simpler than an
arbitrary $8\times8$ matrix would allow. In favourable constructions,

\[
  \mathrm{spec}(H_7) = \{\alpha,\beta,\gamma\},
\]

with appropriate multiplicities that add up to 8. The numbers

\[
  (\alpha,\beta,\gamma)
\]

are then \emph{heptagon eigenvalues}: geometric invariants that capture
how the seven imaginary directions are arranged relative to each other.

Qualitatively:

\begin{itemize}
  \item $\alpha$ may correspond to an eigen-subspace aligned with a
        particular family of edges.
  \item $\beta$ and $\gamma$ correspond to other symmetry-related
        subspaces.
\end{itemize}

The crucial point: instead of seven arbitrary numbers, we get only three
characteristic eigenvalues.

\section*{Why $(\alpha,\beta,\gamma)$ matter for physics}

Later in the calendar, $(\alpha,\beta,\gamma)$ will reappear in several
contexts:

\begin{itemize}
  \item In the construction of the \emph{radius operator} $R$ and its
        spectrum $(a_0,b_0,c_0)$ (5 December).
  \item In geometric expressions for coupling constants, such as the
        fine-structure constant $\alpha$ and the strong coupling
        $\alpha_s$.
  \item In defining angles between rotor directions that enter the
        Weinberg angle $\theta_W$.
\end{itemize}

In other words, the heptagon eigenvalues are a compact \emph{seed} from
which many later observables can be grown. They serve as a small set of
internal numbers that eventually manifest as physical constants.

\section*{Heptagon operator within the operator toolbox}

On the second Advent Sunday (7 December), the calendar will present a full
\emph{operator toolbox}:

\begin{itemize}
  \item heptagon operator $H_7$ with eigenvalues $(\alpha,\beta,\gamma)$,
  \item radius operator $R$ with radii $(a_0,b_0,c_0)$,
  \item sign/signature operators,
  \item rotors (antisymmetric generators of forces),
  \item compressors (symmetric mass and mixing operators).
\end{itemize}

The heptagon operator is the first piece of this toolbox to appear
explicitly. Its role is to turn the combinatorial data ``seven imaginary
units on a heptagon'' into spectral data that can be plugged into
operators, potentials and eventually quantitative formulas.

\section*{Conceptual gain from $H_7$}

Compared to working directly with seven basis elements $e_i$, the
heptagon-operator viewpoint offers:

\begin{enumerate}
  \item \textbf{Compression:} seven directions are summarised by three
    eigenvalues.
  \item \textbf{Invariants:} $(\alpha,\beta,\gamma)$ are invariant under
    $G_2$-compatible re-labellings of the heptagon—they are not artefacts
    of a particular basis choice.
  \item \textbf{Spectral language:} we move from basis-dependent
    multiplication tables to basis-independent spectral data, which is
    the natural language for later spectral geometry.
\end{enumerate}

Thus, $H_7$ is the first place where the octonionic multiplication table
starts to look like something a physicist would recognise as ``spectral
parameters''.

\small
\begin{thebibliography}{9}

\bibitem{Baez2002}
J.~C.~Baez,
\newblock ``The octonions,''
\newblock {\em Bull.\ Amer.\ Math.\ Soc.} \textbf{39}, 145--205 (2002).

\bibitem{Furey2016}
C.~Furey,
\newblock ``Charge quantization from a number operator,''
\newblock {\em Phys.\ Lett.\ B} \textbf{742}, 195--199 (2015).

\bibitem{Internal}
[Internal notes on the heptagon operator and its spectrum:
{\tt chap03\_neu.tex; appK\_neu.tex; oktonionen-basen.tex}.]

\end{thebibliography}
\normalsize

  } % end #6 body
  {The heptagon operator $H_7$ compresses the seven imaginary directions
   of $\mathbb{O}$ into three eigenvalues $(\alpha,\beta,\gamma)$. These
   invariants are the first internal numbers that will later reappear in
   couplings, scales and mixing angles.} % #7 Closing

\end{document}