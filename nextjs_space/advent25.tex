% advent25.tex
% Hero day: December 25, 2025 – Vacuum: the invisible medium of all fields

\documentclass[a4paper,10pt]{article}

\usepackage[utf8]{inputenc}
\usepackage[T1]{fontenc}
\usepackage[english]{babel}
\usepackage{amsmath,amssymb,amsfonts}
\usepackage{xcolor}
\usepackage{tikz}
\usetikzlibrary{calc}
\usepackage[framemethod=TikZ]{mdframed}
\usepackage{titlesec}
\usepackage{lettrine}
\usepackage{eso-pic}
\usepackage{geometry}
\usepackage{multicol}
\usepackage{lmodern}

\geometry{margin=2.0cm}

% advent-layout.tex (corrected for \input usage)

\definecolor{adventred}{HTML}{B3001B}
\definecolor{adventblue}{HTML}{003366}
\definecolor{adventgreen}{HTML}{006633}
\definecolor{adventgold}{HTML}{B59410}

\pagestyle{empty}

\titleformat{\section}
  {\normalfont\large\bfseries\color{adventblue}}{\thesection}{0em}{}
  
\titleformat{\subsection}
  {\normalfont\normalsize\bfseries\color{adventblue}}{\thesubsection}{0em}{}

%----
% AdventFrameTop
%----

\newenvironment{AdventFrameTop}
{%
  \begin{mdframed}[
    linecolor=adventgreen!0,
    linewidth=0pt,
    roundcorner=0pt,
    innertopmargin=10pt,
    innerbottommargin=10pt,
    innerleftmargin=10pt,
    innerrightmargin=10pt,
    backgroundcolor=adventgreen!2
  ]%
}
{%
  \end{mdframed}
}

\newcommand{\BeginAdventPage}{}
\newcommand{\EndAdventPage}{}

%----
% AdventTitleBlock
%----

\newcommand{\AdventTitleBlock}[4]{%
  \begin{center}
    {\Large\textcolor{adventred}{\textbf{#1}}}\par\vspace{4pt}%
    \ifx&#2&\else
      {\large\textbf{#2}}\par\vspace{2pt}%
    \fi
    {\Large\textcolor{adventblue}{\textbf{#3}}}\par
    \ifx&#4&\else
      \vspace{2pt}%
      {\normalsize\textbf{#4}}\par%
    \fi
  \end{center}%
}

%----
% AdventKeyInsight
%----

\newcommand{\AdventKeyInsight}[1]{%
  \vspace{0.5em}%
  \noindent\colorbox{adventred!8}{%
    \parbox{\dimexpr\linewidth-2\fboxsep}{%
    \textbf{\textcolor{adventred}{Key Insight.}}~#1%
    }%
  }%
  \vspace{0.5em}%
}

%----
% AdventStarRule
%----

\newcommand{\AdventStarRule}{%
  \vspace{0.3em}%
  \begin{center}
    {\color{adventgold}%
    \rule[0.5ex]{0.25\linewidth}{0.4pt}\;
    $\ast\;\ast\;\ast$\;
    \rule[0.5ex]{0.25\linewidth}{0.4pt}%
    }%
  \end{center}
  \vspace{0.3em}%
}

%----
% AdventClosing
%----

\newcommand{\AdventClosing}[1]{%
  \vspace{0.4em}%
  \begin{center}
    \textcolor{adventgreen}{\emph{#1}}%
  \end{center}
}

%----
% AdventAuthor
%----

\newcommand{\AdventAuthor}{%
  \AddToShipoutPictureFG{%
    \begin{tikzpicture}[remember picture,overlay]
    \node[anchor=south, yshift=2mm] at (current page.south) {%
    \footnotesize Andreas Müller, Kempten University of Applied Sciences, %
    \texttt{andreas.mueller@hs-kempten.de}%
    };
    \end{tikzpicture}%
  }%
}

%----
% AdventInitial
%----

\newcommand{\AdventInitial}[2]{%
  \lettrine[lines=2,lhang=0.1,loversize=0.15]%
    {\textcolor{adventred}{#1}}%
    {#2}%
}

%----
% AdventPageBackground
%----

\newcommand{\AdventPageBackground}{%
  \AddToShipoutPictureBG{%
    \begin{tikzpicture}[remember picture,overlay]
    \draw[adventgreen!80!black, line width=3pt, rounded corners=12pt]
    ($(current page.north west)+(0.8cm,-0.8cm)$)
    rectangle
    ($(current page.south east)+(-0.8cm,0.8cm)$);
    \fill[adventgold]
    ($(current page.north west)+(1.0cm,-1.0cm)$) circle (1.2pt)
    ($(current page.north east)+(-1.0cm,-1.0cm)$) circle (1.2pt)
    ($(current page.south west)+(1.0cm,1.0cm)$) circle (1.2pt)
    ($(current page.south east)+(-1.0cm,1.0cm)$) circle (1.2pt);
    \end{tikzpicture}
  }%
}

%---- AdventSheet macro (1-page, single column) ----
% #1: Date + occasion
% #2: unused
% #3: Main title
% #4: Subtitle
% #5: Key Insight
% #6: Main body content
% #7: Closing statement
\newcommand{\AdventSheet}[7]{%
  \BeginAdventPage
  \vspace*{1cm}
  \begin{AdventFrameTop}
    \AdventTitleBlock{#1}{#2}{#3}{#4}
    \AdventKeyInsight{#5}
  \end{AdventFrameTop}
  \AdventStarRule
  #6%
  \EndAdventPage
  \AdventClosing{#7}%
}

%---- AdventSheetTwoCol macro (two-column hero sheet) ----
% #1: Date + occasion
% #2: unused
% #3: Main title
% #4: Subtitle
% #5: Key Insight
% #6: Main body content (wrapped in multicols{2})
% #7: Closing statement
\newcommand{\AdventSheetTwoCol}[7]{%
  \BeginAdventPage
  \begin{AdventFrameTop}
    \AdventTitleBlock{#1}{#2}{#3}{#4}
    \AdventKeyInsight{#5}
  \end{AdventFrameTop}
  \AdventStarRule
  \begin{multicols}{2}
    #6%
  \end{multicols}
  \EndAdventPage
  \AdventClosing{#7}%
}

\begin{document}

\AdventPageBackground
\AdventAuthor

\AdventSheetTwoCol
  {December 25, 2025} % #1 Date
  {}                 % #2 unused
  {Vacuum: the invisible medium of all fields} % #3 Main title
  {How $\langle H\rangle$, radii and $Y_S$ structure the ``empty'' state} % #4 Subtitle
  {Vacuum is not empty. In the octonionic model, it is a specific
   configuration of the internal structure: a vacuum expectation value
   $\langle H\rangle$ in $H_3(\mathbb{O})$, a choice of radius spectrum
   $(a_0,b_0,c_0)$ and an electroweak scale $Y_S$. Masses, couplings and
   even vacuum energy arise as properties of this structured medium. The
   vacuum is the invisible actor that sets the stage for all visible
   physics.} % #5 Key Insight
  { % #6 body

\section*{Vacuum as a configuration, not as nothing}

\AdventInitial{I}{n} quantum field theory, the vacuum is the lowest-energy
state of the system. In the octonionic setting, and within the unified master action
$S[D,\Psi]$ introduced on December 24, this state is described
by:

\begin{itemize}
  \item a vacuum configuration $\langle H\rangle \in H_3(\mathbb{O})$,
  \item a radius spectrum $(a_0,b_0,c_0)$ of the operator $R$,
  \item an electroweak scale $Y_S$ defined by the minimum of a potential.
\end{itemize}

Symbolically, one may write the internal vacuum data as

\[
  \mathcal{V}_{\text{int}} = \big(\langle H\rangle; a_0,b_0,c_0; Y_S\big).
\]

Fields are then fluctuations around this background; their masses and
interactions are determined by how they probe this structured medium.

\section*{Masses and couplings from the vacuum}

Several quantities introduced earlier in the calendar are now recognised
as vacuum properties:

\begin{itemize}
  \item Fermion masses: eigenvalues of
        $M = y\,\Pi(\langle H\rangle)$.
  \item Gauge couplings: norms of rotor commutators evaluated in the
        vacuum geometry, e.g.\ $\alpha$, $\alpha_s$, $\sin^2\theta_W$.
  \item Scales: exponentials of $(a_0,b_0,c_0)$ define preferred energies.
\end{itemize}

What looks like a zoo of independent parameters from a naive perspective
turns out to be a small set of derived quantities from
$\mathcal{V}_{\text{int}}$.

\section*{Vacuum energy and dark energy}

The cosmological constant $\Lambda$ can be understood as a contribution
to the effective action coming from the spectrum of an appropriate Dirac
operator $D$ and the vacuum configuration. In a spectral-action viewpoint,

\[
  S_{\text{spec}} = \mathrm{Tr}\, f(D^2/\Lambda^2)
\]

contains a leading term that behaves like a cosmological constant (consistent with the gravitational and defect terms in the master
action $S[D,\Psi]$ discussed on the previous day). In the
octonionic setting:

\begin{itemize}
  \item The internal structure encoded in $\langle H\rangle$ and $R$
        influences the spectrum of $D$.
  \item The corresponding vacuum energy becomes a property of the same
        medium that gives rise to masses and couplings.
\end{itemize}

This suggests a picture in which dark energy is not an exotic fluid but
an aspect of the spectral vacuum defined by the octonionic geometry.

\section*{Vacuum as an invisible actor}

It is tempting to think of particles and forces as the main characters of
physics. In this model, the vacuum plays a more central role:

\begin{itemize}
  \item It determines the scales at which different phenomena occur.
  \item It shapes the spectrum of excitations (masses).
  \item It contributes to the energy content of the universe.
\end{itemize}

Most of the ``mystery parameters'' of particle physics are reinterpreted
as properties of a single invisible actor: the structured vacuum state.

\section*{Conceptual shift}

The conceptual shift can be summarised as follows:

\begin{enumerate}
  \item \textbf{From emptiness to medium:}
    vacuum is not the absence of structure but the most structured part
    of the theory.
  \item \textbf{From input to output:}
    constants, scales and masses are not arbitrary inputs but derived
    characteristics of $\mathcal{V}_{\text{int}}$.
  \item \textbf{From fragmentation to coherence:}
    one unified internal configuration explains many seemingly unrelated
    numbers.
\end{enumerate}

This completes the Advent narrative: from abstract octonionic algebra to
a concrete picture in which the vacuum of the universe is a highly
organised medium living in that algebra.

\small
\begin{thebibliography}{9}

\bibitem{Casimir1948}
H.~B.~G.~Casimir,
\newblock ``On the attraction between two perfectly conducting plates,''
\newblock {\em Proc.\ Kon.\ Ned.\ Akad.\ Wet.} \textbf{51}, 793--795 (1948).

\bibitem{Perlmutter1999}
S.~Perlmutter \emph{et al.},
\newblock ``Measurements of $\Omega$ and $\Lambda$ from 42 high-redshift
supernovae,''
\newblock {\em Astrophys.\ J.} \textbf{517}, 565--586 (1999).

\bibitem{Internal}
[Internal notes on vacuum structure and spectral contributions:
{\tt unified-agebra.tex; chap16\_neu.tex; appM\_neu.tex; appE\_neu.tex}.]

\end{thebibliography}
\normalsize

  } % end #6 body
  {In the octonionic model, the vacuum is a structured medium specified by
   $\langle H\rangle$, $(a_0,b_0,c_0)$ and $Y_S$. Masses, couplings and
   even dark-energy-like terms become properties of this medium rather
   than independent inputs, turning the ``empty'' state into the central
   player of the theory.} % #7 Closing

\end{document}