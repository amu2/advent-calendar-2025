% advent11.tex
% December 11, 2025 – Higgs resonance: mode of the vacuum, not the cause of mass

\documentclass[a4paper,10pt]{article}

\usepackage[utf8]{inputenc}
\usepackage[T1]{fontenc}
\usepackage[english]{babel}
\usepackage{amsmath,amssymb,amsfonts}
\usepackage{xcolor}
\usepackage{tikz}
\usetikzlibrary{calc}
\usepackage[framemethod=TikZ]{mdframed}
\usepackage{titlesec}
\usepackage{lettrine}
\usepackage{eso-pic}
\usepackage{geometry}
\usepackage{multicol}
\usepackage{lmodern}

\geometry{margin=2.0cm}

% advent-layout.tex (corrected for \input usage)

\definecolor{adventred}{HTML}{B3001B}
\definecolor{adventblue}{HTML}{003366}
\definecolor{adventgreen}{HTML}{006633}
\definecolor{adventgold}{HTML}{B59410}

\pagestyle{empty}

\titleformat{\section}
  {\normalfont\large\bfseries\color{adventblue}}{\thesection}{0em}{}
  
\titleformat{\subsection}
  {\normalfont\normalsize\bfseries\color{adventblue}}{\thesubsection}{0em}{}

%----
% AdventFrameTop
%----

\newenvironment{AdventFrameTop}
{%
  \begin{mdframed}[
    linecolor=adventgreen!0,
    linewidth=0pt,
    roundcorner=0pt,
    innertopmargin=10pt,
    innerbottommargin=10pt,
    innerleftmargin=10pt,
    innerrightmargin=10pt,
    backgroundcolor=adventgreen!2
  ]%
}
{%
  \end{mdframed}
}

\newcommand{\BeginAdventPage}{}
\newcommand{\EndAdventPage}{}

%----
% AdventTitleBlock
%----

\newcommand{\AdventTitleBlock}[4]{%
  \begin{center}
    {\Large\textcolor{adventred}{\textbf{#1}}}\par\vspace{4pt}%
    \ifx&#2&\else
      {\large\textbf{#2}}\par\vspace{2pt}%
    \fi
    {\Large\textcolor{adventblue}{\textbf{#3}}}\par
    \ifx&#4&\else
      \vspace{2pt}%
      {\normalsize\textbf{#4}}\par%
    \fi
  \end{center}%
}

%----
% AdventKeyInsight
%----

\newcommand{\AdventKeyInsight}[1]{%
  \vspace{0.5em}%
  \noindent\colorbox{adventred!8}{%
    \parbox{\dimexpr\linewidth-2\fboxsep}{%
    \textbf{\textcolor{adventred}{Key Insight.}}~#1%
    }%
  }%
  \vspace{0.5em}%
}

%----
% AdventStarRule
%----

\newcommand{\AdventStarRule}{%
  \vspace{0.3em}%
  \begin{center}
    {\color{adventgold}%
    \rule[0.5ex]{0.25\linewidth}{0.4pt}\;
    $\ast\;\ast\;\ast$\;
    \rule[0.5ex]{0.25\linewidth}{0.4pt}%
    }%
  \end{center}
  \vspace{0.3em}%
}

%----
% AdventClosing
%----

\newcommand{\AdventClosing}[1]{%
  \vspace{0.4em}%
  \begin{center}
    \textcolor{adventgreen}{\emph{#1}}%
  \end{center}
}

%----
% AdventAuthor
%----

\newcommand{\AdventAuthor}{%
  \AddToShipoutPictureFG{%
    \begin{tikzpicture}[remember picture,overlay]
    \node[anchor=south, yshift=2mm] at (current page.south) {%
    \footnotesize Andreas Müller, Kempten University of Applied Sciences, %
    \texttt{andreas.mueller@hs-kempten.de}%
    };
    \end{tikzpicture}%
  }%
}

%----
% AdventInitial
%----

\newcommand{\AdventInitial}[2]{%
  \lettrine[lines=2,lhang=0.1,loversize=0.15]%
    {\textcolor{adventred}{#1}}%
    {#2}%
}

%----
% AdventPageBackground
%----

\newcommand{\AdventPageBackground}{%
  \AddToShipoutPictureBG{%
    \begin{tikzpicture}[remember picture,overlay]
    \draw[adventgreen!80!black, line width=3pt, rounded corners=12pt]
    ($(current page.north west)+(0.8cm,-0.8cm)$)
    rectangle
    ($(current page.south east)+(-0.8cm,0.8cm)$);
    \fill[adventgold]
    ($(current page.north west)+(1.0cm,-1.0cm)$) circle (1.2pt)
    ($(current page.north east)+(-1.0cm,-1.0cm)$) circle (1.2pt)
    ($(current page.south west)+(1.0cm,1.0cm)$) circle (1.2pt)
    ($(current page.south east)+(-1.0cm,1.0cm)$) circle (1.2pt);
    \end{tikzpicture}
  }%
}

%---- AdventSheet macro (1-page, single column) ----
% #1: Date + occasion
% #2: unused
% #3: Main title
% #4: Subtitle
% #5: Key Insight
% #6: Main body content
% #7: Closing statement
\newcommand{\AdventSheet}[7]{%
  \BeginAdventPage
  \vspace*{1cm}
  \begin{AdventFrameTop}
    \AdventTitleBlock{#1}{#2}{#3}{#4}
    \AdventKeyInsight{#5}
  \end{AdventFrameTop}
  \AdventStarRule
  #6%
  \EndAdventPage
  \AdventClosing{#7}%
}

%---- AdventSheetTwoCol macro (two-column hero sheet) ----
% #1: Date + occasion
% #2: unused
% #3: Main title
% #4: Subtitle
% #5: Key Insight
% #6: Main body content (wrapped in multicols{2})
% #7: Closing statement
\newcommand{\AdventSheetTwoCol}[7]{%
  \BeginAdventPage
  \begin{AdventFrameTop}
    \AdventTitleBlock{#1}{#2}{#3}{#4}
    \AdventKeyInsight{#5}
  \end{AdventFrameTop}
  \AdventStarRule
  \begin{multicols}{2}
    #6%
  \end{multicols}
  \EndAdventPage
  \AdventClosing{#7}%
}

\begin{document}

\AdventPageBackground
\AdventAuthor

\AdventSheetTwoCol
  {December 11, 2025} % #1 Date
  {}                  % #2 unused
  {Higgs resonance: mode of the vacuum, not the cause of mass} % #3 Main title
  {Curvature of a Jordan potential around $\langle H\rangle$}  % #4 Subtitle
  {In the Standard Model narrative, the Higgs field ``gives mass'' to
   particles. In the octonionic/Jordan picture this story is inverted:
   masses arise from the eigenvalues of a mass map $\Pi(\langle H\rangle)$,
   constructed from a vacuum element $\langle H\rangle\in H_3(\mathbb{O})$.
   The observed 125 GeV Higgs particle is then a \emph{resonance} of this
   vacuum — a fluctuation mode around $\langle H\rangle$ determined by the
   curvature of the Jordan potential, not the fundamental origin of mass
   itself.} % #5 Key Insight
  { % #6 body (two columns)

\section*{Masses from the vacuum configuration}

\AdventInitial{T}{he} octonionic model encodes internal structure in the
Albert algebra $H_3(\mathbb{O})$. A vacuum configuration is a fixed Jordan
element
\[
  \langle H\rangle \in H_3(\mathbb{O}),
\]
and the mass map is a linear operator
\[
  \Pi(\langle H\rangle) : V_{\text{int}} \to V_{\text{int}},
\]
acting on the internal space of one generation. Its eigenvalue equation
\[
  \Pi(\langle H\rangle)\Psi_i = m_i\,\Psi_i
\]
provides candidate fermion masses $m_i$; the $\Psi_i$ define mass
eigenstates.

In this view, mass is a property of how the vacuum sits inside
$H_3(\mathbb{O})$, not a ``gift'' handed out by an external scalar field.

\section*{Jordan potential and vacuum stability}

The vacuum $\langle H\rangle$ itself is determined by a Jordan-invariant
potential $V_J(H)$ on $H_3(\mathbb{O})$:
\[
  V_J: H_3(\mathbb{O}) \to \mathbb{R},\qquad
  H \mapsto V_J(H),
\]
invariant under an $F_4$ action on the Albert algebra. The true vacuum is a
minimum of this potential,
\[
  \frac{\partial V_J}{\partial H}\bigg|_{H=\langle H\rangle}=0,\qquad
  \frac{\partial^2 V_J}{\partial H^2}\bigg|_{H=\langle H\rangle} > 0
  \ \text{(in suitable directions)}.
\]

Fluctuations $\delta H$ around $\langle H\rangle$ see a quadratic
approximation
\[
  V_J(\langle H\rangle + \delta H)
  \approx V_J(\langle H\rangle)
        + \tfrac12\,\delta H \cdot
                     \mathcal{M}_J \cdot
                     \delta H
        + \cdots,
\]
where $\mathcal{M}_J$ is the Hessian (second derivative) at the minimum.

\section*{The Higgs as a curvature mode}

Among the many possible fluctuation directions $\delta H$ there is a
distinguished one that largely preserves the internal alignment of
$\langle H\rangle$ but changes its \emph{magnitude}. This direction
corresponds to the traditional electroweak order parameter and gives rise
to the observed Higgs resonance.

Its mass is set by a particular eigenvalue of the Hessian:
\[
  m_H^2 \sim
  \frac{\partial^2 V_J}{\partial h^2}\bigg|_{h=0},
\]
where $h$ parametrises the relevant fluctuation mode. The 125 GeV resonance
is therefore a curvature property of $V_J$ at $\langle H\rangle$, not the
mechanism that created the fermion masses in the first place.

\section*{Decoupling the narrative: cause vs.\ symptom}

This leads to a clean conceptual split:

\begin{itemize}
  \item \textbf{Cause of masses:}
    the spectrum of $\Pi(\langle H\rangle)$, i.e.\ how the vacuum embeds
    into $H_3(\mathbb{O})$ and how this embedding acts on internal states.
  \item \textbf{Symptom of the vacuum:}
    the Higgs resonance as one particular fluctuation of $\langle H\rangle$
    encoded in the curvature of $V_J$.
\end{itemize}

Experimentally we discovered the symptom first and back-inferred the
presence of a nontrivial vacuum. The octonionic/Jordan model reverses the
logical order: it starts from a structured vacuum in an exceptional algebra,
derives the mass map and only then identifies the corresponding resonance.

\section*{What remains of the Standard Model picture}

The familiar elements of the Standard Model story are not thrown away;
they are reinterpreted:

\begin{itemize}
  \item There is still an order parameter playing the role of the Higgs
        vacuum expectation value.
  \item Gauge boson masses still depend on this order parameter through
        couplings to internal directions.
  \item The 125 GeV scalar resonance still appears as a fluctuation of this
        order parameter.
\end{itemize}

What changes is the underlying language:

\begin{itemize}
  \item Instead of a fundamental scalar field added by hand, we have a
        Jordan element $\langle H\rangle$ in $H_3(\mathbb{O})$.
  \item Instead of ad-hoc Yukawa couplings, we have a linear mass map
        $\Pi(\langle H\rangle)$ whose structure is fixed by exceptional
        symmetry.
\end{itemize}

\section*{Why this matters for the bigger picture}

The Higgs day clarifies two broader points in the Advent story:

\begin{enumerate}
  \item It shows how a cornerstone of the Standard Model (the Higgs
        mechanism) can be embedded into a more rigid algebraic framework
        without losing contact with experiment.
  \item It supports the general thesis that many ``fundamental fields''
        are better viewed as collective modes of an exceptional vacuum,
        determined by internal geometry rather than arbitrary Lagrangian
        terms.
\end{enumerate}

If future measurements further constrain Higgs couplings and self-interactions,
they will test not only the Standard Model but also any candidate for the
underlying exceptional vacuum structure that produces the observed 125 GeV
mode.

\small
\begin{thebibliography}{9}

\bibitem{Higgs1964}
P.~W.~Higgs,
\newblock ``Broken symmetries and the masses of gauge bosons,''
\newblock {\em Phys.\ Rev.\ Lett.} \textbf{13}, 508--509 (1964).

\bibitem{JordanNeumannWigner1934}
P.~Jordan, J.~von~Neumann and E.~Wigner,
\newblock ``On an algebraic generalization of the quantum mechanical
formalism,''
\newblock {\em Ann.\ Math.} \textbf{35}, 29--64 (1934).

\bibitem{Internal}
[Internal notes on $V_J(H)$, mass maps and Higgs curvature:
{\tt unified-agebra.tex; chap11\_neu.tex; appM\_neu.tex}.]

\end{thebibliography}
\normalsize

  } % end #6 body
  {The Higgs is not the cause of mass but a resonance of an exceptional
   vacuum: a single curvature mode of a Jordan potential on $H_3(\mathbb{O})$.} % #7 Closing

\end{document}