% advent20.tex
% December 20, 2025 – Dark matter as a shadow sector of compressors

\documentclass[a4paper,10pt]{article}

\usepackage[utf8]{inputenc}
\usepackage[T1]{fontenc}
\usepackage[english]{babel}
\usepackage{amsmath,amssymb,amsfonts}
\usepackage{xcolor}
\usepackage{tikz}
\usetikzlibrary{calc}
\usepackage[framemethod=TikZ]{mdframed}
\usepackage{titlesec}
\usepackage{lettrine}
\usepackage{eso-pic}
\usepackage{geometry}
\usepackage{multicol}
\usepackage{lmodern}

\geometry{margin=2.0cm}

% advent-layout.tex (corrected for \input usage)

\definecolor{adventred}{HTML}{B3001B}
\definecolor{adventblue}{HTML}{003366}
\definecolor{adventgreen}{HTML}{006633}
\definecolor{adventgold}{HTML}{B59410}

\pagestyle{empty}

\titleformat{\section}
  {\normalfont\large\bfseries\color{adventblue}}{\thesection}{0em}{}
  
\titleformat{\subsection}
  {\normalfont\normalsize\bfseries\color{adventblue}}{\thesubsection}{0em}{}

%----
% AdventFrameTop
%----

\newenvironment{AdventFrameTop}
{%
  \begin{mdframed}[
    linecolor=adventgreen!0,
    linewidth=0pt,
    roundcorner=0pt,
    innertopmargin=10pt,
    innerbottommargin=10pt,
    innerleftmargin=10pt,
    innerrightmargin=10pt,
    backgroundcolor=adventgreen!2
  ]%
}
{%
  \end{mdframed}
}

\newcommand{\BeginAdventPage}{}
\newcommand{\EndAdventPage}{}

%----
% AdventTitleBlock
%----

\newcommand{\AdventTitleBlock}[4]{%
  \begin{center}
    {\Large\textcolor{adventred}{\textbf{#1}}}\par\vspace{4pt}%
    \ifx&#2&\else
      {\large\textbf{#2}}\par\vspace{2pt}%
    \fi
    {\Large\textcolor{adventblue}{\textbf{#3}}}\par
    \ifx&#4&\else
      \vspace{2pt}%
      {\normalsize\textbf{#4}}\par%
    \fi
  \end{center}%
}

%----
% AdventKeyInsight
%----

\newcommand{\AdventKeyInsight}[1]{%
  \vspace{0.5em}%
  \noindent\colorbox{adventred!8}{%
    \parbox{\dimexpr\linewidth-2\fboxsep}{%
    \textbf{\textcolor{adventred}{Key Insight.}}~#1%
    }%
  }%
  \vspace{0.5em}%
}

%----
% AdventStarRule
%----

\newcommand{\AdventStarRule}{%
  \vspace{0.3em}%
  \begin{center}
    {\color{adventgold}%
    \rule[0.5ex]{0.25\linewidth}{0.4pt}\;
    $\ast\;\ast\;\ast$\;
    \rule[0.5ex]{0.25\linewidth}{0.4pt}%
    }%
  \end{center}
  \vspace{0.3em}%
}

%----
% AdventClosing
%----

\newcommand{\AdventClosing}[1]{%
  \vspace{0.4em}%
  \begin{center}
    \textcolor{adventgreen}{\emph{#1}}%
  \end{center}
}

%----
% AdventAuthor
%----

\newcommand{\AdventAuthor}{%
  \AddToShipoutPictureFG{%
    \begin{tikzpicture}[remember picture,overlay]
    \node[anchor=south, yshift=2mm] at (current page.south) {%
    \footnotesize Andreas Müller, Kempten University of Applied Sciences, %
    \texttt{andreas.mueller@hs-kempten.de}%
    };
    \end{tikzpicture}%
  }%
}

%----
% AdventInitial
%----

\newcommand{\AdventInitial}[2]{%
  \lettrine[lines=2,lhang=0.1,loversize=0.15]%
    {\textcolor{adventred}{#1}}%
    {#2}%
}

%----
% AdventPageBackground
%----

\newcommand{\AdventPageBackground}{%
  \AddToShipoutPictureBG{%
    \begin{tikzpicture}[remember picture,overlay]
    \draw[adventgreen!80!black, line width=3pt, rounded corners=12pt]
    ($(current page.north west)+(0.8cm,-0.8cm)$)
    rectangle
    ($(current page.south east)+(-0.8cm,0.8cm)$);
    \fill[adventgold]
    ($(current page.north west)+(1.0cm,-1.0cm)$) circle (1.2pt)
    ($(current page.north east)+(-1.0cm,-1.0cm)$) circle (1.2pt)
    ($(current page.south west)+(1.0cm,1.0cm)$) circle (1.2pt)
    ($(current page.south east)+(-1.0cm,1.0cm)$) circle (1.2pt);
    \end{tikzpicture}
  }%
}

%---- AdventSheet macro (1-page, single column) ----
% #1: Date + occasion
% #2: unused
% #3: Main title
% #4: Subtitle
% #5: Key Insight
% #6: Main body content
% #7: Closing statement
\newcommand{\AdventSheet}[7]{%
  \BeginAdventPage
  \vspace*{1cm}
  \begin{AdventFrameTop}
    \AdventTitleBlock{#1}{#2}{#3}{#4}
    \AdventKeyInsight{#5}
  \end{AdventFrameTop}
  \AdventStarRule
  #6%
  \EndAdventPage
  \AdventClosing{#7}%
}

%---- AdventSheetTwoCol macro (two-column hero sheet) ----
% #1: Date + occasion
% #2: unused
% #3: Main title
% #4: Subtitle
% #5: Key Insight
% #6: Main body content (wrapped in multicols{2})
% #7: Closing statement
\newcommand{\AdventSheetTwoCol}[7]{%
  \BeginAdventPage
  \begin{AdventFrameTop}
    \AdventTitleBlock{#1}{#2}{#3}{#4}
    \AdventKeyInsight{#5}
  \end{AdventFrameTop}
  \AdventStarRule
  \begin{multicols}{2}
    #6%
  \end{multicols}
  \EndAdventPage
  \AdventClosing{#7}%
}

\begin{document}

\AdventPageBackground
\AdventAuthor

\AdventSheetTwoCol
  {December 20, 2025} % #1 Date
  {}                  % #2 unused
  {Dark matter: a shadow of the compressors?} % #3 Main title
  {Invisible eigenvalues in an exceptional corner} % #4 Subtitle
  {If masses and mixings of visible matter arise from the spectra of
   symmetric compressors derived from $\langle H\rangle\in H_3(\mathbb{O})$,
   it is natural to ask whether the same operator family could also
   generate a dark sector. The octonionic model suggests a speculative but
   structurally simple answer: dark matter corresponds to eigenvalues of
   \emph{sterile} compressors—modes that couple only gravitationally,
   because their eigenvectors sit in a corner of the Albert algebra that
   is decoupled from the rotor network.} % #5 Key Insight
  { % #6 body (two columns)

\section*{Visible matter from compressors}

\AdventInitial{B}{y} now, the compressor picture of visible matter is
familiar:

\begin{itemize}
  \item A vacuum configuration $\langle H\rangle$ in the Albert algebra
        $H_3(\mathbb{O})$ defines a mass map $\Pi(\langle H\rangle)$.
  \item Its eigenvalues give prototype fermion masses,
        $m_i \sim \mathrm{eig}_i\bigl(\Pi(\langle H\rangle)\bigr)$.
  \item Misalignment between different compressors yields flavour mixing
        (CKM, PMNS).
\end{itemize}

The crucial point is that all of this uses the \emph{same} operator
toolbox: symmetric compressors built from $\langle H\rangle$, living on the
same internal octonionic stage that also hosts rotors.

\section*{Where is the dark sector?}

Astronomy and cosmology tell us that there is about five times more dark
matter than ordinary baryonic matter. In the octonionic setting, the most
economical question is not ``What exotic field should we add?'', but:

\begin{quote}
  Which parts of the existing compressor structure remain invisible to
  rotors and Standard-Model charges?
\end{quote}

The Albert algebra $H_3(\mathbb{O})$ is large enough to contain subspaces
that do not talk directly to the visible sector. Geometrically, one can
distinguish:

\begin{itemize}
  \item \textbf{bright directions:} eigenvectors that carry nontrivial
        rotor charges and thus couple to known forces;
  \item \textbf{shadow directions:} eigenvectors that are neutral under all
        rotors associated with $SU(3)\times SU(2)\times U(1)$, but still
        present in the compressor spectra.
\end{itemize}

Eigenvalues along these shadow directions contribute mass, but not
electromagnetic or color charge. They are prime candidates for a dark
sector.

\section*{Sterile compressors}

Formally, a \emph{sterile compressor} $C_{\text{sterile}}$ is a symmetric
operator on the internal space with the following properties:

\begin{itemize}
  \item Its eigenvectors are eigenstates of all relevant rotors with
        \emph{trivial} eigenvalues (no charge, no weak isospin, no color).
  \item It appears in the same algebraic representation as the visible
        compressors $C_{\text{vis}}$, but in a different block that does
        not mix with them at tree level.
\end{itemize}

We can symbolically write the mass eigenvalues as
\[
  m_{\text{DM}} \;\sim\; \mathrm{eig}\bigl(C_{\text{sterile}}\bigr),\qquad
  m_{\text{vis}} \;\sim\; \mathrm{eig}\bigl(C_{\text{vis}}\bigr),
\]
with
\[
  [C_{\text{sterile}},G_a] \approx 0
  \quad\text{for all visible rotors }G_a,
\]
while $[C_{\text{vis}},G_a]$ is generically nonzero.

Such a sterile compressor creates mass without visible force couplings:
its eigenmodes are massive but dark.

\section*{Mass ranges and stability}

The octonionic model itself does not yet predict an absolute dark-matter
mass scale; that would require a detailed choice of $\langle H\rangle$ and
its embedding in $H_3(\mathbb{O})$. But structurally, it constrains \emph{how}
dark matter can look:

\begin{itemize}
  \item Dark modes share the same Jordan invariants as visible ones; they
        are not arbitrary fields but live in the same exceptional algebra.
  \item Stability is natural: if shadow eigenvectors do not couple to
        rotors, there are no fast decay channels into visible particles.
  \item Interactions among dark modes can still exist via their own rotor
        network in the sterile block, potentially leading to self-interacting
        dark matter scenarios.
\end{itemize}

Cosmologically, such a sector would gravitate like ordinary matter, but be
invisible to electromagnetic and strong probes—precisely what astronomical
data suggest.

\section*{Why this is not just ``add a scalar''}

Many dark-matter models add new fields or symmetries by hand. The
compressor picture is conceptually different:

\begin{itemize}
  \item No new algebraic ingredient is introduced beyond the existing
        exceptional structure.
  \item Dark matter arises as an \emph{unused corner} of the same mass map
        that generates visible fermion spectra.
  \item The number of dark degrees of freedom, their charges (or lack
        thereof) and their possible self-interactions are constrained by
        the representation theory of $H_3(\mathbb{O})$ and $F_4$.
\end{itemize}

In that sense, the model does not explain ``why there is dark matter'' from
first principles, but it offers a natural place for it to live—with much
less arbitrariness than a generic beyond-the-Standard-Model scenario.

\section*{What could falsify this picture?}

A compressor-based dark sector makes several expectations that can, in
principle, be tested:

\begin{itemize}
  \item \textbf{Gravitational only (to first approximation):}
        direct electromagnetic signatures (e.g.\ electric charge of dark
        matter) should be absent or extremely suppressed.
  \item \textbf{Structured mass spectrum:}
        the dark sector should not be a completely flat continuum, but
        exhibit a discrete spectrum or at least preferred mass scales,
        echoing the structure of visible masses.
  \item \textbf{Possible self-interactions:}
        the same formalism that builds rotors for the visible sector can
        build a sterile rotor network; this opens the door to dark
        self-interaction signals (e.g.\ in halo shapes or cluster mergers).
\end{itemize}

If future observations conclusively ruled out any such patterns—e.g.\ if
dark matter were shown to be almost featureless, perfectly cold and
non-interacting even with itself—the compressor-shadow picture would lose
much of its appeal.

\section*{A structural, not final, proposal}

This Advent sheet is deliberately cautious:

\begin{itemize}
  \item It does not claim to have \emph{the} dark-matter candidate, with a
        precise mass and cross section ready for exclusion plots.
  \item It does claim that, once you take the octonionic/Albert structure
        seriously, a dark sector is almost unavoidable: there are simply
        too many directions that do not couple to visible rotors.
\end{itemize}

Dark matter thus appears less as an alien addition and more as a
structural shadow of the same exceptional geometry that organises quarks
and leptons. The same operator toolbox—rotors and compressors on
$H_3(\mathbb{O})$—carries both visible and dark sectors; the difference
lies in which eigenvectors we are able to illuminate.

\small
\begin{thebibliography}{9}

\bibitem{Zwicky1933}
F.~Zwicky,
\newblock ``Die Rotverschiebung von extragalaktischen Nebeln,''
\newblock {\em Helv.\ Phys.\ Acta} \textbf{6}, 110–127 (1933).

\bibitem{Rubin1970}
V.~Rubin and W.~K.~Ford Jr.,
\newblock ``Rotation of the Andromeda nebula from a spectroscopic survey of
emission regions,''
\newblock {\em Astrophys.\ J.} \textbf{159}, 379 (1970).

\bibitem{Chap14}
[Internal notes on compressor sectors and dark modes, see {\tt chap14\_neu.tex;
chap20\_neu.tex}.]

\end{thebibliography}
\normalsize

  } % end #6 body
  {If masses come from compressors, dark matter is naturally a shadow:
   eigenvalues of sterile compressors that live in an exceptional corner
   of the same Albert algebra as visible matter.} % #7 Closing

\end{document}