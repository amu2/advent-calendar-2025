% advent01.tex
% December 1, 2025 – Quaternions as prototype for SU(2) and the weak force

\documentclass[a4paper,10pt]{article}

\usepackage[utf8]{inputenc}
\usepackage[T1]{fontenc}
\usepackage[english]{babel}
\usepackage{amsmath,amssymb,amsfonts}
\usepackage{xcolor}
\usepackage{tikz}
\usetikzlibrary{calc}
\usepackage[framemethod=TikZ]{mdframed}
\usepackage{titlesec}
\usepackage{lettrine}
\usepackage{eso-pic}
\usepackage{geometry}
\usepackage{multicol}
\usepackage{lmodern}

\geometry{margin=2.0cm}

% advent-layout.tex (corrected for \input usage)

\definecolor{adventred}{HTML}{B3001B}
\definecolor{adventblue}{HTML}{003366}
\definecolor{adventgreen}{HTML}{006633}
\definecolor{adventgold}{HTML}{B59410}

\pagestyle{empty}

\titleformat{\section}
  {\normalfont\large\bfseries\color{adventblue}}{\thesection}{0em}{}
  
\titleformat{\subsection}
  {\normalfont\normalsize\bfseries\color{adventblue}}{\thesubsection}{0em}{}

%----
% AdventFrameTop
%----

\newenvironment{AdventFrameTop}
{%
  \begin{mdframed}[
    linecolor=adventgreen!0,
    linewidth=0pt,
    roundcorner=0pt,
    innertopmargin=10pt,
    innerbottommargin=10pt,
    innerleftmargin=10pt,
    innerrightmargin=10pt,
    backgroundcolor=adventgreen!2
  ]%
}
{%
  \end{mdframed}
}

\newcommand{\BeginAdventPage}{}
\newcommand{\EndAdventPage}{}

%----
% AdventTitleBlock
%----

\newcommand{\AdventTitleBlock}[4]{%
  \begin{center}
    {\Large\textcolor{adventred}{\textbf{#1}}}\par\vspace{4pt}%
    \ifx&#2&\else
      {\large\textbf{#2}}\par\vspace{2pt}%
    \fi
    {\Large\textcolor{adventblue}{\textbf{#3}}}\par
    \ifx&#4&\else
      \vspace{2pt}%
      {\normalsize\textbf{#4}}\par%
    \fi
  \end{center}%
}

%----
% AdventKeyInsight
%----

\newcommand{\AdventKeyInsight}[1]{%
  \vspace{0.5em}%
  \noindent\colorbox{adventred!8}{%
    \parbox{\dimexpr\linewidth-2\fboxsep}{%
    \textbf{\textcolor{adventred}{Key Insight.}}~#1%
    }%
  }%
  \vspace{0.5em}%
}

%----
% AdventStarRule
%----

\newcommand{\AdventStarRule}{%
  \vspace{0.3em}%
  \begin{center}
    {\color{adventgold}%
    \rule[0.5ex]{0.25\linewidth}{0.4pt}\;
    $\ast\;\ast\;\ast$\;
    \rule[0.5ex]{0.25\linewidth}{0.4pt}%
    }%
  \end{center}
  \vspace{0.3em}%
}

%----
% AdventClosing
%----

\newcommand{\AdventClosing}[1]{%
  \vspace{0.4em}%
  \begin{center}
    \textcolor{adventgreen}{\emph{#1}}%
  \end{center}
}

%----
% AdventAuthor
%----

\newcommand{\AdventAuthor}{%
  \AddToShipoutPictureFG{%
    \begin{tikzpicture}[remember picture,overlay]
    \node[anchor=south, yshift=2mm] at (current page.south) {%
    \footnotesize Andreas Müller, Kempten University of Applied Sciences, %
    \texttt{andreas.mueller@hs-kempten.de}%
    };
    \end{tikzpicture}%
  }%
}

%----
% AdventInitial
%----

\newcommand{\AdventInitial}[2]{%
  \lettrine[lines=2,lhang=0.1,loversize=0.15]%
    {\textcolor{adventred}{#1}}%
    {#2}%
}

%----
% AdventPageBackground
%----

\newcommand{\AdventPageBackground}{%
  \AddToShipoutPictureBG{%
    \begin{tikzpicture}[remember picture,overlay]
    \draw[adventgreen!80!black, line width=3pt, rounded corners=12pt]
    ($(current page.north west)+(0.8cm,-0.8cm)$)
    rectangle
    ($(current page.south east)+(-0.8cm,0.8cm)$);
    \fill[adventgold]
    ($(current page.north west)+(1.0cm,-1.0cm)$) circle (1.2pt)
    ($(current page.north east)+(-1.0cm,-1.0cm)$) circle (1.2pt)
    ($(current page.south west)+(1.0cm,1.0cm)$) circle (1.2pt)
    ($(current page.south east)+(-1.0cm,1.0cm)$) circle (1.2pt);
    \end{tikzpicture}
  }%
}

%---- AdventSheet macro (1-page, single column) ----
% #1: Date + occasion
% #2: unused
% #3: Main title
% #4: Subtitle
% #5: Key Insight
% #6: Main body content
% #7: Closing statement
\newcommand{\AdventSheet}[7]{%
  \BeginAdventPage
  \vspace*{1cm}
  \begin{AdventFrameTop}
    \AdventTitleBlock{#1}{#2}{#3}{#4}
    \AdventKeyInsight{#5}
  \end{AdventFrameTop}
  \AdventStarRule
  #6%
  \EndAdventPage
  \AdventClosing{#7}%
}

%---- AdventSheetTwoCol macro (two-column hero sheet) ----
% #1: Date + occasion
% #2: unused
% #3: Main title
% #4: Subtitle
% #5: Key Insight
% #6: Main body content (wrapped in multicols{2})
% #7: Closing statement
\newcommand{\AdventSheetTwoCol}[7]{%
  \BeginAdventPage
  \begin{AdventFrameTop}
    \AdventTitleBlock{#1}{#2}{#3}{#4}
    \AdventKeyInsight{#5}
  \end{AdventFrameTop}
  \AdventStarRule
  \begin{multicols}{2}
    #6%
  \end{multicols}
  \EndAdventPage
  \AdventClosing{#7}%
}

\begin{document}

\AdventPageBackground
\AdventAuthor

\AdventSheetTwoCol
  {December 1, 2025} % #1 Date
  {}    % #2 unused
  {Quaternions as a prototype: $SU(2)$ and the weak force} % #3 Main title
  {A four-dimensional rehearsal before the octonionic stage}    % #4 Subtitle
  {Quaternions $\mathbb{H}\cong\mathbb{R}^4$ provide a rigid
   $1\oplus3$ split into one scalar and three imaginary directions and
   realise the double-cover $SU(2)$ of spatial rotations. In de
   Casteljau's matrix picture, unit quaternions act as birotations in
   four dimensions. The weak isospin group is thus not an abstract
   label but the symmetry of a specific four-dimensional number system.
   This $1\oplus3$ pattern is the warm-up for the octonionic
   eight-dimensional stage of one generation.} % #5 Key Insight
  { % #6 body (two columns)

\section*{A gentle introduction: Why quaternions matter}

\AdventInitial{I}{magine} you want to describe the internal structure of
elementary particles—not where they are in space, but the hidden
properties that make a left-handed electron different from a
right-handed one, or a neutrino different from a quark. One natural
question is: what kind of ``number system'' could serve as the stage for
these internal degrees of freedom?

The usual real numbers $\mathbb{R}$ give us one dimension: a single
line. Complex numbers $\mathbb{C}$ give us two dimensions and have
proven essential in quantum mechanics. But what if we need more
dimensions, and what if we want something that behaves like
multiplication—where combining two elements gives another element of the
same type?

This is where \emph{quaternions} enter the story. Discovered in 1843 by
William Rowan Hamilton, quaternions extend complex numbers to four
dimensions. You can think of them as having one ``real'' part and three
``imaginary'' parts, usually called $\mathbf{i},\mathbf{j},\mathbf{k}$.
These three imaginary units look like coordinates in 3D space, but with
a special multiplication rule that makes them \emph{noncommutative}:
in general, $\mathbf{i}\mathbf{j}\neq\mathbf{j}\mathbf{i}$.

Formally,
\[
  \mathbb{H} = \{\, a + b\,\mathbf{i} + c\,\mathbf{j} + d\,\mathbf{k}
    \mid a,b,c,d\in\mathbb{R} \,\},
\]
with
\[
  \mathbf{i}^2=\mathbf{j}^2=\mathbf{k}^2 = \mathbf{ijk} = -1.
\]
The key structural feature is the rigid split
\[
  \mathbb{H} \;\cong\; \mathbb{R} \oplus \mathbb{R}^3,
\]
one scalar component and a three-dimensional imaginary part: a
$1\oplus3$ pattern.

\section*{De Casteljau's matrix view: quaternions as birotations}

A particularly clear picture, developed in detail by de Casteljau
\cite{deCasteljau1987}, is to represent quaternions as $4\times4$ real
matrices acting on a four-dimensional Euclidean space $E_4$. In this
language, a unit quaternion becomes a \emph{birotation}: a simultaneous
rotation in two orthogonal 2-planes of $E_4$.

Schematically, one can write a unit quaternion in a normal form where
its matrix representation looks like
\[
Q_N(\varphi) \;=\;
\rho\begin{pmatrix}
\cos\varphi & -\sin\varphi & 0 & 0 \\
\sin\varphi &  \cos\varphi & 0 & 0 \\
0           &  0           & \cos\varphi & -\sin\varphi \\
0           &  0           & \sin\varphi &  \cos\varphi
\end{pmatrix},
\]
up to a suitable choice of orthonormal basis. Geometrically:

\begin{itemize}
  \item The first $2\times2$ block rotates one plane in $E_4$ by angle
    $\varphi$.
  \item The second $2\times2$ block rotates an orthogonal plane by the
    \emph{same} angle.
\end{itemize}

The associated ``antiquaternion'' corresponds to a contra-rotation
where one of the planes is rotated in the opposite sense. De Casteljau
uses this birotation picture to make explicit the eigenvalues, the
determinant structure and the rare circumstances under which such
$4\times4$ unitary matrices commute.

For us, the important message is: quaternions are not just formal
symbols; they encode very concrete four-dimensional rotations with a
rigid internal structure.

\section*{$SU(2)$ and the $1\oplus3$ pattern}

From the quaternion point of view, the group $SU(2)$ of weak isospin is
nothing but the group of unit quaternions:
\[
  \{q\in\mathbb{H}\mid |q|=1\} \;\simeq\; SU(2).
\]
Left multiplication by a unit quaternion acts as an $SU(2)$ transformation
on the imaginary part $\mathbb{R}^3$, while the scalar part is left
invariant.

Physically, this means:

\begin{itemize}
  \item The scalar direction (the ``$1$'' in $1\oplus3$) behaves like an
    isospin singlet.
  \item The three imaginary directions (the ``$3$'') form a triplet under
    $SU(2)$, in perfect analogy with weak isospin triplets.
\end{itemize}

In other words, weak isospin is not an arbitrary label group we bolt
onto particles. It is built into the structure of a specific
four-dimensional number system and its matrix representation as
birotations.

\section*{From $\mathbb{H}$ to $\mathbb{O}$: doubling the rehearsal}

Why spend an entire Advent day on this four-dimensional rehearsal if our
main stage is the eight-dimensional octonionic world?

Because several patterns scale up almost literally:

\begin{itemize}
  \item The split $\mathbb{H} \cong 1\oplus3$ becomes, in the octonionic
    setting, a richer decomposition of $\mathbb{R}^8$ into blocks that
    can host one full generation of internal quantum numbers.
  \item The role of unit quaternions as birotations in $E_4$ is taken
    over by suitable $8\times8$ action matrices built from octonionic
    left/right multiplication on $\mathbb{R}^8$.
  \item The place of $SU(2)\subset\mathrm{Aut}(\mathbb{H})$ is taken
    over by the exceptional group $G_2\subset\mathrm{Aut}(\mathbb{O})$,
    and the Spin(8) triality structure introduced on First Advent
    Sunday.
\end{itemize}

Seen from this angle, moving from quaternions to octonions is not a
wild leap into a bizarre algebra. It is the next and final step in a
sequence of division algebras: $\mathbb{R}$, $\mathbb{C}$,
$\mathbb{H}$, $\mathbb{O}$—each adding just enough structure to host a
richer symmetry.

\section*{Why this day matters in the Advent story}

The quaternion day serves as a conceptual bridge between two worlds:

\begin{enumerate}
  \item It anchors the exotic ideas of octonions, $G_2$ and triality in
    a familiar setting: $SU(2)$, spin, weak isospin and four-dimensional
    rotations.
  \item It shows that the algebraic backbone we propose for one
    generation (octonions and Albert algebra) is a natural extension of
    the quaternionic picture, not an unrelated construction.
  \item It prepares us to read later operator identities and action
    matrices as geometric statements about rotations, not as ad hoc
    matrix tricks.
\end{enumerate}

After this gentle quaternion warm-up, the following sheets return to the
full octonionic stage, now with a clearer intuition of where $SU(2)$ and
its $1\oplus3$ pattern are coming from in the algebraic background.

\small
\begin{thebibliography}{9}

\bibitem{deCasteljau1987}
P.~de Casteljau,
\newblock {\em Les Quaternions},
\newblock Paris: Hermès, 1987.

\bibitem{Hamilton1844}
W.~R.~Hamilton,
\newblock ``On quaternions; or on a new system of imaginaries in algebra,''
\newblock {\em Philos.\ Mag.} \textbf{25}, 489--495 (1844).

\end{thebibliography}
\normalsize

  } % end #6 body
  {Quaternions provide a four-dimensional rehearsal: a rigid $1\oplus3$
   pattern and $SU(2)$ as the symmetry of a concrete number system,
   anticipating the octonionic stage of one full generation.} % #7 Closing

\end{document}