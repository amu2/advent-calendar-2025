% advent12.tex
% Hero day: December 12, 2025 – Three generations from triality

\documentclass[a4paper,10pt]{article}

\usepackage[utf8]{inputenc}
\usepackage[T1]{fontenc}
\usepackage[english]{babel}
\usepackage{amsmath,amssymb,amsfonts}
\usepackage{xcolor}
\usepackage{tikz}
\usetikzlibrary{calc}
\usepackage[framemethod=TikZ]{mdframed}
\usepackage{titlesec}
\usepackage{lettrine}
\usepackage{eso-pic}
\usepackage{geometry}
\usepackage{multicol}
\usepackage{lmodern}

\geometry{margin=2.0cm}

% advent-layout.tex (corrected for \input usage)

\definecolor{adventred}{HTML}{B3001B}
\definecolor{adventblue}{HTML}{003366}
\definecolor{adventgreen}{HTML}{006633}
\definecolor{adventgold}{HTML}{B59410}

\pagestyle{empty}

\titleformat{\section}
  {\normalfont\large\bfseries\color{adventblue}}{\thesection}{0em}{}
  
\titleformat{\subsection}
  {\normalfont\normalsize\bfseries\color{adventblue}}{\thesubsection}{0em}{}

%----
% AdventFrameTop
%----

\newenvironment{AdventFrameTop}
{%
  \begin{mdframed}[
    linecolor=adventgreen!0,
    linewidth=0pt,
    roundcorner=0pt,
    innertopmargin=10pt,
    innerbottommargin=10pt,
    innerleftmargin=10pt,
    innerrightmargin=10pt,
    backgroundcolor=adventgreen!2
  ]%
}
{%
  \end{mdframed}
}

\newcommand{\BeginAdventPage}{}
\newcommand{\EndAdventPage}{}

%----
% AdventTitleBlock
%----

\newcommand{\AdventTitleBlock}[4]{%
  \begin{center}
    {\Large\textcolor{adventred}{\textbf{#1}}}\par\vspace{4pt}%
    \ifx&#2&\else
      {\large\textbf{#2}}\par\vspace{2pt}%
    \fi
    {\Large\textcolor{adventblue}{\textbf{#3}}}\par
    \ifx&#4&\else
      \vspace{2pt}%
      {\normalsize\textbf{#4}}\par%
    \fi
  \end{center}%
}

%----
% AdventKeyInsight
%----

\newcommand{\AdventKeyInsight}[1]{%
  \vspace{0.5em}%
  \noindent\colorbox{adventred!8}{%
    \parbox{\dimexpr\linewidth-2\fboxsep}{%
    \textbf{\textcolor{adventred}{Key Insight.}}~#1%
    }%
  }%
  \vspace{0.5em}%
}

%----
% AdventStarRule
%----

\newcommand{\AdventStarRule}{%
  \vspace{0.3em}%
  \begin{center}
    {\color{adventgold}%
    \rule[0.5ex]{0.25\linewidth}{0.4pt}\;
    $\ast\;\ast\;\ast$\;
    \rule[0.5ex]{0.25\linewidth}{0.4pt}%
    }%
  \end{center}
  \vspace{0.3em}%
}

%----
% AdventClosing
%----

\newcommand{\AdventClosing}[1]{%
  \vspace{0.4em}%
  \begin{center}
    \textcolor{adventgreen}{\emph{#1}}%
  \end{center}
}

%----
% AdventAuthor
%----

\newcommand{\AdventAuthor}{%
  \AddToShipoutPictureFG{%
    \begin{tikzpicture}[remember picture,overlay]
    \node[anchor=south, yshift=2mm] at (current page.south) {%
    \footnotesize Andreas Müller, Kempten University of Applied Sciences, %
    \texttt{andreas.mueller@hs-kempten.de}%
    };
    \end{tikzpicture}%
  }%
}

%----
% AdventInitial
%----

\newcommand{\AdventInitial}[2]{%
  \lettrine[lines=2,lhang=0.1,loversize=0.15]%
    {\textcolor{adventred}{#1}}%
    {#2}%
}

%----
% AdventPageBackground
%----

\newcommand{\AdventPageBackground}{%
  \AddToShipoutPictureBG{%
    \begin{tikzpicture}[remember picture,overlay]
    \draw[adventgreen!80!black, line width=3pt, rounded corners=12pt]
    ($(current page.north west)+(0.8cm,-0.8cm)$)
    rectangle
    ($(current page.south east)+(-0.8cm,0.8cm)$);
    \fill[adventgold]
    ($(current page.north west)+(1.0cm,-1.0cm)$) circle (1.2pt)
    ($(current page.north east)+(-1.0cm,-1.0cm)$) circle (1.2pt)
    ($(current page.south west)+(1.0cm,1.0cm)$) circle (1.2pt)
    ($(current page.south east)+(-1.0cm,1.0cm)$) circle (1.2pt);
    \end{tikzpicture}
  }%
}

%---- AdventSheet macro (1-page, single column) ----
% #1: Date + occasion
% #2: unused
% #3: Main title
% #4: Subtitle
% #5: Key Insight
% #6: Main body content
% #7: Closing statement
\newcommand{\AdventSheet}[7]{%
  \BeginAdventPage
  \vspace*{1cm}
  \begin{AdventFrameTop}
    \AdventTitleBlock{#1}{#2}{#3}{#4}
    \AdventKeyInsight{#5}
  \end{AdventFrameTop}
  \AdventStarRule
  #6%
  \EndAdventPage
  \AdventClosing{#7}%
}

%---- AdventSheetTwoCol macro (two-column hero sheet) ----
% #1: Date + occasion
% #2: unused
% #3: Main title
% #4: Subtitle
% #5: Key Insight
% #6: Main body content (wrapped in multicols{2})
% #7: Closing statement
\newcommand{\AdventSheetTwoCol}[7]{%
  \BeginAdventPage
  \begin{AdventFrameTop}
    \AdventTitleBlock{#1}{#2}{#3}{#4}
    \AdventKeyInsight{#5}
  \end{AdventFrameTop}
  \AdventStarRule
  \begin{multicols}{2}
    #6%
  \end{multicols}
  \EndAdventPage
  \AdventClosing{#7}%
}

\begin{document}

\AdventPageBackground
\AdventAuthor

\AdventSheetTwoCol
  {December 12, 2025} % #1 Date + occasion
  {}                                             % #2 unused
  {Three Generations from Triality}              % #3 Main title
  {One internal block, three consistent readings} % #4 Subtitle
  {The octonionic model does not postulate three generations by hand.
   Instead, it starts from a single eight-dimensional internal block with
   Spin(8) triality. The three triality-related irreducible
   representations --- vector, left-handed spinor, right-handed spinor ---
   are read as three coherent ways of organizing the same internal data.
   When this structure is embedded into the Albert algebra, it naturally
   unfolds into three fermion generations with correlated mass and mixing
   patterns.}                                    % #5 Key Insight
  { % #6 main body (two columns)

\section*{The puzzle of three generations}

\AdventInitial{T}{he} Standard Model contains three generations of quarks and
leptons. They have identical gauge quantum numbers but very different masses
and mixings. From the perspective of the usual gauge-group story this is a
mystery: why not one generation, or five, or an arbitrary number?

Most approaches simply accept three generations as an experimental fact and
add family indices. The octonionic model takes a different route. It asks
whether the number three could arise from the internal representation theory
of the exceptional structures that underlie the model.

The key player is the triality of $\mathrm{Spin}(8)$ and its embedding into
the Albert algebra $H_3(\mathbb{O})$.

\section*{Spin(8) and triality}

The group $\mathrm{Spin}(8)$, the double cover of $\mathrm{SO}(8)$, has an
exceptional property: it possesses three inequivalent eight-dimensional
irreducible representations,
\[
  V_8,\qquad S_8^+,\qquad S_8^-,
\]
usually called the vector, left-handed spinor, and right-handed spinor
representations. An outer automorphism of $\mathrm{Spin}(8)$ permutes these
three representations. This is the triality symmetry.

In abstract group theory, triality is often presented as a curiosity of
Dynkin type $D_4$. In the present model it is taken seriously as structural
input:

\begin{itemize}
  \item We start from \emph{one} internal eight-dimensional block, not three
        unrelated ones.
  \item We insist that this block can be read in three coherent ways:
        as $V_8$, as $S_8^+$, and as $S_8^-$.
  \item The consistency of these three readings constrains how internal
        operators can act.
\end{itemize}

This ``one block, three readings'' principle will later be matched to the
three observed fermion generations.

\section*{From octonions to the Albert algebra}

The internal degrees of freedom of the model are organised using the Albert
algebra $H_3(\mathbb{O})$, the Jordan algebra of $3\times 3$ hermitian
octonionic matrices. Its automorphism group is the exceptional Lie group
$F_4$:
\[
  F_4 \;=\; \mathrm{Aut}\bigl(H_3(\mathbb{O})\bigr).
\]

Within $H_3(\mathbb{O})$, each diagonal entry can be associated with an
octonionic ``slot'' hosting an internal Spin(8) structure. Roughly
speaking:

\begin{itemize}
  \item One diagonal octonion slot is associated with the vector
        representation.
  \item The second diagonal slot carries a left-handed spinor structure.
  \item The third diagonal slot carries a right-handed spinor structure.
\end{itemize}

Triality then manifests itself as a structured permutation of these roles,
implemented by elements of $F_4$ that reshuffle the internal directions in a
controlled way.

\section*{Reading generations from triality sectors}

When we couple the internal $H_3(\mathbb{O})$ structure to the spacetime
Dirac operator and the mass map $\Pi(H)$, each triality sector produces a
family of fermionic modes. Schematicly:

\begin{itemize}
  \item The \emph{vector-like} reading of the internal block organises one
        set of states with a characteristic mass pattern.
  \item The \emph{left-handed spinor} reading gives rise to a second set
        of states, with masses related but not identical to the first set.
  \item The \emph{right-handed spinor} reading yields a third set, again
        correlated but distinct.
\end{itemize}

These three correlated sets are identified with the three fermion
generations. The crucial point is not that there are exactly three
representations, but that they are related by an \emph{outer} automorphism:
they are three faces of one internal object, not three arbitrary copies.

\section*{Constraints on mass and mixing patterns}

Because the three generations arise from a single internal block with
triality symmetry, their mass and mixing parameters cannot be chosen
independently. Several qualitative features follow:

\begin{enumerate}
  \item \textbf{Hierarchy:} The eigenvalues of the mass map $\Pi(H)$, when
        restricted to the three triality-related sectors, typically split
        into bands with a built-in hierarchy. This echoes the observed
        pattern of light, medium, and heavy generations.

  \item \textbf{Mixing structure:} The allowed off-diagonal couplings
        between triality sectors are constrained by the Jordan structure of
        $H_3(\mathbb{O})$ and the embedding of the Standard Model gauge
        group. This shapes the form of the CKM and PMNS matrices.

  \item \textbf{Stability:} Because the three generations share a common
        internal origin, small deformations of the vacuum configuration
        $\langle H\rangle$ tend to move all three in a correlated way,
        rather than producing arbitrary new families.
\end{enumerate}

The goal is not to ``explain every digit'' of the mass spectrum, but to
show that three generations with a hierarchical and mixing-rich structure
are the \emph{natural} outcome of the exceptional geometry.

\section*{Comparison with ad hoc family replication}

In more conventional settings, one starts from a gauge group and then
adds three copies of the fermion content:
\[
  \psi \;\longrightarrow\; (\psi^{(1)},\psi^{(2)},\psi^{(3)}),
\]
with a family index labelling generations. This move works phenomenologically,
but it does not tell us why there are three copies, nor why their masses
are ordered the way we see them.

The octonionic approach reverses the logic:

\begin{itemize}
  \item Start from a single octonionic/Spin(8) block with triality.
  \item Embed it into an exceptional algebra ($H_3(\mathbb{O})$ with $F_4$
        symmetry).
  \item Let the internal mass map and vacuum pick out three correlated
        triality sectors.
\end{itemize}

Family replication is not assumed; it is read off from the structure of
the internal algebra.

\small
\begin{thebibliography}{9}

\bibitem{Baez2002}
J.~C.~Baez,
\newblock ``The octonions,''
\newblock {\em Bull.\ Amer.\ Math.\ Soc.} \textbf{39}, 145--205 (2002).

\bibitem{GurseyTze1996}
F.~Gürsey and H.~C.~Tze,
\newblock {\em On the Role of Division, Jordan and Related Algebras in Particle
Physics},
\newblock World Scientific, 1996.

\bibitem{Ramond2010}
P.~Ramond,
\newblock {\em Group Theory: A Physicist's Survey},
\newblock Cambridge University Press, 2010.

\end{thebibliography}
\normalsize

  } % end #6 body
  {Three generations appear as three triality-related faces of a single exceptional internal block, not as three arbitrary copies.} % #7 Closing

\end{document}