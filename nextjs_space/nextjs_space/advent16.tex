% advent16.tex
% December 16, 2025 – Gravitation: the missing corner of the model

\documentclass[a4paper,10pt]{article}

\usepackage[utf8]{inputenc}
\usepackage[T1]{fontenc}
\usepackage[english]{babel}
\usepackage{amsmath,amssymb,amsfonts}
\usepackage{xcolor}
\usepackage{tikz}
\usetikzlibrary{calc}
\usepackage[framemethod=TikZ]{mdframed}
\usepackage{titlesec}
\usepackage{lettrine}
\usepackage{eso-pic}
\usepackage{geometry}
\usepackage{multicol}
\usepackage{lmodern}

\geometry{margin=2.0cm}

% advent-layout.tex (corrected for \input usage)

\definecolor{adventred}{HTML}{B3001B}
\definecolor{adventblue}{HTML}{003366}
\definecolor{adventgreen}{HTML}{006633}
\definecolor{adventgold}{HTML}{B59410}

\pagestyle{empty}

\titleformat{\section}
  {\normalfont\large\bfseries\color{adventblue}}{\thesection}{0em}{}
  
\titleformat{\subsection}
  {\normalfont\normalsize\bfseries\color{adventblue}}{\thesubsection}{0em}{}

%----
% AdventFrameTop
%----

\newenvironment{AdventFrameTop}
{%
  \begin{mdframed}[
    linecolor=adventgreen!0,
    linewidth=0pt,
    roundcorner=0pt,
    innertopmargin=10pt,
    innerbottommargin=10pt,
    innerleftmargin=10pt,
    innerrightmargin=10pt,
    backgroundcolor=adventgreen!2
  ]%
}
{%
  \end{mdframed}
}

\newcommand{\BeginAdventPage}{}
\newcommand{\EndAdventPage}{}

%----
% AdventTitleBlock
%----

\newcommand{\AdventTitleBlock}[4]{%
  \begin{center}
    {\Large\textcolor{adventred}{\textbf{#1}}}\par\vspace{4pt}%
    \ifx&#2&\else
      {\large\textbf{#2}}\par\vspace{2pt}%
    \fi
    {\Large\textcolor{adventblue}{\textbf{#3}}}\par
    \ifx&#4&\else
      \vspace{2pt}%
      {\normalsize\textbf{#4}}\par%
    \fi
  \end{center}%
}

%----
% AdventKeyInsight
%----

\newcommand{\AdventKeyInsight}[1]{%
  \vspace{0.5em}%
  \noindent\colorbox{adventred!8}{%
    \parbox{\dimexpr\linewidth-2\fboxsep}{%
    \textbf{\textcolor{adventred}{Key Insight.}}~#1%
    }%
  }%
  \vspace{0.5em}%
}

%----
% AdventStarRule
%----

\newcommand{\AdventStarRule}{%
  \vspace{0.3em}%
  \begin{center}
    {\color{adventgold}%
    \rule[0.5ex]{0.25\linewidth}{0.4pt}\;
    $\ast\;\ast\;\ast$\;
    \rule[0.5ex]{0.25\linewidth}{0.4pt}%
    }%
  \end{center}
  \vspace{0.3em}%
}

%----
% AdventClosing
%----

\newcommand{\AdventClosing}[1]{%
  \vspace{0.4em}%
  \begin{center}
    \textcolor{adventgreen}{\emph{#1}}%
  \end{center}
}

%----
% AdventAuthor
%----

\newcommand{\AdventAuthor}{%
  \AddToShipoutPictureFG{%
    \begin{tikzpicture}[remember picture,overlay]
    \node[anchor=south, yshift=2mm] at (current page.south) {%
    \footnotesize Andreas Müller, Kempten University of Applied Sciences, %
    \texttt{andreas.mueller@hs-kempten.de}%
    };
    \end{tikzpicture}%
  }%
}

%----
% AdventInitial
%----

\newcommand{\AdventInitial}[2]{%
  \lettrine[lines=2,lhang=0.1,loversize=0.15]%
    {\textcolor{adventred}{#1}}%
    {#2}%
}

%----
% AdventPageBackground
%----

\newcommand{\AdventPageBackground}{%
  \AddToShipoutPictureBG{%
    \begin{tikzpicture}[remember picture,overlay]
    \draw[adventgreen!80!black, line width=3pt, rounded corners=12pt]
    ($(current page.north west)+(0.8cm,-0.8cm)$)
    rectangle
    ($(current page.south east)+(-0.8cm,0.8cm)$);
    \fill[adventgold]
    ($(current page.north west)+(1.0cm,-1.0cm)$) circle (1.2pt)
    ($(current page.north east)+(-1.0cm,-1.0cm)$) circle (1.2pt)
    ($(current page.south west)+(1.0cm,1.0cm)$) circle (1.2pt)
    ($(current page.south east)+(-1.0cm,1.0cm)$) circle (1.2pt);
    \end{tikzpicture}
  }%
}

%---- AdventSheet macro (1-page, single column) ----
% #1: Date + occasion
% #2: unused
% #3: Main title
% #4: Subtitle
% #5: Key Insight
% #6: Main body content
% #7: Closing statement
\newcommand{\AdventSheet}[7]{%
  \BeginAdventPage
  \vspace*{1cm}
  \begin{AdventFrameTop}
    \AdventTitleBlock{#1}{#2}{#3}{#4}
    \AdventKeyInsight{#5}
  \end{AdventFrameTop}
  \AdventStarRule
  #6%
  \EndAdventPage
  \AdventClosing{#7}%
}

%---- AdventSheetTwoCol macro (two-column hero sheet) ----
% #1: Date + occasion
% #2: unused
% #3: Main title
% #4: Subtitle
% #5: Key Insight
% #6: Main body content (wrapped in multicols{2})
% #7: Closing statement
\newcommand{\AdventSheetTwoCol}[7]{%
  \BeginAdventPage
  \begin{AdventFrameTop}
    \AdventTitleBlock{#1}{#2}{#3}{#4}
    \AdventKeyInsight{#5}
  \end{AdventFrameTop}
  \AdventStarRule
  \begin{multicols}{2}
    #6%
  \end{multicols}
  \EndAdventPage
  \AdventClosing{#7}%
}

\begin{document}

\AdventPageBackground
\AdventAuthor

\AdventSheetTwoCol
  {December 16, 2025} % #1 Date
  {}                  % #2 unused
  {Gravity: the missing corner of the model} % #3 Main title
  {Where the octonionic story honestly stops (for now)} % #4 Subtitle
  {The octonionic/Albert framework goes remarkably far: it organises
   internal quantum numbers, masses, mixings, couplings and scales in a
   single operator toolbox. But one major block remains incomplete:
   gravitation. The proton--Planck mass ratio
   $$\kappa = \frac{m_p}{m_P}, \qquad
     m_P = \sqrt{\frac{\hbar c}{G}}$$
   still enters as an external constant. This is not swept under the rug:
   the model marks gravity as an open assignment for future spectral
   geometry and for a unified
master action $S[D,\Psi]$ in which gravity and the octonionic internal
sector are treated on the same footing, not as a solved problem.} % #5 Key Insight
  { % #6 body (two columns)

\section*{How much is already covered}

\AdventInitial{B}{efore} talking about what is missing, it is worth
recalling what is already under octonionic control:

\begin{itemize}
  \item Internal quantum numbers are organised on an exceptional stage
        ($\mathbb{O}$, $H_3(\mathbb{O})$, $G_2$, $F_4$).
  \item Couplings and mixing angles appear as norms and angles of rotor
        commutators.
  \item Fermion masses and flavour structure arise as spectra of
        compressors derived from a vacuum $\langle H\rangle$.
  \item Characteristic energy scales (Planck, electroweak, QCD) are linked
        to invariants of the radius operator and attractor mechanisms.
\end{itemize}

Within this internal sector, the story is fairly coherent: many quantities
that used to be independent inputs become different faces of the same
exceptional geometry.

\section*{Enter gravity: a separate constant}

Gravitation is encoded in general relativity by Newton's constant $G$, or,
more conveniently, by the Planck mass
\[
  m_P = \sqrt{\frac{\hbar c}{G}}.
\]
The proton mass $m_p$ then defines a dimensionless ratio
\[
  \kappa = \frac{m_p}{m_P},
\]
numerically of order $10^{-19}$.

In the present octonionic framework, $m_p$ is accessible in principle via
the compressor structure (it is one of the low-lying eigenvalues of the
mass map), but $m_P$ is not yet derived: it is imported from gravitational
physics. Consequently, $\kappa$ appears as an external constant rather than
an internal invariant.

\section*{Why this is not a minor detail}

At first glance, an extra constant might look harmless. But conceptually it
matters:

\begin{itemize}
  \item Internally, the model claims a structural explanation of many
        numbers (charges, couplings, fermion masses).
  \item Externally, gravity fixes the overall scale at which quantum
        fields backreact on spacetime.
\end{itemize}

Without a link between the internal exceptional geometry and spacetime
geometry, the story is incomplete: we know how fields behave on a given
background, but we do not yet explain how that background itself emerges
from the same algebraic structure.

\section*{Spectral geometry as the natural bridge}

There is, however, a natural candidate framework to close this gap:
\emph{spectral geometry}. Instead of describing spacetime by a metric
$g_{\mu\nu}$, one describes it by a Dirac operator $D$ on a suitable
Hilbert space. The spectral action principle then postulates an action
\[
  S_{\text{spec}} = \mathrm{Tr}\,f(D^2/\Lambda^2),
\]
whose heat-kernel expansion generates:

\begin{itemize}
  \item the Einstein--Hilbert term (gravity),
  \item a cosmological constant term,
  \item gauge and matter kinetic terms,
  \item and higher-order corrections.
\end{itemize}

In the octonionic context, the hope is clear:

\begin{quote}
  Build a Dirac operator $D$ that encodes both spacetime geometry and the
  internal octonionic/Albert structure, then read gravitational couplings
  and the scale $\Lambda$ from its spectrum.
\end{quote}

In the language of this project, such a Dirac operator $D$ is not an
add-on but the same master operator that already organises the internal
rotors, compressors and mass maps. Later in the calendar, this idea
will be made explicit in the form of a single master action $S[D,\Psi]$
that packages internal and gravitational dynamics into one object.
Here, we only flag this direction; the actual construction is deferred.

This would turn $G$, $\kappa$ and even aspects of dark energy into
spectral invariants of the same operator that already knows about internal
rotors and compressors.

\section*{What is missing right now}

The present model does \emph{not} yet provide a final construction of such
a Dirac operator. Concretely:

\begin{itemize}
  \item The exact Hilbert space on which $D$ acts, combining spacetime
        spinors with octonionic internal data, has not been fixed.
  \item The interplay between nonassociative internal multiplication and
        the standard differential-geometric structure of $D$ is not yet
        fully understood.
  \item The computation that would derive $\kappa$ from the spectrum of
        $D$ is not available; $\kappa$ is treated as known input.
\end{itemize}

This is why gravity is called the ``missing corner'': it is not absent
because it does not fit, but because the necessary spectral-geometric
machinery has not yet been fully built.

\section*{Why it is still worth taking the model seriously}

Admitting an open gravitational block is not a weakness of honesty; it is
a strength of design:

\begin{enumerate}
  \item It clearly marks where the current octonionic explanations end and
        where new ideas are needed.
  \item It suggests a concrete research programme rather than a vague hope:
        construct an octonionic spectral triple and compute the resulting
        gravitational sector.
  \item It prevents overclaiming: the model explains many things about
        internal structure, but it does not pretend to have solved quantum
        gravity.
\end{enumerate}

Seen this way, the gravitational day of the Advent calendar is less a
conclusion than a pointer: it ties the exceptional internal geometry to
the next natural step in modern mathematical physics.

\section*{A clear boundary, not a blind spot}

The main message of this sheet is thus intentionally simple:

\begin{quote}
  The octonionic model currently stops at the boundary of gravity. The
  ratio $\kappa = m_p/m_P$ is acknowledged as an external input, and a
  spectral-gravity extension is needed to pull it inside the algebraic
  story, eventually in the form of a unified action $S[D,\Psi]$ whose
gravitational coefficients are fixed by the same spectral data as the
internal ones.
\end{quote}

The rest of the calendar—especially the spectral-geometry and AQFT days—
shows that this next step is plausible, but it is not taken yet. One
hundred years after Heisenberg, it is intellectually healthier to draw a
sharp line between what has been structurally organised and what remains
to be done.

\small
\begin{thebibliography}{9}

\bibitem{Einstein1915}
A.~Einstein,
\newblock ``Die Feldgleichungen der Gravitation,''
\newblock {\em Sitzungsber.\ Preuss.\ Akad.\ Wiss.} (1915), 844--847.

\bibitem{Connes1994}
A.~Connes,
\newblock {\em Noncommutative Geometry},
\newblock Academic Press, 1994.

\bibitem{ConnesChamseddine1997}
A.~Connes and A.~H.~Chamseddine,
\newblock ``The spectral action principle,''
\newblock {\em Commun.\ Math.\ Phys.} \textbf{186}, 731--750 (1997).

\bibitem{Internal}
[Internal notes on constants and gravity: {\tt arxiv-const.tex;
appE\_neu.tex; chap20\_neu.tex}.]

\end{thebibliography}
\normalsize

  } % end #6 body
  {The octonionic model organises internal physics but leaves gravity as
   an honest open corner: the ratio $\kappa=m_p/m_P$ still waits for a
   spectral-geometry explanation.} % #7 Closing

\end{document}