% advent18.tex
% December 18, 2025 – Algebraic QFT: fields without diagram dogma

\documentclass[a4paper,10pt]{article}

\usepackage[utf8]{inputenc}
\usepackage[T1]{fontenc}
\usepackage[english]{babel}
\usepackage{amsmath,amssymb,amsfonts}
\usepackage{xcolor}
\usepackage{tikz}
\usetikzlibrary{calc}
\usepackage[framemethod=TikZ]{mdframed}
\usepackage{titlesec}
\usepackage{lettrine}
\usepackage{eso-pic}
\usepackage{geometry}
\usepackage{multicol}
\usepackage{lmodern}

\geometry{margin=2.0cm}

% advent-layout.tex (corrected for \input usage)

\definecolor{adventred}{HTML}{B3001B}
\definecolor{adventblue}{HTML}{003366}
\definecolor{adventgreen}{HTML}{006633}
\definecolor{adventgold}{HTML}{B59410}

\pagestyle{empty}

\titleformat{\section}
  {\normalfont\large\bfseries\color{adventblue}}{\thesection}{0em}{}
  
\titleformat{\subsection}
  {\normalfont\normalsize\bfseries\color{adventblue}}{\thesubsection}{0em}{}

%----
% AdventFrameTop
%----

\newenvironment{AdventFrameTop}
{%
  \begin{mdframed}[
    linecolor=adventgreen!0,
    linewidth=0pt,
    roundcorner=0pt,
    innertopmargin=10pt,
    innerbottommargin=10pt,
    innerleftmargin=10pt,
    innerrightmargin=10pt,
    backgroundcolor=adventgreen!2
  ]%
}
{%
  \end{mdframed}
}

\newcommand{\BeginAdventPage}{}
\newcommand{\EndAdventPage}{}

%----
% AdventTitleBlock
%----

\newcommand{\AdventTitleBlock}[4]{%
  \begin{center}
    {\Large\textcolor{adventred}{\textbf{#1}}}\par\vspace{4pt}%
    \ifx&#2&\else
      {\large\textbf{#2}}\par\vspace{2pt}%
    \fi
    {\Large\textcolor{adventblue}{\textbf{#3}}}\par
    \ifx&#4&\else
      \vspace{2pt}%
      {\normalsize\textbf{#4}}\par%
    \fi
  \end{center}%
}

%----
% AdventKeyInsight
%----

\newcommand{\AdventKeyInsight}[1]{%
  \vspace{0.5em}%
  \noindent\colorbox{adventred!8}{%
    \parbox{\dimexpr\linewidth-2\fboxsep}{%
    \textbf{\textcolor{adventred}{Key Insight.}}~#1%
    }%
  }%
  \vspace{0.5em}%
}

%----
% AdventStarRule
%----

\newcommand{\AdventStarRule}{%
  \vspace{0.3em}%
  \begin{center}
    {\color{adventgold}%
    \rule[0.5ex]{0.25\linewidth}{0.4pt}\;
    $\ast\;\ast\;\ast$\;
    \rule[0.5ex]{0.25\linewidth}{0.4pt}%
    }%
  \end{center}
  \vspace{0.3em}%
}

%----
% AdventClosing
%----

\newcommand{\AdventClosing}[1]{%
  \vspace{0.4em}%
  \begin{center}
    \textcolor{adventgreen}{\emph{#1}}%
  \end{center}
}

%----
% AdventAuthor
%----

\newcommand{\AdventAuthor}{%
  \AddToShipoutPictureFG{%
    \begin{tikzpicture}[remember picture,overlay]
    \node[anchor=south, yshift=2mm] at (current page.south) {%
    \footnotesize Andreas Müller, Kempten University of Applied Sciences, %
    \texttt{andreas.mueller@hs-kempten.de}%
    };
    \end{tikzpicture}%
  }%
}

%----
% AdventInitial
%----

\newcommand{\AdventInitial}[2]{%
  \lettrine[lines=2,lhang=0.1,loversize=0.15]%
    {\textcolor{adventred}{#1}}%
    {#2}%
}

%----
% AdventPageBackground
%----

\newcommand{\AdventPageBackground}{%
  \AddToShipoutPictureBG{%
    \begin{tikzpicture}[remember picture,overlay]
    \draw[adventgreen!80!black, line width=3pt, rounded corners=12pt]
    ($(current page.north west)+(0.8cm,-0.8cm)$)
    rectangle
    ($(current page.south east)+(-0.8cm,0.8cm)$);
    \fill[adventgold]
    ($(current page.north west)+(1.0cm,-1.0cm)$) circle (1.2pt)
    ($(current page.north east)+(-1.0cm,-1.0cm)$) circle (1.2pt)
    ($(current page.south west)+(1.0cm,1.0cm)$) circle (1.2pt)
    ($(current page.south east)+(-1.0cm,1.0cm)$) circle (1.2pt);
    \end{tikzpicture}
  }%
}

%---- AdventSheet macro (1-page, single column) ----
% #1: Date + occasion
% #2: unused
% #3: Main title
% #4: Subtitle
% #5: Key Insight
% #6: Main body content
% #7: Closing statement
\newcommand{\AdventSheet}[7]{%
  \BeginAdventPage
  \vspace*{1cm}
  \begin{AdventFrameTop}
    \AdventTitleBlock{#1}{#2}{#3}{#4}
    \AdventKeyInsight{#5}
  \end{AdventFrameTop}
  \AdventStarRule
  #6%
  \EndAdventPage
  \AdventClosing{#7}%
}

%---- AdventSheetTwoCol macro (two-column hero sheet) ----
% #1: Date + occasion
% #2: unused
% #3: Main title
% #4: Subtitle
% #5: Key Insight
% #6: Main body content (wrapped in multicols{2})
% #7: Closing statement
\newcommand{\AdventSheetTwoCol}[7]{%
  \BeginAdventPage
  \begin{AdventFrameTop}
    \AdventTitleBlock{#1}{#2}{#3}{#4}
    \AdventKeyInsight{#5}
  \end{AdventFrameTop}
  \AdventStarRule
  \begin{multicols}{2}
    #6%
  \end{multicols}
  \EndAdventPage
  \AdventClosing{#7}%
}

\begin{document}

\AdventPageBackground
\AdventAuthor

\AdventSheetTwoCol
  {December 18, 2025} % #1 Date
  {}                  % #2 unused
  {Algebraic QFT: beyond Feynman diagrams and Hilbert-space dogma} % #3 Main title
  {Local operator algebras on an octonionic stage}                  % #4 Subtitle
  {Most physicists meet quantum field theory (QFT) through path integrals
   and Feynman diagrams on a fixed Hilbert space. For an octonionic model
   with a rich internal operator toolbox, this language is too rigid.
   Algebraic QFT (AQFT) replaces fields by nets of local algebras
   $\mathcal{O}\mapsto\mathcal{A}(\mathcal{O})$ and treats states as
   positive linear functionals, not as privileged vectors. This is the
   natural QFT framework for a nonassociative internal geometry.} % #5 Key Insight
  { % #6 body (two columns)

\section*{From fields to nets of algebras}

\AdventInitial{I}{n} the textbook picture, a quantum field theory is given
by a Lagrangian, a path integral and a set of Feynman rules. Hilbert
space, canonical commutation relations and perturbative expansions are
baked in from the start.

Algebraic QFT (Haag--Kastler) takes a different route. The basic object is
a net of C$^*$- or von Neumann algebras
\[
  \mathcal{O} \longmapsto \mathcal{A}(\mathcal{O}),
\]
where $\mathcal{O}$ runs over open regions of spacetime. This net obeys:

\begin{itemize}
  \item \textbf{Isotony:} $\mathcal{O}_1\subset\mathcal{O}_2$
        $\Rightarrow$ $\mathcal{A}(\mathcal{O}_1)\subset\mathcal{A}(\mathcal{O}_2)$.
  \item \textbf{Locality:} spacelike separated regions commute,
        $[\mathcal{A}(\mathcal{O}_1),\mathcal{A}(\mathcal{O}_2)]=0$.
  \item \textbf{Covariance:} a symmetry group $G$ acts by automorphisms
        $\alpha_g$ with
        $\alpha_g(\mathcal{A}(\mathcal{O}))=\mathcal{A}(g\mathcal{O})$.
\end{itemize}

States are positive linear functionals
$\omega:\mathcal{A}\to\mathbb{C}$, not a priori elements of a given Fock
space. A Hilbert-space representation arises afterwards via the GNS
construction.

\section*{Why this fits the octonionic toolbox}

The octonionic model organises internal physics in terms of operators:

\begin{itemize}
  \item rotors (internal symmetry generators),
  \item compressors (mass/mixing operators),
  \item the radius operator and attractor structures,
  \item Jordan elements $H \in H_3(\mathbb{O})$ and their potentials.
\end{itemize}

These are naturally interpreted as elements of local algebras rather than
as components of a single canonical field on a fixed Hilbert space. AQFT
offers exactly the right viewpoint:

\begin{itemize}
  \item The internal nonassociative structure influences which operators
        can be multiplied and localised, but the algebras themselves
        remain associative operator algebras.
  \item Different vacuum choices (different $\omega$) lead to different
        Hilbert-space representations, without changing the underlying
        net $\mathcal{A}(\mathcal{O})$.
\end{itemize}

\section*{Octonionic local algebras}

In an octonionic context, one can sketch the following assignment:

\begin{itemize}
  \item To each spacetime region $\mathcal{O}$ we assign an algebra
        $\mathcal{A}(\mathcal{O})$ generated by:
        \begin{itemize}
          \item localised rotor operators (internal symmetries),
          \item compressor fields encoding mass/mixing effects,
          \item fluctuation modes of the Jordan element $H$.
        \end{itemize}
  \item These generators satisfy commutation relations constrained by the
        exceptional symmetry ($G_2$, $F_4$) and by locality.
\end{itemize}

The internal nonassociativity shows up in how these generators combine and
in the structure of their spectra, not in the C$^*$-algebra axioms
themselves (which remain associative).

\section*{AQFT vs.\ Feynman-diagram intuition}

The algebraic language does not forbid Feynman diagrams; it puts them in
perspective. In simple regimes and near-Gaussian states, one can still
expand correlation functions in diagrams. But:

\begin{itemize}
  \item The existence of a Fock space and a perturbative expansion is no
        longer an axiom, but a property of certain states.
  \item Nonperturbative aspects (phase structure, superselection sectors,
        thermal states) can be discussed directly in terms of algebras and
        states.
  \item The nonassociative internal geometry is built into the operator
        content, not into a particular path integral measure.
\end{itemize}

For an exceptional, nonassociative internal space, this is conceptually
safer than starting from a single classical Lagrangian and hoping that
perturbation theory can see everything.

\section*{Connection to spectral geometry and 21 December}

AQFT and spectral geometry address complementary aspects:

\begin{itemize}
  \item \textbf{Spectral geometry} (yesterday) tells us how to build an
        action from a Dirac spectrum that includes gravity and gauge
        fields.
  \item \textbf{AQFT} (today) tells us how to formulate the quantum theory
        of the resulting fields in a representation-free way.
\end{itemize}

On December 21, the fourth Advent Sunday, these threads meet: an
octonionic version of AQFT is combined with an $F_4$-symmetric Jordan
potential. The electroweak scale then appears as an equilibrium quantity
of the internal geometry, and the local algebras inherit this scale
through their operator content.

\section*{Why this day matters conceptually}

The AQFT day is less about new numbers and more about cleaning up the
foundations:

\begin{enumerate}
  \item It frees the octonionic model from the straightjacket of a single
        Fock space and a fixed perturbative expansion.
  \item It highlights that what matters fundamentally are \emph{operators}
        and their algebraic relations, not a particular graphical
        technique.
  \item It prepares the ground for rigorous discussions of locality,
        causality and phases in a world with an exceptional internal
        structure.
\end{enumerate}

One hundred years after Heisenberg's operator approach, AQFT can be read
as a modern continuation of his original spirit—now extended to an
octonionic internal world that he could not have imagined.

\small
\begin{thebibliography}{9}

\bibitem{HaagKastler1964}
R.~Haag and D.~Kastler,
\newblock ``An algebraic approach to quantum field theory,''
\newblock {\em J.\ Math.\ Phys.} \textbf{5}, 848--861 (1964).

\bibitem{Haag1996}
R.~Haag,
\newblock {\em Local Quantum Physics},
\newblock Springer, 1996.

\bibitem{Internal}
[Internal notes on AQFT formulation and octonionic local algebras:
{\tt chap18\_neu.tex; appF\_neu.tex}.]

\end{thebibliography}
\normalsize

  } % end #6 body
  {Algebraic QFT replaces Feynman-diagram dogma by nets of local algebras
   and states—exactly the language needed to house an octonionic internal
   geometry in a consistent quantum field theory.} % #7 Closing

\end{document}