% advent31.tex
% Hero day: December 31, 2025 – Farewell to the parameter zoo

\documentclass[a4paper,10pt]{article}

\usepackage[utf8]{inputenc}
\usepackage[T1]{fontenc}
\usepackage[english]{babel}
\usepackage{amsmath,amssymb,amsfonts}
\usepackage{xcolor}
\usepackage{tikz}
\usetikzlibrary{calc}
\usepackage[framemethod=TikZ]{mdframed}
\usepackage{titlesec}
\usepackage{lettrine}
\usepackage{eso-pic}
\usepackage{geometry}
\usepackage{multicol}
\usepackage{lmodern}

\geometry{margin=2.0cm}

% advent-layout.tex (corrected for \input usage)

\definecolor{adventred}{HTML}{B3001B}
\definecolor{adventblue}{HTML}{003366}
\definecolor{adventgreen}{HTML}{006633}
\definecolor{adventgold}{HTML}{B59410}

\pagestyle{empty}

\titleformat{\section}
  {\normalfont\large\bfseries\color{adventblue}}{\thesection}{0em}{}
  
\titleformat{\subsection}
  {\normalfont\normalsize\bfseries\color{adventblue}}{\thesubsection}{0em}{}

%----
% AdventFrameTop
%----

\newenvironment{AdventFrameTop}
{%
  \begin{mdframed}[
    linecolor=adventgreen!0,
    linewidth=0pt,
    roundcorner=0pt,
    innertopmargin=10pt,
    innerbottommargin=10pt,
    innerleftmargin=10pt,
    innerrightmargin=10pt,
    backgroundcolor=adventgreen!2
  ]%
}
{%
  \end{mdframed}
}

\newcommand{\BeginAdventPage}{}
\newcommand{\EndAdventPage}{}

%----
% AdventTitleBlock
%----

\newcommand{\AdventTitleBlock}[4]{%
  \begin{center}
    {\Large\textcolor{adventred}{\textbf{#1}}}\par\vspace{4pt}%
    \ifx&#2&\else
      {\large\textbf{#2}}\par\vspace{2pt}%
    \fi
    {\Large\textcolor{adventblue}{\textbf{#3}}}\par
    \ifx&#4&\else
      \vspace{2pt}%
      {\normalsize\textbf{#4}}\par%
    \fi
  \end{center}%
}

%----
% AdventKeyInsight
%----

\newcommand{\AdventKeyInsight}[1]{%
  \vspace{0.5em}%
  \noindent\colorbox{adventred!8}{%
    \parbox{\dimexpr\linewidth-2\fboxsep}{%
    \textbf{\textcolor{adventred}{Key Insight.}}~#1%
    }%
  }%
  \vspace{0.5em}%
}

%----
% AdventStarRule
%----

\newcommand{\AdventStarRule}{%
  \vspace{0.3em}%
  \begin{center}
    {\color{adventgold}%
    \rule[0.5ex]{0.25\linewidth}{0.4pt}\;
    $\ast\;\ast\;\ast$\;
    \rule[0.5ex]{0.25\linewidth}{0.4pt}%
    }%
  \end{center}
  \vspace{0.3em}%
}

%----
% AdventClosing
%----

\newcommand{\AdventClosing}[1]{%
  \vspace{0.4em}%
  \begin{center}
    \textcolor{adventgreen}{\emph{#1}}%
  \end{center}
}

%----
% AdventAuthor
%----

\newcommand{\AdventAuthor}{%
  \AddToShipoutPictureFG{%
    \begin{tikzpicture}[remember picture,overlay]
    \node[anchor=south, yshift=2mm] at (current page.south) {%
    \footnotesize Andreas Müller, Kempten University of Applied Sciences, %
    \texttt{andreas.mueller@hs-kempten.de}%
    };
    \end{tikzpicture}%
  }%
}

%----
% AdventInitial
%----

\newcommand{\AdventInitial}[2]{%
  \lettrine[lines=2,lhang=0.1,loversize=0.15]%
    {\textcolor{adventred}{#1}}%
    {#2}%
}

%----
% AdventPageBackground
%----

\newcommand{\AdventPageBackground}{%
  \AddToShipoutPictureBG{%
    \begin{tikzpicture}[remember picture,overlay]
    \draw[adventgreen!80!black, line width=3pt, rounded corners=12pt]
    ($(current page.north west)+(0.8cm,-0.8cm)$)
    rectangle
    ($(current page.south east)+(-0.8cm,0.8cm)$);
    \fill[adventgold]
    ($(current page.north west)+(1.0cm,-1.0cm)$) circle (1.2pt)
    ($(current page.north east)+(-1.0cm,-1.0cm)$) circle (1.2pt)
    ($(current page.south west)+(1.0cm,1.0cm)$) circle (1.2pt)
    ($(current page.south east)+(-1.0cm,1.0cm)$) circle (1.2pt);
    \end{tikzpicture}
  }%
}

%---- AdventSheet macro (1-page, single column) ----
% #1: Date + occasion
% #2: unused
% #3: Main title
% #4: Subtitle
% #5: Key Insight
% #6: Main body content
% #7: Closing statement
\newcommand{\AdventSheet}[7]{%
  \BeginAdventPage
  \vspace*{1cm}
  \begin{AdventFrameTop}
    \AdventTitleBlock{#1}{#2}{#3}{#4}
    \AdventKeyInsight{#5}
  \end{AdventFrameTop}
  \AdventStarRule
  #6%
  \EndAdventPage
  \AdventClosing{#7}%
}

%---- AdventSheetTwoCol macro (two-column hero sheet) ----
% #1: Date + occasion
% #2: unused
% #3: Main title
% #4: Subtitle
% #5: Key Insight
% #6: Main body content (wrapped in multicols{2})
% #7: Closing statement
\newcommand{\AdventSheetTwoCol}[7]{%
  \BeginAdventPage
  \begin{AdventFrameTop}
    \AdventTitleBlock{#1}{#2}{#3}{#4}
    \AdventKeyInsight{#5}
  \end{AdventFrameTop}
  \AdventStarRule
  \begin{multicols}{2}
    #6%
  \end{multicols}
  \EndAdventPage
  \AdventClosing{#7}%
}

\begin{document}

\AdventPageBackground
\AdventAuthor

\AdventSheetTwoCol
  {December 31, 2025} % #1 Date
  {}                 % #2 unused
  {Farewell to the parameter zoo: bilan 2025} % #3 Main title
  {100 years after Heisenberg: from matrices to octonions} % #4 Subtitle
  {Looking back, many seemingly independent parameters of particle physics
   (couplings, scales, mass ratios, mixing angles) can be traced back to a
   small set of octonionic and Albert-algebra structures: octaves,
   $G_2$ and $F_4$, heptagon and radius operators, rotors and compressors,
   and the vacuum configuration $\langle H\rangle$. One hundred years after
   Heisenberg's matrix mechanics, the algebraic heritage of quantum theory
   is alive and extended: from noncommutative matrices to nonassociative
   octonions. At the dynamical level, this heritage takes the form of a single master
action $S[D,\Psi]$ whose few coefficients and algebraic invariants
replace the many ad hoc parameters of the Standard Model.} % #5 Key Insight
  { % #6 body

\section*{From a zoo of inputs to a structured atlas}

\AdventInitial{T}{he} Standard Model appears to need dozens of free
parameters: gauge couplings, Yukawas, mixing angles, mass ratios, vacuum
expectation values, hierarchy scales. Over the course of this Advent
calendar, many of them have been reinterpreted as:

\begin{itemize}
  \item invariants of the $F_4$ symmetry atlas on $H_3(\mathbb{O})$,
  \item eigenvalues of internal operators ($H_7$, $R$, compressors),
  \item equilibrium values of internal potentials (e.g.\ $Y_S$),
  \item spectral data of a Dirac operator on a structured vacuum.
\end{itemize}

What looked like a parameter zoo is largely a shadow of one coherent
geometric structure.

\section*{The algebraic backbone}

The backbone of the model consists of:

\begin{itemize}
  \item \textbf{Number system:} octonions $\mathbb{O}$ and their
        automorphism group $G_2$.
  \item \textbf{Symmetry atlas:} the Albert algebra $H_3(\mathbb{O})$ and
        its automorphism group $F_4$.
  \item \textbf{Operators:} heptagon operator $H_7$, radius operator $R$,
        rotors and compressors.
  \item \textbf{Vacuum:} a distinguished configuration $\langle H\rangle$
        and associated scales $(a_0,b_0,c_0)$, $Y_S$.
\end{itemize}

From this small list, one can systematically derive:

\begin{itemize}
  \item the existence of three generations (triality),
  \item the pattern of gauge couplings (via rotor norms),
  \item the presence of three main scales (via $\mathrm{spec}(R)$),
  \item the structure of mass matrices and mixings (via compressors and
        $\Pi(\langle H\rangle)$),
  \item candidates for dark sectors and vacuum energy contributions.
\end{itemize}

\section*{From many parameters to one action}

In a conventional presentation, the Standard Model Lagrangian carries a
long list of independent couplings, masses and mixing parameters. In the
octonionic formulation used in this project, the dynamics is organised
by a single master action
\[
  S[D,\Psi]
  = \int_{M_4} \sqrt{-g}\,\Big(
      \mathcal{L}_{\text{kin}}[D]
    + \mathcal{L}_F[D]
    + \mathcal{L}_G[D]
    + \mathcal{L}_{\text{Defekt}}[D]
    + \mathcal{L}_{\text{matter}}[D,\Psi]
  \Big),
\]
where $D$ encodes spin connection, gauge fields and compressor/Yukawa
blocks, and $\Psi$ collects three generations of fermions.

Most of the ``parameters'' then become:

\begin{itemize}
  \item discrete choices of vacuum data $(\langle H\rangle; a_0,b_0,c_0; Y_S)$,
  \item algebraic invariants of the $F_4$ symmetry atlas,
  \item and a small set of dimensionless coefficients in front of the
    operator blocks in $S[D,\Psi]$.
\end{itemize}

The parameter zoo is not abolished, but it is \emph{relocated}: from
many unrelated input numbers to a handful of structural choices in the
exceptional master action.

\section*{Open block: gravity}

Not everything is solved. A central open question remains:

\[
  \kappa = \frac{m_p}{m_P},
\]

the ratio of proton mass to Planck mass. Together with the cosmological
scale $\Lambda$, it marks the gravitational block that has not yet been
fully integrated into the octonionic sector.

The spectral-action approach suggests a path:

\[
  S_{\text{spec}} = \mathrm{Tr}\, f(D^2/\Lambda^2),
\]

where $D$ encodes both space-time and internal octonionic geometry. In a
future project, this may allow $\kappa$ and $\Lambda$ to be understood as
spectral invariants, closing the remaining gap —in other words, to derive the remaining gravitational coefficients in
$S[D,\Psi]$ from the same spectral data that govern the internal sector.

\section*{A century after Heisenberg}

In 1925, Heisenberg introduced matrix mechanics, marking the beginning of
quantum theory as an algebraic theory of observables. One hundred years
later, the present model can be read as a continuation of that line of
thought:

\begin{itemize}
  \item from matrices to nonassociative algebras,
  \item from $SU(2)$ to $G_2$ and $F_4$,
  \item from a list of parameters to a structured atlas of invariants.
\end{itemize}

The core lesson is not that octonions are ``exotic'', but that algebraic
structure remains a powerful guide for understanding physical reality.

\section*{From fitting to understanding}

The transition can be summarised in one sentence:

\begin{quote}
  We move from \emph{fitting} independent constants to \emph{understanding}
  them as coordinates on orbits in an exceptional symmetry atlas.
\end{quote}

This is not the final word—open questions remain, especially on gravity
and cosmology—but it marks a coherent step towards a less fragmented
picture of fundamental physics.

The reduction of the parameter zoo is thus conceptual and structural
rather than yet numerically complete: many patterns are explained, but
precise values and quantum corrections remain an open and testable part
of the programme.

\small
\begin{thebibliography}{9}

\bibitem{Heisenberg1925}
W.~Heisenberg,
\newblock ``\"Uber quantentheoretische Umdeutung kinematischer und
mechanischer Beziehungen,''
\newblock {\em Z.\ Phys.} \textbf{33}, 879--893 (1925).

\bibitem{Dirac1937}
P.~A.~M.~Dirac,
\newblock ``The cosmological constants,''
\newblock {\em Nature} \textbf{139}, 323 (1937).


\end{thebibliography}
\normalsize

  } % end #6 body
  {Many ``free parameters'' of particle physics become invariants of an
   exceptional symmetry atlas built from octonions and the Albert algebra.
   One hundred years after Heisenberg, the algebraic spirit of quantum
   mechanics extends from matrices to nonassociative structures, pointing
   towards a less fragmented understanding of constants and scales.} % #7 Closing

\end{document}