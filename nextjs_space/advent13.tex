% advent13.tex
% December 12, 2025 – Top quark: heaviest fermion as vacuum anchor

\documentclass[a4paper,10pt]{article}

\usepackage[utf8]{inputenc}
\usepackage[T1]{fontenc}
\usepackage[english]{babel}
\usepackage{amsmath,amssymb,amsfonts}
\usepackage{xcolor}
\usepackage{tikz}
\usetikzlibrary{calc}
\usepackage[framemethod=TikZ]{mdframed}
\usepackage{titlesec}
\usepackage{lettrine}
\usepackage{eso-pic}
\usepackage{geometry}
\usepackage{multicol}
\usepackage{lmodern}

\geometry{margin=2.0cm}

% advent-layout.tex (corrected for \input usage)

\definecolor{adventred}{HTML}{B3001B}
\definecolor{adventblue}{HTML}{003366}
\definecolor{adventgreen}{HTML}{006633}
\definecolor{adventgold}{HTML}{B59410}

\pagestyle{empty}

\titleformat{\section}
  {\normalfont\large\bfseries\color{adventblue}}{\thesection}{0em}{}
  
\titleformat{\subsection}
  {\normalfont\normalsize\bfseries\color{adventblue}}{\thesubsection}{0em}{}

%----
% AdventFrameTop
%----

\newenvironment{AdventFrameTop}
{%
  \begin{mdframed}[
    linecolor=adventgreen!0,
    linewidth=0pt,
    roundcorner=0pt,
    innertopmargin=10pt,
    innerbottommargin=10pt,
    innerleftmargin=10pt,
    innerrightmargin=10pt,
    backgroundcolor=adventgreen!2
  ]%
}
{%
  \end{mdframed}
}

\newcommand{\BeginAdventPage}{}
\newcommand{\EndAdventPage}{}

%----
% AdventTitleBlock
%----

\newcommand{\AdventTitleBlock}[4]{%
  \begin{center}
    {\Large\textcolor{adventred}{\textbf{#1}}}\par\vspace{4pt}%
    \ifx&#2&\else
      {\large\textbf{#2}}\par\vspace{2pt}%
    \fi
    {\Large\textcolor{adventblue}{\textbf{#3}}}\par
    \ifx&#4&\else
      \vspace{2pt}%
      {\normalsize\textbf{#4}}\par%
    \fi
  \end{center}%
}

%----
% AdventKeyInsight
%----

\newcommand{\AdventKeyInsight}[1]{%
  \vspace{0.5em}%
  \noindent\colorbox{adventred!8}{%
    \parbox{\dimexpr\linewidth-2\fboxsep}{%
    \textbf{\textcolor{adventred}{Key Insight.}}~#1%
    }%
  }%
  \vspace{0.5em}%
}

%----
% AdventStarRule
%----

\newcommand{\AdventStarRule}{%
  \vspace{0.3em}%
  \begin{center}
    {\color{adventgold}%
    \rule[0.5ex]{0.25\linewidth}{0.4pt}\;
    $\ast\;\ast\;\ast$\;
    \rule[0.5ex]{0.25\linewidth}{0.4pt}%
    }%
  \end{center}
  \vspace{0.3em}%
}

%----
% AdventClosing
%----

\newcommand{\AdventClosing}[1]{%
  \vspace{0.4em}%
  \begin{center}
    \textcolor{adventgreen}{\emph{#1}}%
  \end{center}
}

%----
% AdventAuthor
%----

\newcommand{\AdventAuthor}{%
  \AddToShipoutPictureFG{%
    \begin{tikzpicture}[remember picture,overlay]
    \node[anchor=south, yshift=2mm] at (current page.south) {%
    \footnotesize Andreas Müller, Kempten University of Applied Sciences, %
    \texttt{andreas.mueller@hs-kempten.de}%
    };
    \end{tikzpicture}%
  }%
}

%----
% AdventInitial
%----

\newcommand{\AdventInitial}[2]{%
  \lettrine[lines=2,lhang=0.1,loversize=0.15]%
    {\textcolor{adventred}{#1}}%
    {#2}%
}

%----
% AdventPageBackground
%----

\newcommand{\AdventPageBackground}{%
  \AddToShipoutPictureBG{%
    \begin{tikzpicture}[remember picture,overlay]
    \draw[adventgreen!80!black, line width=3pt, rounded corners=12pt]
    ($(current page.north west)+(0.8cm,-0.8cm)$)
    rectangle
    ($(current page.south east)+(-0.8cm,0.8cm)$);
    \fill[adventgold]
    ($(current page.north west)+(1.0cm,-1.0cm)$) circle (1.2pt)
    ($(current page.north east)+(-1.0cm,-1.0cm)$) circle (1.2pt)
    ($(current page.south west)+(1.0cm,1.0cm)$) circle (1.2pt)
    ($(current page.south east)+(-1.0cm,1.0cm)$) circle (1.2pt);
    \end{tikzpicture}
  }%
}

%---- AdventSheet macro (1-page, single column) ----
% #1: Date + occasion
% #2: unused
% #3: Main title
% #4: Subtitle
% #5: Key Insight
% #6: Main body content
% #7: Closing statement
\newcommand{\AdventSheet}[7]{%
  \BeginAdventPage
  \vspace*{1cm}
  \begin{AdventFrameTop}
    \AdventTitleBlock{#1}{#2}{#3}{#4}
    \AdventKeyInsight{#5}
  \end{AdventFrameTop}
  \AdventStarRule
  #6%
  \EndAdventPage
  \AdventClosing{#7}%
}

%---- AdventSheetTwoCol macro (two-column hero sheet) ----
% #1: Date + occasion
% #2: unused
% #3: Main title
% #4: Subtitle
% #5: Key Insight
% #6: Main body content (wrapped in multicols{2})
% #7: Closing statement
\newcommand{\AdventSheetTwoCol}[7]{%
  \BeginAdventPage
  \begin{AdventFrameTop}
    \AdventTitleBlock{#1}{#2}{#3}{#4}
    \AdventKeyInsight{#5}
  \end{AdventFrameTop}
  \AdventStarRule
  \begin{multicols}{2}
    #6%
  \end{multicols}
  \EndAdventPage
  \AdventClosing{#7}%
}

\begin{document}

\AdventPageBackground
\AdventAuthor

\AdventSheetTwoCol
  {December 13, 2025} % #1 Date
  {}                 % #2 unused
  {Top quark: heaviest fermion as vacuum anchor} % #3 Main title
  {Why $m_t \approx 173\,$GeV sits next to the electroweak scale} % #4 Subtitle
  {The top quark mass $m_t \approx 173\,$GeV is not an outlier but the
   largest eigenvalue of the mass map $\Pi(\langle H\rangle)$ associated
   with the vacuum configuration $\langle H\rangle$ in the Albert algebra
   $H_3(\mathbb{O})$. It acts as an \emph{anchor} for the electroweak
   scale: the same structure that fixes the vacuum also singles out the
   top as the heaviest fermion. Without such an anchor, the electroweak
   scale would be unstable.} % #5 Key Insight
  { % #6 body

\section*{Masses from the mass map $\Pi(\langle H\rangle)$}

\AdventInitial{I}{n} the octonionic model, fermion masses are not free
Yukawa coefficients but eigenvalues of a linear map

\[
  \Pi : H_3(\mathbb{O}) \longrightarrow \mathrm{End}(\mathbb{R}^8),
\]

evaluated at the vacuum configuration $\langle H\rangle \in H_3(\mathbb{O})$.
The physical mass matrix is schematically

\[
  M = y\,\Pi(\langle H\rangle),
\]

with $y$ a universal coupling. Its eigenvalues

\[
  m_i = \lambda_i\big(\Pi(\langle H\rangle)\big)
\]

are the fermion masses. Among these, the largest eigenvalue is identified
with the top quark mass $m_t$.

\section*{The top as the maximal eigenvalue}

The spectrum of $\Pi(\langle H\rangle)$ is strongly constrained:

\begin{itemize}
  \item $\langle H\rangle$ is not arbitrary; it minimises a Jordan
        potential $V_J(H)$ under symmetry constraints.
  \item $\Pi$ respects the $F_4$ symmetry of the Albert algebra.
  \item The resulting eigenvalues come in structured patterns rather than
        random numbers.
\end{itemize}

In this setting, the top quark mass appears as

\[
  m_t = \max \mathrm{spec}\big(\Pi(\langle H\rangle)\big),
\]

the largest eigenvalue in the relevant sector. This is not just a
numerical statement; it has a stability interpretation.

\section*{Vacuum anchoring at the electroweak scale}

The electroweak scale $Y_S$ is fixed by the minimum of an internal
potential. Schematically,

\[
  Y_S^2 = -\frac{\mu^2}{2(\lambda+\kappa c)},
\]

with $\mu^2,\lambda,\kappa,c$ determined by the symmetry atlas. The
vacuum configuration $\langle H\rangle$ sits at this minimum, and
$\Pi(\langle H\rangle)$ inherits that structure.

The largest eigenvalue $m_t$ then behaves as an \emph{anchor}:

\begin{itemize}
  \item If $m_t$ were much smaller, the curvature of the potential near
        the minimum would change, destabilising the electroweak scale.
  \item If $m_t$ were much larger, the same structure would shift the
        position of the minimum, again spoiling the observed scale.
  \item The observed value $m_t\approx 173\,$GeV lies naturally next to
        the electroweak scale, reflecting the shared origin in
        $\langle H\rangle$.
\end{itemize}

\section*{Top quark versus lighter fermions}

Lighter fermions (electron, muon, tau; light quarks) correspond to
smaller eigenvalues of the same map. Their smallness is explained by the
geometry of $\langle H\rangle$ in the symmetry atlas:

\begin{itemize}
  \item Some directions in $H_3(\mathbb{O})$ generate large eigenvalues,
        associated with heavy fermions.
  \item Other directions generate exponentially suppressed eigenvalues,
        associated with light fermions (as discussed for neutrinos on
        15 December).
\end{itemize}

The top quark is simply the fermion whose eigenvector aligns best with
the ``steep'' direction of the potential at the vacuum point.

\section*{Conceptual picture}

In the usual Standard Model narrative, Yukawa couplings are free
parameters, and the large top Yukawa is a brute fact. In the octonionic
picture:

\begin{enumerate}
  \item The vacuum configuration $\langle H\rangle$ is fixed by internal
    geometry and potential minimisation.
  \item The mass operator $\Pi(\langle H\rangle)$ is uniquely determined
    by this vacuum and the symmetry atlas.
  \item The top quark emerges as the strongest-coupled mode to this
    vacuum, i.e.\ the maximal eigenvalue.
\end{enumerate}

The hierarchy ``$m_t$ heavy, others light'' is no longer a collection of
independent choices but a single structural statement about one operator
evaluated at one point in the Albert algebra.

\small
\begin{thebibliography}{9}

\bibitem{CDFD02014}
CDF and D0 Collaborations,
\newblock ``Combination of CDF and D0 results on the mass of the top quark,''
\newblock {\em Phys.\ Rev.\ D} \textbf{89}, 072001 (2014).

\bibitem{JordanVNW1934}
P.~Jordan, J.~von Neumann and E.~Wigner,
\newblock ``On an algebraic generalization of the quantum mechanical
formalism,''
\newblock {\em Ann.\ Math.} \textbf{35}, 29--64 (1934).

\bibitem{Internal}
[Internal notes on mass maps and the role of the top quark:
{\tt unified-agebra.tex; chap10\_neu.tex; appM\_neu.tex}.]

\end{thebibliography}
\normalsize

  } % end #6 body
  {The top quark is not an accidental heavy outlier. It is the largest
   eigenvalue of the mass map $\Pi(\langle H\rangle)$ and thus an anchor
   of the electroweak vacuum. The same internal structure that fixes the
   electroweak scale also singles out the top as the heaviest fermion.} % #7 Closing
   
\end{document}