% advent23.tex
% December 23, 2025 – QFT foundations reloaded: what changes in an exceptional world?

\documentclass[a4paper,10pt]{article}

\usepackage[utf8]{inputenc}
\usepackage[T1]{fontenc}
\usepackage[english]{babel}
\usepackage{amsmath,amssymb,amsfonts}
\usepackage{xcolor}
\usepackage{tikz}
\usetikzlibrary{calc}
\usepackage[framemethod=TikZ]{mdframed}
\usepackage{titlesec}
\usepackage{lettrine}
\usepackage{eso-pic}
\usepackage{geometry}
\usepackage{multicol}
\usepackage{lmodern}

\geometry{margin=2.0cm}

% advent-layout.tex (corrected for \input usage)

\definecolor{adventred}{HTML}{B3001B}
\definecolor{adventblue}{HTML}{003366}
\definecolor{adventgreen}{HTML}{006633}
\definecolor{adventgold}{HTML}{B59410}

\pagestyle{empty}

\titleformat{\section}
  {\normalfont\large\bfseries\color{adventblue}}{\thesection}{0em}{}
  
\titleformat{\subsection}
  {\normalfont\normalsize\bfseries\color{adventblue}}{\thesubsection}{0em}{}

%----
% AdventFrameTop
%----

\newenvironment{AdventFrameTop}
{%
  \begin{mdframed}[
    linecolor=adventgreen!0,
    linewidth=0pt,
    roundcorner=0pt,
    innertopmargin=10pt,
    innerbottommargin=10pt,
    innerleftmargin=10pt,
    innerrightmargin=10pt,
    backgroundcolor=adventgreen!2
  ]%
}
{%
  \end{mdframed}
}

\newcommand{\BeginAdventPage}{}
\newcommand{\EndAdventPage}{}

%----
% AdventTitleBlock
%----

\newcommand{\AdventTitleBlock}[4]{%
  \begin{center}
    {\Large\textcolor{adventred}{\textbf{#1}}}\par\vspace{4pt}%
    \ifx&#2&\else
      {\large\textbf{#2}}\par\vspace{2pt}%
    \fi
    {\Large\textcolor{adventblue}{\textbf{#3}}}\par
    \ifx&#4&\else
      \vspace{2pt}%
      {\normalsize\textbf{#4}}\par%
    \fi
  \end{center}%
}

%----
% AdventKeyInsight
%----

\newcommand{\AdventKeyInsight}[1]{%
  \vspace{0.5em}%
  \noindent\colorbox{adventred!8}{%
    \parbox{\dimexpr\linewidth-2\fboxsep}{%
    \textbf{\textcolor{adventred}{Key Insight.}}~#1%
    }%
  }%
  \vspace{0.5em}%
}

%----
% AdventStarRule
%----

\newcommand{\AdventStarRule}{%
  \vspace{0.3em}%
  \begin{center}
    {\color{adventgold}%
    \rule[0.5ex]{0.25\linewidth}{0.4pt}\;
    $\ast\;\ast\;\ast$\;
    \rule[0.5ex]{0.25\linewidth}{0.4pt}%
    }%
  \end{center}
  \vspace{0.3em}%
}

%----
% AdventClosing
%----

\newcommand{\AdventClosing}[1]{%
  \vspace{0.4em}%
  \begin{center}
    \textcolor{adventgreen}{\emph{#1}}%
  \end{center}
}

%----
% AdventAuthor
%----

\newcommand{\AdventAuthor}{%
  \AddToShipoutPictureFG{%
    \begin{tikzpicture}[remember picture,overlay]
    \node[anchor=south, yshift=2mm] at (current page.south) {%
    \footnotesize Andreas Müller, Kempten University of Applied Sciences, %
    \texttt{andreas.mueller@hs-kempten.de}%
    };
    \end{tikzpicture}%
  }%
}

%----
% AdventInitial
%----

\newcommand{\AdventInitial}[2]{%
  \lettrine[lines=2,lhang=0.1,loversize=0.15]%
    {\textcolor{adventred}{#1}}%
    {#2}%
}

%----
% AdventPageBackground
%----

\newcommand{\AdventPageBackground}{%
  \AddToShipoutPictureBG{%
    \begin{tikzpicture}[remember picture,overlay]
    \draw[adventgreen!80!black, line width=3pt, rounded corners=12pt]
    ($(current page.north west)+(0.8cm,-0.8cm)$)
    rectangle
    ($(current page.south east)+(-0.8cm,0.8cm)$);
    \fill[adventgold]
    ($(current page.north west)+(1.0cm,-1.0cm)$) circle (1.2pt)
    ($(current page.north east)+(-1.0cm,-1.0cm)$) circle (1.2pt)
    ($(current page.south west)+(1.0cm,1.0cm)$) circle (1.2pt)
    ($(current page.south east)+(-1.0cm,1.0cm)$) circle (1.2pt);
    \end{tikzpicture}
  }%
}

%---- AdventSheet macro (1-page, single column) ----
% #1: Date + occasion
% #2: unused
% #3: Main title
% #4: Subtitle
% #5: Key Insight
% #6: Main body content
% #7: Closing statement
\newcommand{\AdventSheet}[7]{%
  \BeginAdventPage
  \vspace*{1cm}
  \begin{AdventFrameTop}
    \AdventTitleBlock{#1}{#2}{#3}{#4}
    \AdventKeyInsight{#5}
  \end{AdventFrameTop}
  \AdventStarRule
  #6%
  \EndAdventPage
  \AdventClosing{#7}%
}

%---- AdventSheetTwoCol macro (two-column hero sheet) ----
% #1: Date + occasion
% #2: unused
% #3: Main title
% #4: Subtitle
% #5: Key Insight
% #6: Main body content (wrapped in multicols{2})
% #7: Closing statement
\newcommand{\AdventSheetTwoCol}[7]{%
  \BeginAdventPage
  \begin{AdventFrameTop}
    \AdventTitleBlock{#1}{#2}{#3}{#4}
    \AdventKeyInsight{#5}
  \end{AdventFrameTop}
  \AdventStarRule
  \begin{multicols}{2}
    #6%
  \end{multicols}
  \EndAdventPage
  \AdventClosing{#7}%
}

\begin{document}

\AdventPageBackground
\AdventAuthor

\AdventSheetTwoCol
  {December 23, 2025} % #1 Date
  {}                  % #2 unused
  {QFT foundations reloaded in an exceptional world} % #3 Main title
  {What really changes – and what does not – with octonions, AQFT and spectra} % #4 Subtitle
  {By now the calendar has introduced octonionic internal geometry,
   rotor/compressor operators, spectral geometry and algebraic QFT. This
   day steps back and asks: what does all this \emph{really} change in
   quantum field theory? The answer is both modest and radical: the basic
   probabilistic structure of QFT remains, but the kinematical stage,
   the list of ``fundamental'' fields and the status of parameters are
   reshaped by exceptional algebra and operator language.} % #5 Key Insight
  { % #6 body (two columns)

\section*{What survives untouched}

\AdventInitial{D}{espite} octonions, Jordan algebras and spectral tricks,
several pillars of QFT remain in place:

\begin{itemize}
  \item Locality and causality are still enforced via commutation
        relations or, in AQFT language, via locality of the net
        $\mathcal{O}\mapsto\mathcal{A}(\mathcal{O})$.
  \item States are still positive linear functionals on an operator
        algebra; probabilities are still given by expectation values and
        Born's rule.
  \item Renormalisation, running couplings and effective-field-theory
        reasoning retain their conceptual role.
\end{itemize}

In that sense, the octonionic programme is not a revolution \emph{against}
QFT; it is a reorganisation \emph{within} its operator framework.

\section*{What changes in the kinematical stage}

The radical part lies in the choice of underlying structures:

\begin{itemize}
  \item The internal Hilbert space is no longer an arbitrary finite tensor
        product; it is tied to the representation theory of
        $\mathbb{O}$ and $H_3(\mathbb{O})$.
  \item Gauge groups are not postulated separately; they emerge from
        automorphisms of these exceptional structures.
  \item Mass and mixing operators (compressors, mass map) are not generic
        matrices but constrained by the geometry of the Albert algebra.
\end{itemize}

Instead of a long list of independent fields and representations, we start
from a \emph{single} exceptional stage and let its operator content
unfold. QFT then lives on this stage rather than on an arbitrary internal
tensor product.

\section*{From Lagrangians to operator toolboxes}

Traditional QFT is written in terms of Lagrangian densities:
\[
  \mathcal{L}
  = \mathcal{L}_{\text{kinetic}}
  + \mathcal{L}_{\text{gauge}}
  + \mathcal{L}_{\text{Yukawa}}
  + \cdots,
\]
with many a priori free couplings. In the octonionic/AQFT picture the
primary objects are:

\begin{itemize}
  \item local algebras $\mathcal{A}(\mathcal{O})$ containing rotors,
        compressors and $H$-fluctuation modes,
  \item a Dirac operator $D$ (spectral geometry) encoding kinetic and
        gauge structure,
  \item a Jordan potential $V_J(H)$ encoding vacuum structure.
\end{itemize}

A Lagrangian can still be written down as an effective description, but it
is now a \emph{derived summary} of spectral and algebraic data, not the
fundamental starting point. This flips the usual order:

\begin{quote}
  Operator geometry first, Feynman rules later.
\end{quote}

\section*{Fields vs.\ modes of an exceptional vacuum}

Another shift concerns what we call ``fundamental fields''. In the
Standard Model language:

\begin{itemize}
  \item gauge fields $A_\mu$,
  \item fermion fields $\psi$,
  \item the Higgs scalar $H$,
\end{itemize}
appear side by side as basic degrees of freedom.

In the exceptional picture:

\begin{itemize}
  \item Fermion multiplets arise as specific representations of
        $H_3(\mathbb{O})$ and its rotor algebra.
  \item Masses come from the spectrum of the mass map $\Pi(\langle H\rangle)$.
  \item The Higgs becomes a fluctuation mode of the vacuum element
        $\langle H\rangle$, not the origin of mass (11 December).
\end{itemize}

Gauge fields still appear as connections, but their group structure and
couplings are read from rotor geometry. The distinction between ``matter''
and ``mediator'' fields becomes partly a matter of which operators we
choose to treat as background (vacuum) and which as excitations.

\section*{Parameters vs.\ equilibrium data}

From 22 December we learned to reclassify constants:

\begin{itemize}
  \item Many couplings and mass scales become equilibrium values:
        minima of an $F_4$-symmetric potential or attractor radii.
  \item Angles like $\theta_W$ become literal angles between rotor
        directions.
  \item Only a small core of numbers (notably those tied to gravity)
        remain external in the current state of the model.
\end{itemize}

In QFT language this means: the space of possible Lagrangians is no longer
a huge free parameter space; it is the image of a much smaller space of
exceptional-geometric choices (vacuum embeddings, embeddings of the gauge
group, spectral scales).

\section*{AQFT and spectral geometry as consistency checks}

Finally, AQFT and spectral geometry serve as consistency filters:

\begin{itemize}
  \item \textbf{AQFT} asks whether the operator content can be organised
        into a local, covariant net of algebras with reasonable state
        structure (18 and 21 December).
  \item \textbf{Spectral geometry} asks whether there exists a Dirac
        operator whose spectral action reproduces both the internal
        dynamics and gravity in a controlled expansion (17 December).
\end{itemize}

Many aesthetically pleasing algebraic constructions fail one or both of
these tests. The exceptional model sketched in this calendar is designed
to at least plausibly pass them, even if the full computations are still
pending.

\section*{What this means for the next century of QFT}

Seen from the long Heisenberg-to-now perspective, the 23 December message
is:

\begin{enumerate}
  \item QFT is not in crisis; its operator foundation is robust enough to
        host even nonassociative internal geometries.
  \item The real opportunity lies in upgrading the internal stage from an
        ad-hoc product of vector spaces to a rigid exceptional algebra
        with built-in hierarchies and symmetry.
  \item Spectral geometry and AQFT are not fringe formalisms but natural
        languages for such a stage.
\end{enumerate}

If this programme succeeds, the textbooks of 2125 might still teach
Feynman diagrams and path integrals—but as calculational tools built on
top of a deeper story where octonions, spectra and local algebras quietly
decide which Lagrangians are even allowed.

\small
\begin{thebibliography}{9}

\bibitem{Haag1996}
R.~Haag,
\newblock {\em Local Quantum Physics},
\newblock Springer, 1996.

\bibitem{Connes1994}
A.~Connes,
\newblock {\em Noncommutative Geometry},
\newblock Academic Press, 1994.

\bibitem{Heisenberg1925}
W.~Heisenberg,
\newblock ``\"Uber quantentheoretische Umdeutung kinematischer und
mechanischer Beziehungen,''
\newblock {\em Z.\ Phys.} \textbf{33}, 879--893 (1925).

\bibitem{Internal}
[Internal notes on foundational aspects of the octonionic model:
{\tt chap18\_neu.tex; chap21\_neu.tex; appF\_neu.tex}.]

\end{thebibliography}
\normalsize

  } % end #6 body
  {The exceptional programme does not overthrow QFT; it rewrites its
   kinematical stage and parameter space in terms of octonionic geometry,
   spectra and local algebras—leaving probability intact but reshaping
   what counts as ``fundamental''.} % #7 Closing

\end{document}