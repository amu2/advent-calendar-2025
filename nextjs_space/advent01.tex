% advent01.tex
% December 1, 2025 – Quaternions as prototype for SU(2) and the weak force

\documentclass[a4paper,10pt]{article}

\usepackage[utf8]{inputenc}
\usepackage[T1]{fontenc}
\usepackage[english]{babel}
\usepackage{amsmath,amssymb,amsfonts}
\usepackage{xcolor}
\usepackage{tikz}
\usetikzlibrary{calc}
\usepackage[framemethod=TikZ]{mdframed}
\usepackage{titlesec}
\usepackage{lettrine}
\usepackage{eso-pic}
\usepackage{geometry}
\usepackage{multicol}
\usepackage{lmodern}

\geometry{margin=2.0cm}

% advent-layout.tex (corrected for \input usage)

\definecolor{adventred}{HTML}{B3001B}
\definecolor{adventblue}{HTML}{003366}
\definecolor{adventgreen}{HTML}{006633}
\definecolor{adventgold}{HTML}{B59410}

\pagestyle{empty}

\titleformat{\section}
  {\normalfont\large\bfseries\color{adventblue}}{\thesection}{0em}{}
  
\titleformat{\subsection}
  {\normalfont\normalsize\bfseries\color{adventblue}}{\thesubsection}{0em}{}

%----
% AdventFrameTop
%----

\newenvironment{AdventFrameTop}
{%
  \begin{mdframed}[
    linecolor=adventgreen!0,
    linewidth=0pt,
    roundcorner=0pt,
    innertopmargin=10pt,
    innerbottommargin=10pt,
    innerleftmargin=10pt,
    innerrightmargin=10pt,
    backgroundcolor=adventgreen!2
  ]%
}
{%
  \end{mdframed}
}

\newcommand{\BeginAdventPage}{}
\newcommand{\EndAdventPage}{}

%----
% AdventTitleBlock
%----

\newcommand{\AdventTitleBlock}[4]{%
  \begin{center}
    {\Large\textcolor{adventred}{\textbf{#1}}}\par\vspace{4pt}%
    \ifx&#2&\else
      {\large\textbf{#2}}\par\vspace{2pt}%
    \fi
    {\Large\textcolor{adventblue}{\textbf{#3}}}\par
    \ifx&#4&\else
      \vspace{2pt}%
      {\normalsize\textbf{#4}}\par%
    \fi
  \end{center}%
}

%----
% AdventKeyInsight
%----

\newcommand{\AdventKeyInsight}[1]{%
  \vspace{0.5em}%
  \noindent\colorbox{adventred!8}{%
    \parbox{\dimexpr\linewidth-2\fboxsep}{%
    \textbf{\textcolor{adventred}{Key Insight.}}~#1%
    }%
  }%
  \vspace{0.5em}%
}

%----
% AdventStarRule
%----

\newcommand{\AdventStarRule}{%
  \vspace{0.3em}%
  \begin{center}
    {\color{adventgold}%
    \rule[0.5ex]{0.25\linewidth}{0.4pt}\;
    $\ast\;\ast\;\ast$\;
    \rule[0.5ex]{0.25\linewidth}{0.4pt}%
    }%
  \end{center}
  \vspace{0.3em}%
}

%----
% AdventClosing
%----

\newcommand{\AdventClosing}[1]{%
  \vspace{0.4em}%
  \begin{center}
    \textcolor{adventgreen}{\emph{#1}}%
  \end{center}
}

%----
% AdventAuthor
%----

\newcommand{\AdventAuthor}{%
  \AddToShipoutPictureFG{%
    \begin{tikzpicture}[remember picture,overlay]
    \node[anchor=south, yshift=2mm] at (current page.south) {%
    \footnotesize Andreas Müller, Kempten University of Applied Sciences, %
    \texttt{andreas.mueller@hs-kempten.de}%
    };
    \end{tikzpicture}%
  }%
}

%----
% AdventInitial
%----

\newcommand{\AdventInitial}[2]{%
  \lettrine[lines=2,lhang=0.1,loversize=0.15]%
    {\textcolor{adventred}{#1}}%
    {#2}%
}

%----
% AdventPageBackground
%----

\newcommand{\AdventPageBackground}{%
  \AddToShipoutPictureBG{%
    \begin{tikzpicture}[remember picture,overlay]
    \draw[adventgreen!80!black, line width=3pt, rounded corners=12pt]
    ($(current page.north west)+(0.8cm,-0.8cm)$)
    rectangle
    ($(current page.south east)+(-0.8cm,0.8cm)$);
    \fill[adventgold]
    ($(current page.north west)+(1.0cm,-1.0cm)$) circle (1.2pt)
    ($(current page.north east)+(-1.0cm,-1.0cm)$) circle (1.2pt)
    ($(current page.south west)+(1.0cm,1.0cm)$) circle (1.2pt)
    ($(current page.south east)+(-1.0cm,1.0cm)$) circle (1.2pt);
    \end{tikzpicture}
  }%
}

%---- AdventSheet macro (1-page, single column) ----
% #1: Date + occasion
% #2: unused
% #3: Main title
% #4: Subtitle
% #5: Key Insight
% #6: Main body content
% #7: Closing statement
\newcommand{\AdventSheet}[7]{%
  \BeginAdventPage
  \vspace*{1cm}
  \begin{AdventFrameTop}
    \AdventTitleBlock{#1}{#2}{#3}{#4}
    \AdventKeyInsight{#5}
  \end{AdventFrameTop}
  \AdventStarRule
  #6%
  \EndAdventPage
  \AdventClosing{#7}%
}

%---- AdventSheetTwoCol macro (two-column hero sheet) ----
% #1: Date + occasion
% #2: unused
% #3: Main title
% #4: Subtitle
% #5: Key Insight
% #6: Main body content (wrapped in multicols{2})
% #7: Closing statement
\newcommand{\AdventSheetTwoCol}[7]{%
  \BeginAdventPage
  \begin{AdventFrameTop}
    \AdventTitleBlock{#1}{#2}{#3}{#4}
    \AdventKeyInsight{#5}
  \end{AdventFrameTop}
  \AdventStarRule
  \begin{multicols}{2}
    #6%
  \end{multicols}
  \EndAdventPage
  \AdventClosing{#7}%
}

\begin{document}

\AdventPageBackground
\AdventAuthor

\AdventSheetTwoCol
  {December 1, 2025} % #1 Date
  {}                 % #2 unused
  {Quaternions as a prototype: $SU(2)$ and the weak force} % #3 Main title
  {The $1\oplus3$ pattern that scales up to octonions}     % #4 Subtitle
  {Before we climb to the eight-dimensional octonions, it is worth
   pausing at their four-dimensional cousins, the quaternions
   $\mathbb{H}\cong\mathbb{R}^4$. The quaternion units realise a rigid
   $1\oplus3$ split of scalar and vector parts, and their automorphism
   group contains $SU(2)$—the very group of weak isospin. The octonionic
   $8\times8$ action simply doubles this pattern: what $\mathbb{H}$ does
   for $SU(2)$, $\mathbb{O}$ does for the full internal structure of one
   generation.} % #5 Key Insight
  { % #6 body (two columns)

\section*{A gentle introduction: Why quaternions matter for physics}

\AdventInitial{I}{magine} you want to describe the internal structure of
elementary particles—not where they are in space, but the hidden properties
that make a left-handed electron different from a right-handed one, or a
neutrino different from a quark. One natural question is: what kind of
``number system'' could serve as the stage for these internal degrees of
freedom?

The usual real numbers $\mathbb{R}$ give us one dimension: a single number.
Complex numbers $\mathbb{C}$ give us two dimensions and have proven essential
in quantum mechanics. But what if we need more dimensions, and what if we
want something that behaves like multiplication—where combining two elements
gives another element of the same type?

This is where quaternions enter the story. Discovered in 1843 by William
Rowan Hamilton, quaternions extend complex numbers to four dimensions. You
can think of them as having one ``real'' part and three ``imaginary''
parts, which we call $\mathbf{i}$, $\mathbf{j}$, and $\mathbf{k}$. These
three imaginary units are like coordinates in 3D space, but with a special
multiplication rule that makes them noncommutative: $\mathbf{ij} \neq
\mathbf{ji}$.

Why does this matter for particle physics? The key observation is that the
group of symmetries preserving quaternion multiplication contains
$SU(2)$—the exact mathematical structure describing weak nuclear force and
the distinction between left- and right-handed particles. In other words,
weak isospin is not an arbitrary label we invented; it is built into the
structure of a four-dimensional number system.

The split into one real part and three imaginary parts ($1\oplus3$) mirrors
the splitting we see in nature: one neutral component and three charged
components in weak interactions. Before we tackle the full complexity of
eight-dimensional octonions, which host an entire generation of particles,
quaternions give us a familiar ``rehearsal'' of the same pattern at half
the size.

Think of today's sheet as a warm-up exercise: if four dimensions and
quaternions give us $SU(2)$ and the structure of weak interactions, then
eight dimensions and octonions will give us the full Standard Model
structure. Let us see how this works in detail.

\section*{The quaternion stage in four dimensions}

\AdventInitial{T}{he} quaternions
\[
  \mathbb{H} = \{\, a + b\,\mathbf{i} + c\,\mathbf{j} + d\,\mathbf{k}
            \mid a,b,c,d\in\mathbb{R}\,\}
\]
form a four-dimensional division algebra with basis
$1,\mathbf{i},\mathbf{j},\mathbf{k}$ and relations
\[
  \mathbf{i}^2=\mathbf{j}^2=\mathbf{k}^2 = \mathbf{ijk} = -1.
\]

Two structural facts are important for physics:

\begin{itemize}
  \item $\mathbb{H}$ splits as scalar $\oplus$ vector:
        \[
          \mathbb{H} \cong \mathbb{R}\,\oplus\,\mathbb{R}^3,
        \]
        i.e.\ $1\oplus3$.
  \item Left and right multiplication by unit quaternions act as
        rotations on the vector part $\mathbb{R}^3$.
\end{itemize}

This already looks like a toy model for spin and isospin: one distinguished
scalar component, three correlated vector components.

\section*{$SU(2)$ inside the quaternion automorphisms}

Unit quaternions form a group isomorphic to $SU(2)$:
\[
  \{q\in\mathbb{H} \mid |q|=1\} \;\simeq\; SU(2).
\]
Left multiplication by such a unit quaternion acts as an $SU(2)$ matrix on
the $(\mathbf{i},\mathbf{j},\mathbf{k})$ components. Concretely:

\begin{itemize}
  \item The scalar part $a$ stays invariant.
  \item The vector part $(b,c,d)$ transforms as a 3-vector under $SO(3)$,
        double-covered by $SU(2)$.
\end{itemize}

In this language, weak isospin $SU(2)$ is not an abstract label group; it
is \emph{literally} the automorphism group of a number system.

\section*{Action matrices and the $1\oplus3$ pattern}

In the full model, internal degrees of freedom are encoded via action
matrices built from left and right multiplication:
\[
  L_q: x \mapsto qx,\qquad
  R_q: x \mapsto xq.
\]

Already for $\mathbb{H}$, these left/right actions decompose the
four-dimensional real space into a scalar plus a three-dimensional
subspace. In matrix form this appears as a rigid $1\oplus3$ block
structure in the $4\times4$ representation of $L_q$ and $R_q$.

Physically:

\begin{itemize}
  \item The singlet direction (scalar part) is a natural candidate for
        an isospin-neutral component.
  \item The triplet directions (vector part) form a weak isospin
        triplet under $SU(2)$.
\end{itemize}

This is the exact pattern that will be reused and doubled when we move to
the octonionic $8\times8$ actions.

\section*{From quaternions to octonions}

The step from $\mathbb{H}$ to $\mathbb{O}$ looks dramatic—nonassociativity,
seven imaginary units—but in the action-matrix picture it becomes
transparent:

\begin{itemize}
  \item The four real components of $\mathbb{H}$ become eight real
        components for $\mathbb{O}$.
  \item The $1\oplus3$ pattern of scalar/vector parts becomes a richer
        pattern that can host one full generation of internal quantum
        numbers.
  \item The role of $SU(2)\subset\mathrm{Aut}(\mathbb{H})$ is taken over by
        $G_2\subset\mathrm{Aut}(\mathbb{O})$.
\end{itemize}

From this vantage point, $SU(2)$-weak is not an isolated gauge factor; it
is the visible shadow of quaternionic automorphisms inside a larger
octonionic structure.

\section*{Why this detour matters}

The quaternion day serves three purposes in the Advent story:

\begin{enumerate}
  \item \textbf{Familiarity:} it connects the exotic octonionic picture
        to the well-known role of $SU(2)$ in spin and weak isospin.
  \item \textbf{Pattern recognition:} it highlights the $1\oplus3$
        structure that later scales up to the more intricate block
        structure of the octonionic action.
  \item \textbf{Continuity:} it shows that the leap to $\mathbb{O}$ is a
        continuation of a number-theoretic line, not a wild guess.
\end{enumerate}

After this quaternion warm-up, the following days return to the full
octonionic stage, now with a clearer sense of where $SU(2)$ ``comes from''
in the algebraic background.

\small
\begin{thebibliography}{9}

\bibitem{ConwaySmith2003}
J.~H.~Conway and D.~A.~Smith,
\newblock {\em On Quaternions and Octonions},
\newblock A.K.~Peters, 2003.

\bibitem{Hamilton1844}
W.~R.~Hamilton,
\newblock ``On quaternions; or on a new system of imaginaries in algebra,''
\newblock {\em Philos.\ Mag.} \textbf{25}, 489--495 (1844).

\end{thebibliography}
\normalsize

  } % end #6 body
  {Quaternions provide a four-dimensional rehearsal: a rigid $1\oplus3$
   pattern and an $SU(2)$ of unit quaternions that foreshadow the full
   octonionic stage of one generation.} % #7 Closing

\end{document}